\chapter{数学知识}


\section{质数}

\subsection{AcWing 866. 试除法判定质数}
\begin{titledbox}{AcWing 866. 试除法判定质数}
    给定 $n$ 个正整数 $a_i$,判定每个数是否是质数。

    输入格式:

    第一行包含整数 $n$。 接下来 $n$ 行,每行包含一个正整数 $a_i$。

    输出格式:

    共 $n$ 行,其中第 $i$ 行输出第 $i$ 个正整数 $a_i$ 是否为质数,是则输出 \inlinecode{Yes},否则输出 \inlinecode{No}。

    数据范围:

    $1 \le n \le 100$, $1 \le a_i \le 2^{31}-1$

    \begin{inputblock}
        \inlinecode{2} \\
        \inlinecode{2} \\
        \inlinecode{6}
    \end{inputblock}
    \begin{outputblock}
        \inlinecode{Yes} \\
        \inlinecode{No}
    \end{outputblock}
\end{titledbox}

\subsection{AcWing 867. 分解质因数}
\begin{titledbox}{AcWing 867. 分解质因数}
    给定 $n$ 个正整数 $a_i$,将每个数分解质因数,并按照质因数从小到大的顺序输出每个质因数的底数和指数。

    输入格式:

    第一行包含整数 $n$。 接下来 $n$ 行,每行包含一个正整数 $a_i$。

    输出格式:

    对于每个正整数 $a_i$,按照从小到大的顺序输出其分解质因数后,每个质因数的底数和指数,每个底数和指数占一行。 每个正整数的质因数全部输出完毕后,输出一个空行。

    数据范围:

    $1 \le n \le 100$, $1 \le a_i \le 2 \times 10^9$

    \begin{inputblock}
        \inlinecode{2} \\
        \inlinecode{6} \\
        \inlinecode{8}
    \end{inputblock}
    \begin{outputblock}
        \inlinecode{2 1} \\
        \inlinecode{3 1} \\
        \\
        \inlinecode{2 3} \\

    \end{outputblock}
\end{titledbox}

\subsection{AcWing 868. 筛质数}
\begin{titledbox}{AcWing 868. 筛质数}
    给定一个正整数 $n$,请你求出 $1 \sim n$ 中质数的个数。

    输入格式:

    共一行,包含整数 $n$。

    输出格式:

    共一行,包含一个整数,表示 $1 \sim n$ 中质数的个数。

    数据范围:

    $1 \le n \le 10^6$

    \begin{inputblock}
        \inlinecode{8}
    \end{inputblock}
    \begin{outputblock}
        \inlinecode{4}
    \end{outputblock}
\end{titledbox}


\section{约数}

\subsection{AcWing 869. 试除法求约数}
\begin{titledbox}{AcWing 869. 试除法求约数}
    给定 $n$ 个正整数 $a_i$,对于每个整数 $a_i$,请你按照从小到大的顺序输出它的所有约数。

    输入格式:

    第一行包含整数 $n$。 接下来 $n$ 行,每行包含一个整数 $a_i$。

    输出格式:

    输出共 $n$ 行,其中第 $i$ 行输出第 $i$ 个整数 $a_i$ 的所有约数。

    数据范围:

    $1 \le n \le 100$, $2 \le a_i \le 2 \times 10^9$

    \begin{inputblock}
        \inlinecode{2} \\
        \inlinecode{6} \\
        \inlinecode{8}
    \end{inputblock}
    \begin{outputblock}
        \inlinecode{1 2 3 6} \\
        \inlinecode{1 2 4 8}
    \end{outputblock}
\end{titledbox}

\subsection{AcWing 870. 约数个数}

\begin{titledbox}{AcWing 870. 约数个数}
    给定 $n$ 个正整数 $a_i$,请你输出这些数的乘积的约数个数,答案对 $10^9+7$ 取模。

    输入格式:

    第一行包含整数 $n$。 接下来 $n$ 行,每行包含一个整数 $a_i$。

    输出格式:

    输出一个整数,表示所给正整数的乘积的约数个数,答案需对 $10^9+7$ 取模。

    数据范围:

    $1 \le n \le 100$, $1 \le a_i \le 2 \times 10^9$

    \begin{inputblock}
        \inlinecode{3} \\
        \inlinecode{2} \\
        \inlinecode{6} \\
        \inlinecode{8}
    \end{inputblock}
    \begin{outputblock}
        \inlinecode{12}
    \end{outputblock}
\end{titledbox}

\subsection{AcWing 871. 约数之和}
\begin{titledbox}{AcWing 871. 约数之和}
    给定 $n$ 个正整数 $a_i$,请你输出这些数的乘积的约数之和,答案对 $10^9+7$ 取模。

    输入格式:

    第一行包含整数 $n$。 接下来 $n$ 行,每行包含一个整数 $a_i$。

    输出格式:

    输出一个整数,表示所给正整数的乘积的约数之和,答案需对 $10^9+7$ 取模。

    数据范围:

    $1 \le n \le 100$, $1 \le a_i \le 2 \times 10^9$

    \begin{inputblock}
        \inlinecode{3} \\
        \inlinecode{2} \\
        \inlinecode{6} \\
        \inlinecode{8}
    \end{inputblock}
    \begin{outputblock}
        \inlinecode{252}
    \end{outputblock}
\end{titledbox}

\subsection{AcWing 872. 最大公约数}
\begin{titledbox}
    给定 $n$ 对正整数 $a_i, b_i$,请你求出每对数的最大公约数。

    输入格式:

    第一行包含整数 $n$。 接下来 $n$ 行,每行包含一个整数对 $a_i,b_i$。

    输出格式:

    输出共 $n$ 行,每行输出一个整数对的最大公约数。

    数据范围:

    $1 \le n \le 10^5$, $1 \le a_i, b_i \le 2 \times 10^9$

    \begin{inputblock}
        \inlinecode{2} \\
        \inlinecode{3 6} \\
        \inlinecode{4 6}
    \end{inputblock}
    \begin{outputblock}
        \inlinecode{3} \\
        \inlinecode{2}
    \end{outputblock}
\end{titledbox}


\section{欧拉函数}

\subsection{AcWing 873. 欧拉函数}
\begin{titledbox}{AcWing 873. 欧拉函数}
    给定 $n$ 个正整数 $a_i$,请你求出每个数的欧拉函数。

    欧拉函数的定义:

    \begin{quote}
        $1 \sim N$ 中与 $N$ 互质的数的个数被称为欧拉函数,记为 $\varphi(N)$。

        若在算数基本定理中,$N = p_1^{a_1}p_2^{a_2}\dots p_m^{a_m}$,则:

        $\varphi (N)$ = $N \times \frac{p_1-1}{p_1} \times \frac{p_2-1}{p_2} \times \dots \times \frac{p_m-1}{p_m}$
    \end{quote}

    输入格式:

    第一行包含整数 $n$。 接下来 $n$ 行,每行包含一个正整数 $a_i$。

    输出格式:

    输出共 $n$ 行,每行输出一个正整数 $a_i$ 的欧拉函数。

    数据范围:

    $1 \le n \le 100$, $1 \le a_i \le 2 \times 10^9$

    \begin{inputblock}
        \inlinecode{3} \\
        \inlinecode{3} \\
        \inlinecode{6} \\
        \inlinecode{8}
    \end{inputblock}
    \begin{outputblock}
        \inlinecode{2} \\
        \inlinecode{2} \\
        \inlinecode{4}
    \end{outputblock}
\end{titledbox}

\subsection{AcWing 874. 筛法求欧拉函数}
\begin{titledbox}{AcWing 874. 筛法求欧拉函数}
    给定一个正整数 $n$,求 $1 \sim n$ 中每个数的欧拉函数之和。

    输入格式:

    共一行,包含一个整数 $n$。

    输出格式:

    共一行,包含一个整数,表示 $1 \sim n$ 中每个数的欧拉函数之和。

    数据范围:

    $1 \le n \le 10^6$

    \begin{inputblock}
        \inlinecode{6}
    \end{inputblock}
    \begin{outputblock}
        \inlinecode{12}
    \end{outputblock}
\end{titledbox}


\section{快速幂}

\subsection{AcWing 875. 快速幂}

\subsection{AcWing 876. 快速幂求逆元}


\section{扩展欧几里得算法}

\subsection{AcWing 877. 扩展欧几里得算法}

\subsection{AcWing 878. 线性同余方程}


\section{中国剩余定理}

\subsection{AcWing 204. 表达整数的奇怪方式}


\section{高斯消元}

\subsection{AcWing 883. 高斯消元解线性方程组}

\subsection{AcWing 884. 高斯消元解异或线性方程组}


\section{求组合数}

\subsection{AcWing 885. 求组合数 I}

\subsection{AcWing 886. 求组合数 II}

\subsection{AcWing 887. 求组合数 III}

\subsection{AcWing 888. 求组合数 IV}

\subsection{AcWing 889. 满足条件的01序列}


\section{容斥原理}

\subsection{AcWing 890. 能被整除的数}


\section{博弈论}

\subsection{AcWing 891. Nim游戏}

\subsection{AcWing 892. 台阶-Nim游戏}

\subsection{AcWing 893. 集合-Nim游戏}

\subsection{AcWing 894. 拆分-Nim游戏}
