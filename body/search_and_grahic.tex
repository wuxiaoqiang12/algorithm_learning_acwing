\chapter{搜索和图论}

稠密图:邻接矩阵 \lstinline{g[N][N]}, a和b之间是否有边,且边的权重是多少

稀疏图:邻接表 \lstinline{h[N], e[N], ne[N], idx}, 和拉链法的哈希是一样的数据结构

\section{DFS}
\subsection{AcWing 842. 排列数字}
\subsection{AcWing 843. n-皇后问题}

有两种解法,第一种:因发现每一行每一列有且仅有一个皇后,可以枚举每一行,来看看可以将皇后放置在第几列。

\begin{mycpptwocol}[n-皇后问题:解法一]
#include <stdio.h>
#include <stdlib.h>
#include <string.h>
#include <stdbool.h>

#define N 20

char g[N][N];
bool col[N];
bool dg[N];
bool udg[N];

void dfs(int u, int n)
{
    if (u == n) {
        for (int i = 0; i < n; i++) {
            puts(g[i]);
        }
        printf("\n");
        return;
    }
    for (int i = 0; i < n; i++) {
        if (col[i] || dg[u + i] || udg[u - i + n]) {
            continue;
        }
        g[u][i] = 'Q';
        col[i] = dg[u + i] = udg[u - i + n] = true;
        dfs(u + 1, n);
        g[u][i] = '.';
        col[i] = dg[u + i] = udg[u - i + n] = false;
    }
}

int main()
{
    int n;
    scanf("%d", &n);
    for (int i = 0; i < n; i++) {
        for (int j = 0; j < n; j++) {
            g[i][j] = '.';
        }
    }
    dfs(0,  n);
    return 0;
}
\end{mycpptwocol}

解法二:从左上角开始枚举每一个格子,每个格子有两种状态(放或者不放),在枚举过程中检查每一个格子是否可以继续放置。若皇后数量已经达到最大数量则找到了一个方案

\begin{mycpptwocol}[n-皇后问题:解法二]
#include <stdio.h>
#include <stdlib.h>
#include <string.h>
#include <stdbool.h>

#define N 20

char g[N][N];
bool row[N];
bool col[N];
bool dg[N];
bool udg[N];

void dfs(int x, int y, int s, int n)
{
    if (y == n) {
        y = 0;
        x++;
    }
    if (x == n) {
        if (s == n) {
            for (int i = 0; i < n; i++) {
                puts(g[i]);
            }
            puts("");
        }
        return;
    }
    // do not put queen here
    dfs(x, y + 1, s, n);

    // put queen here
    if (!row[x] && !col[y] && !dg[x + y] && !udg[x - y + n]) {
        g[x][y] = 'Q';
        row[x] = col[y] = dg[x + y] = udg[x - y + n] = true;
        dfs(x, y + 1, s + 1, n);
        g[x][y] = '.';
        row[x] = col[y] = dg[x + y] = udg[x - y + n] = false;
    }
}

int main()
{
    int n;
    scanf("%d", &n);
    for (int i = 0; i < n; i++) {
        for (int j = 0; j < n; j++) {
            g[i][j] = '.';
        }
    }
    dfs(0, 0, 0, n);
    return 0;
}
\end{mycpptwocol}

\section{BFS}
\subsection{AcWing 844. 走迷宫}
\subsection{AcWing 845. 八数码}

\section{树与图的深度优先遍历}
\subsection{AcWing 846. 树的重心}

\section{树与图的广度优先遍历}
\subsection{AcWing 847. 图中点的层次}

\section{拓扑排序}
\subsection{AcWing 848. 有向图的拓扑序列}

\section{Dijkstra}
\subsection{AcWing 849. Dijkstra求最短路 I}
\subsection{AcWing 850. Dijkstra求最短路 II}

\section{bellman-ford}
\subsection{AcWing 853. 有边数限制的最短路}

\section{spfa}
\subsection{AcWing 851. spfa求最短路}
\subsection{AcWing 852. spfa判断负环}

\section{Floyd}
\subsection{AcWing 854. Floyd求最短路}

\section{Prim}
\subsection{AcWing 858. Prim算法求最小生成树}

\section{Kruskal}
\subsection{AcWing 859. Kruskal算法求最小生成树}

\section{染色法判定二分图}
\subsection{AcWing 860. 染色法判定二分图}

\section{匈牙利算法}
\subsection{AcWing 861. 二分图的最大匹配}
