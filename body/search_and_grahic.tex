\chapter{搜索和图论}

稠密图:邻接矩阵 \lstinline{g[N][N]}, a和b之间是否有边,且边的权重是多少

稀疏图:邻接表 \lstinline{h[N], e[N], ne[N], idx}, 和拉链法的哈希是一样的数据结构

\section{DFS}
\subsection{AcWing 842. 排列数字}
\subsection{AcWing 843. n-皇后问题}

\section{BFS}
\subsection{AcWing 844. 走迷宫}
\subsection{AcWing 845. 八数码}

\section{树与图的深度优先遍历}
\subsection{AcWing 846. 树的重心}

\section{树与图的广度优先遍历}
\subsection{AcWing 847. 图中点的层次}

\section{拓扑排序}
\subsection{AcWing 848. 有向图的拓扑序列}

\section{Dijkstra}
\subsection{AcWing 849. Dijkstra求最短路 I}
\subsection{AcWing 850. Dijkstra求最短路 II}

\section{bellman-ford}
\subsection{AcWing 853. 有边数限制的最短路}

\section{spfa}
\subsection{AcWing 851. spfa求最短路}
\subsection{AcWing 852. spfa判断负环}

\section{Floyd}
\subsection{AcWing 854. Floyd求最短路}

\section{Prim}
\subsection{AcWing 858. Prim算法求最小生成树}

\section{Kruskal}
\subsection{AcWing 859. Kruskal算法求最小生成树}

\section{染色法判定二分图}
\subsection{AcWing 860. 染色法判定二分图}

\section{匈牙利算法}
\subsection{AcWing 861. 二分图的最大匹配}