\chapter{贪心算法}


\section{区间问题}

\subsection{AcWing 905. 区间选点}
\begin{titledbox}{AcWing 905. 区间选点}
    给定 $N$ 个闭区间 $[a_i,b_i]$,请你在数轴上选择尽量少的点,使得每个区间内至少包含一个选出的点。输出选择的点的最小数量。 位于区间端点上的点也算作区间内。

    输入格式:

    第一行包含整数 $N$,表示区间数。 接下来 $N$ 行,每行包含两个整数 $a_i,b_i$,表示一个区间的两个端点。

    输出格式:

    输出一个整数,表示所需的点的最小数量。

    数据范围:

    $1 \le N \le 10^5$, $-10^9 \le a_i \le b_i \le 10^9$

    \begin{inputblock}
        \inlinecode{3} \\
        \inlinecode{-1 1} \\
        \inlinecode{2 4} \\
        \inlinecode{3 5}
    \end{inputblock}
    \begin{outputblock}
        \inlinecode{2}
    \end{outputblock}
\end{titledbox}

\subsection{AcWing 908. 最大不相交区间数量}
\begin{titledbox}{AcWing 908. 最大不相交区间数量}
    给定 $N$ 个闭区间 $[a_i,b_i]$,请你在数轴上选择若干区间,使得选中的区间之间互不相交(包括端点)。输出可选取区间的最大数量。

    输入格式:

    第一行包含整数 $N$,表示区间数。接下来 $N$ 行,每行包含两个整数 $a_i,b_i$,表示一个区间的两个端点。

    输出格式:

    输出一个整数,表示可选取区间的最大数量。

    数据范围:

    $1 \le N \le 10^5$, $-10^9 \le a_i \le b_i \le 10^9$

    \begin{inputblock}
        \inlinecode{3} \\
        \inlinecode{-1 1} \\
        \inlinecode{2 4} \\
        \inlinecode{3 5}
    \end{inputblock}
    \begin{outputblock}
        \inlinecode{2}
    \end{outputblock}
\end{titledbox}

\subsection{AcWing 906. 区间分组}
\begin{titledbox}{AcWing 906. 区间分组}
    给定 $N$ 个闭区间 $[a_i,b_i]$,请你将这些区间分成若干组,使得每组内部的区间两两之间(包括端点)没有交集,并使得组数尽可能小。输出最小组数。

    输入格式:

    第一行包含整数 $N$,表示区间数。接下来 $N$ 行,每行包含两个整数 $a_i,b_i$,表示一个区间的两个端点。

    输出格式:

    输出一个整数,表示最小组数。

    数据范围:

    $1 \le N \le 10^5$, $-10^9 \le a_i \le b_i \le 10^9$

    \begin{inputblock}
        \inlinecode{3} \\
        \inlinecode{-1 1} \\
        \inlinecode{2 4} \\
        \inlinecode{3 5}
    \end{inputblock}
    \begin{outputblock}
        \inlinecode{2}
    \end{outputblock}
\end{titledbox}

\subsection{AcWing 907. 区间覆盖}
\begin{titledbox}{AcWing 907. 区间覆盖}
    给定 $N$ 个闭区间 $[a_i,b_i]$ 以及一个线段区间 $[s,t]$,请你选择尽量少的区间,将指定线段区间完全覆盖。 输出最少区间数,如果无法完全覆盖则输出 $-1$。

    输入格式:

    第一行包含两个整数 $s$ 和 $t$,表示给定线段区间的两个端点。第二行包含整数 $N$,表示给定区间数。 接下来 $N$ 行,每行包含两个整数 $a_i,b_i$,表示一个区间的两个端点。

    输出格式:

    输出一个整数,表示所需最少区间数。 如果无解,则输出 $-1$。

    数据范围:

    $1 \le N \le 10^5$, $-10^9 \le a_i \le b_i \le 10^9$, $-10^9 \le s \le t \le 10^9$

    \begin{inputblock}
        \inlinecode{3} \\
        \inlinecode{-1 1} \\
        \inlinecode{2 4} \\
        \inlinecode{3 5}
    \end{inputblock}
    \begin{outputblock}
        \inlinecode{2}
    \end{outputblock}
\end{titledbox}


\section{Huffman树}

\subsection{AcWing 148. 合并果子}
\begin{titledbox}{AcWing 148. 合并果子}
    在一个果园里,达达已经将所有的果子打了下来,而且按果子的不同种类分成了不同的堆。 达达决定把所有的果子合成一堆。 每一次合并,达达可以把两堆果子合并到一起,消耗的体力等于两堆果子的重量之和。 可以看出,所有的果子经过 $n-1$ 次合并之后,就只剩下一堆了。 达达在合并果子时总共消耗的体力等于每次合并所耗体力之和。 因为还要花大力气把这些果子搬回家,所以达达在合并果子时要尽可能地节省体力。 假定每个果子重量都为 $1$,并且已知果子的种类数和每种果子的数目,你的任务是设计出合并的次序方案,使达达耗费的体力最少,并输出这个最小的体力耗费值。

    例如有 $3$ 种果子,数目依次为 $1,2,9$。可以先将 $1、2$ 堆合并,新堆数目为 $3$,耗费体力为 $3$。接着,将新堆与原先的第三堆合并,又得到新的堆,数目为 $12$,耗费体力为 $12$。所以达达总共耗费体力$=3+12=15$。可以证明 $15$ 为最小的体力耗费值。

    输入格式:

    输入包括两行,第一行是一个整数 $n$,表示果子的种类数。第二行包含 $n$ 个整数,用空格分隔,第 $i$ 个整数 $a_i$ 是第 $i$ 种果子的数目。

    输出格式:

    输出包括一行,这一行只包含一个整数,也就是最小的体力耗费值。输入数据保证这个值小于 $2^{31}$。

    数据范围:

    $1 \le n \le 10000$, $1 \le a_i \le 20000$

    \begin{inputblock}
        \inlinecode{3} \\
        \inlinecode{1 2 9}
    \end{inputblock}
    \begin{outputblock}
        \inlinecode{15}
    \end{outputblock}
\end{titledbox}


\section{排序不等式}

\subsection{AcWing 913. 排队打水}
\begin{titledbox}{AcWing 913. 排队打水}
    有 $n$ 个人排队到 $1$ 个水龙头处打水,第 $i$ 个人装满水桶所需的时间是 $t_i$,请问如何安排他们的打水顺序才能使所有人的等待时间之和最小?

    输入格式:

    第一行包含整数 $n$。第二行包含 $n$ 个整数,其中第 $i$ 个整数表示第 $i$ 个人装满水桶所花费的时间 $t_i$。

    输出格式:

    输出一个整数,表示最小的等待时间之和。

    数据范围:

    $1 \le n \le 10^5$, $1 \le t_i \le 10^4$

    \begin{inputblock}
        \inlinecode{7} \\
        \inlinecode{3 6 1 4 2 5 7}
    \end{inputblock}
    \begin{outputblock}
        \inlinecode{56}
    \end{outputblock}
\end{titledbox}


\section{绝对值不等式}

\subsection{AcWing 104. 货仓选址}
\begin{titledbox}{AcWing 104. 货仓选址}
    在一条数轴上有 $N$ 家商店,它们的坐标分别为 $A_1 \sim A_N$。现在需要在数轴上建立一家货仓,每天清晨,从货仓到每家商店都要运送一车商品。 为了提高效率,求把货仓建在何处,可以使得货仓到每家商店的距离之和最小。

    输入格式:

    第一行输入整数 $N$。第二行 $N$ 个整数 $A_1 \sim A_N$。

    输出格式:

    输出一个整数,表示距离之和的最小值。

    数据范围:

    $1 \le N \le 100000$, $0 \le A_i \le 40000$

    \begin{inputblock}
        \inlinecode{4} \\
        \inlinecode{6 2 9 1}
    \end{inputblock}
    \begin{outputblock}
        \inlinecode{12}
    \end{outputblock}
\end{titledbox}


\section{推公式}

\subsection{AcWing 125. 耍杂技的牛}
\begin{titledbox}{AcWing 125. 耍杂技的牛}
    农民约翰的 $N$ 头奶牛(编号为 $1..N$)计划逃跑并加入马戏团,为此它们决定练习表演杂技。奶牛们不是非常有创意,只提出了一个杂技表演: 叠罗汉,表演时,奶牛们站在彼此的身上,形成一个高高的垂直堆叠。奶牛们正在试图找到自己在这个堆叠中应该所处的位置顺序。这 $N$ 头奶牛中的每一头都有着自己的重量 $W_i$ 以及自己的强壮程度 $S_i$。一头牛支撑不住的可能性取决于它头上所有牛的总重量(不包括它自己)减去它的身体强壮程度的值,现在称该数值为风险值,风险值越大,这只牛撑不住的可能性越高。您的任务是确定奶牛的排序,使得所有奶牛的风险值中的最大值尽可能的小。

    输入格式:

    第一行输入整数 $N$,表示奶牛数量。接下来 $N$ 行,每行输入两个整数,表示牛的重量和强壮程度,第 $i$ 行表示第 $i$ 头牛的重量 $W_i$ 以及它的强壮程度 $S_i$。

    输出格式:

    输出一个整数,表示最大风险值的最小可能值。

    数据范围:

    $1 \le N \le 50000$, $1 \le W_i \le 10,000$, $1 \le S_i \le 1,000,000,000$

    \begin{inputblock}
        \inlinecode{3} \\
        \inlinecode{10 3} \\
        \inlinecode{2 5} \\
        \inlinecode{3 3}
    \end{inputblock}
    \begin{outputblock}
        \inlinecode{2}
    \end{outputblock}
\end{titledbox}
