\chapter{数据结构}


\section{单链表}

\subsection{AcWing 826. 单链表}

\begin{titledbox}{AcWing 826. 单链表}
    实现一个单链表,链表初始为空,支持三种操作:
    \begin{myenum}
        \item 向链表头插入一个数;
        \item 删除第 $k$ 个插入的数后面的数;
        \item 在第 $k$ 个插入的数后插入一个数。
    \end{myenum}

    现在要对该链表进行 $M$ 次操作,进行完所有操作后,从头到尾输出整个链表。
    \textbf{注意}: 题目中第 $k$ 个插入的数并不是指当前链表的第 $k$ 个数。例如操作过程中一共插入了 $n$ 个数,则按照插入的时间顺序,这 $n$ 个数依次为:第 $1$ 个插入的数,第 $2$ 个插入的数,$\dots$第 $n$ 个插入的数。

    输入格式:

    第一行包含整数 $M$,表示操作次数。接下来 $M$ 行,每行包含一个操作命令,操作命令可能为以下几种:

    \begin{myenum}
        \item \inlinecode{H x},表示向链表头插入一个数 $x$。
        \item \inlinecode{D k},表示删除第 $k$ 个插入的数后面的数(当 $k$ 为 $0$ 时,表示删除头结点)。
        \item \inlinecode{I k x},表示在第 $k$ 个插入的数后面插入一个数 $x$(此操作中 $k$ 均大于 $0$)。
    \end{myenum}

    输出格式:

    共一行,将整个链表从头到尾输出。

    数据范围:

    $1 \le M \le 100000$ 所有操作保证合法。

    \begin{inputblock}
        \inlinecode{10} \\
        \inlinecode{H 9} \\
        \inlinecode{I 1 1} \\
        \inlinecode{D 1} \\
        \inlinecode{D 0} \\
        \inlinecode{H 6} \\
        \inlinecode{I 3 6} \\
        \inlinecode{I 4 5} \\
        \inlinecode{I 4 5} \\
        \inlinecode{I 3 4} \\
        \inlinecode{D 6}
    \end{inputblock}
    \begin{outputblock}
        \inlinecode{6 4 6 5}
    \end{outputblock}
\end{titledbox}

\begin{mycpptwocol}[linked list]
    #include <stdio.h>
    #include <stdlib.h>
    #include <stdbool.h>

    #define N 100010
    int idx; // 标识当前用到了哪个节点
    int head; // 头节点指向的元素
    int e[N]; // 存储节点的值
    int ne[N]; // 存储节点的next指针

    // 初始化
    void Init() {
        idx = -1;
        head = -1;
    }

    // 在头节点后插入元素
    void AddHead(int x) {
        e[++idx] = x;
        ne[idx] = head;
        head = idx;
    }

    // 在k节点之后插入元素
    void Add(int k, int x) {
        e[++idx] = x;
        ne[idx] = ne[k];
        ne[k] = idx;
    }

    // 删除k之后的那个元素
    void Remove(int k) {
        ne[k] = ne[ne[k]];
    }

    int main() {
        Init();
        int n;
        scanf("%d", &n);
        while (n--) {
            int x;
            int k;
            char op;
            scanf(" %c", &op);
            if (op == 'H') {
                scanf("%d", &x);
                AddHead(x);
            }
            if (op == 'I') {
                scanf("%d %d", &k, &x);
                Add(k - 1, x);
            }
            if (op == 'D') {
                scanf("%d", &k);
                if (k == 0) {
                    head = ne[head];
                }
                Remove(k - 1);
            }
        }

        int tmp = head;
        while (tmp != -1) {
            printf("%d ", e[tmp]);
            tmp = ne[tmp];
        }
        return 0;
    }
\end{mycpptwocol}


\section{双链表}

\subsection{AcWing 827. 双链表}
\begin{titledbox}{AcWing 827. 双链表}
    实现一个双链表,双链表初始为空,支持 $5$ 种操作:

    \begin{myenum}
        \item 在最左侧插入一个数;
        \item 在最右侧插入一个数;
        \item 将第 $k$ 个插入的数删除;
        \item 在第 $k$ 个插入的数左侧插入一个数;
        \item 在第 $k$ 个插入的数右侧插入一个数。
    \end{myenum}

    现在要对该链表进行 $M$ 次操作,进行完所有操作后,从左到右输出整个链表。\textbf{注意}: 题目中第 $k$ 个插入的数并不是指当前链表的第 $k$ 个数。例如操作过程中一共插入了 $n$ 个数,则按照插入的时间顺序,这 $n$ 个数依次为:第 $1$ 个插入的数,第 $2$ 个插入的数,…第 $n$ 个插入的数。

    输入格式:

    第一行包含整数 $M$,表示操作次数。接下来 $M$ 行,每行包含一个操作命令,操作命令可能为以下几种:

    \begin{myenum}
        \item \inlinecode{L x},表示在链表的最左端插入数 $x$。
        \item \inlinecode{R x},表示在链表的最右端插入数 $x$。
        \item \inlinecode{D k},表示将第 $k$ 个插入的数删除。
        \item \inlinecode{IL k x},表示在第 $k$ 个插入的数左侧插入一个数。
        \item \inlinecode{IR k x},表示在第 $k$ 个插入的数右侧插入一个数。
    \end{myenum}

    输出格式:
    一行,将整个链表从左到右输出

    数据范围:

    $1 \le M \le 100000$ 所有操作保证合法。

    \begin{inputblock}
        \inlinecode{10} \\
        \inlinecode{R 7} \\
        \inlinecode{D 1} \\
        \inlinecode{L 3} \\
        \inlinecode{IL 2 10} \\
        \inlinecode{D 3} \\
        \inlinecode{IL 2 7} \\
        \inlinecode{L 8} \\
        \inlinecode{R 9} \\
        \inlinecode{IL 4 7} \\
        \inlinecode{IR 2 2}
    \end{inputblock}
    \begin{outputblock}
        \inlinecode{8 7 7 3 2 9}
    \end{outputblock}
\end{titledbox}

\begin{mycpptwocol}[双链表]
    #include <stdio.h>
    #include <stdlib.h>

    #define N 100010

    int e[N];
    int l[N];
    int r[N];
    int idx = 1;

    void Init() {
        r[0] = 1;
        l[1] = 0;
        idx = 1;
    }

    void AddRight(int k, int x) {
        e[++idx] = x;
        r[idx] = r[k];
        l[r[k]] = idx;
        l[idx] = k;
        r[k] = idx;
    }

    void Remove(int k) {
        r[l[k]] = r[k];
        l[r[k]] = l[k];
    }

    int main() {
        Init();
        int n;
        scanf("%d", &n);
        while (n--) {
            char op[3];
            scanf("%s", op);
            int k;
            int x;
            if (op[0] == 'R') {
                scanf("%d", &x);
                AddRight(l[1], x);
            }
            if (op[0] == 'D') {
                scanf("%d", &k);
                Remove(k + 1);
            }
            if (op[0] == 'L') {
                scanf("%d", &x);
                AddRight(0, x);
            }
            if (op[0] == 'I') {
                scanf("%d %d", &k, &x);
                if (op[1] == 'L') {
                    AddRight(l[k + 1], x);
                } else {
                    AddRight(k + 1, x);
                }
            }
        }
        for (int i = r[0]; i != 1; i = r[i]) {
            printf("%d ", e[i]);
        }
        return 0;
    }
\end{mycpptwocol}


\section{栈}

\subsection{AcWing 828. 模拟栈}
\begin{titledbox}{AcWing 828. 模拟栈}
    实现一个栈,栈初始为空,支持四种操作:

    \begin{myenum}
        \item \inlinecode{push x} - 向栈顶插入一个数 $x$;
        \item \inlinecode{pop} - 从栈顶弹出一个数;
        \item \inlinecode{empty} - 判断栈是否为空;
        \item \inlinecode{query} - 查询栈顶元素
    \end{myenum}

    现在要对栈进行 $M$ 个操作,其中的每个操作 $3$ 和操作 $4$ 都要输出相应的结果。

    输入格式:

    第一行包含整数 $M$,表示操作次数。接下来 $M$ 行,每行包含一个操作命令,操作命令为 \inlinecode{push x},\inlinecode{pop},\inlinecode{empty},\inlinecode{query} 中的一种。

    输出格式:

    对于每个 \inlinecode{empty} 和 \inlinecode{query} 操作都要输出一个查询结果,每个结果占一行。其中,\inlinecode{empty} 操作的查询结果为 \inlinecode{YES} 或 \inlinecode{NO},\inlinecode{query} 操作的查询结果为一个整数,表示栈顶元素的值。

    数据范围:

    $1 \le M \le 100000$, $1 \le x \le 10^9$ 所有操作保证合法。

    \begin{inputblock}
        \inlinecode{10} \\
        \inlinecode{push 5} \\
        \inlinecode{query} \\
        \inlinecode{push 6} \\
        \inlinecode{pop} \\
        \inlinecode{query} \\
        \inlinecode{pop } \\
        \inlinecode{empty} \\
        \inlinecode{push 4} \\
        \inlinecode{query} \\
        \inlinecode{empty}
    \end{inputblock}
    \begin{outputblock}
        \inlinecode{5} \\
        \inlinecode{5} \\
        \inlinecode{YES} \\
        \inlinecode{4} \\
        \inlinecode{NO}
    \end{outputblock}
\end{titledbox}

\begin{mycpptwocol}[模拟栈]
    #include <stdio.h>
    #include <stdlib.h>

    int main() {
        int n;
        scanf("%d", &n);
        int *stack = (int *)calloc(n, sizeof(int));
        int top = 0;
        while (n--) {
            char op[10];
            scanf("%s", op);
            int x;
            if (strcmp(op, "push") == 0) {
                scanf("%d", &x);
                stack[top++] = x;
            }
            if (strcmp(op, "query") == 0) {
                printf("%d\n", stack[top - 1]);
            }
            if (strcmp(op, "pop") == 0) {
                top--;
            }
            if (strcmp(op, "empty") == 0) {
                printf("%s\n", top == 0 ? "YES" : "NO");
            }
        }
        free(stack);
        return 0;
    }
\end{mycpptwocol}

\subsection{AcWing 3302. 表达式求值}


\section{队列}

\subsection{AcWing 829. 模拟队列}

\begin{titledbox}{AcWing 829. 模拟队列}
    实现一个队列,队列初始为空,支持四种操作:

    \begin{myenum}
        \item \inlinecode{push x} - 向队尾插入一个数 $x$;
        \item \inlinecode{pop} - 从队头弹出一个数;
        \item \inlinecode{empty} - 判断队列是否为空;
        \item \inlinecode{query} - 查询队头元素
    \end{myenum}

    现在要对队列进行 $M$ 个操作,其中的每个操作 $3$ 和操作 $4$ 都要输出相应的结果。

    输入格式:

    第一行包含整数 $M$,表示操作次数。接下来 $M$ 行,每行包含一个操作命令,操作命令为 \inlinecode{push x},\inlinecode{pop},\inlinecode{empty},\inlinecode{query} 中的一种。

    输出格式:

    对于每个 \inlinecode{empty} 和 \inlinecode{query} 操作都要输出一个查询结果,每个结果占一行。其中,\inlinecode{empty} 操作的查询结果为 \inlinecode{YES} 或 \inlinecode{NO},\inlinecode{query} 操作的查询结果为一个整数,表示栈顶元素的值。

    数据范围:

    $1 \le M \le 100000$, $1 \le x \le 10^9$ 所有操作保证合法。

    \begin{inputblock}
        \inlinecode{10} \\
        \inlinecode{push 6} \\
        \inlinecode{empty} \\
        \inlinecode{query} \\
        \inlinecode{pop} \\
        \inlinecode{empty} \\
        \inlinecode{push 3} \\
        \inlinecode{push 4} \\
        \inlinecode{pop} \\
        \inlinecode{query} \\
        \inlinecode{push 6}
    \end{inputblock}
    \begin{outputblock}
        \inlinecode{NO} \\
        \inlinecode{YES} \\
        \inlinecode{6} \\
        \inlinecode{4}
    \end{outputblock}
\end{titledbox}

\begin{mycpptwocol}[模拟队列]
    #include <stdio.h>
    #include <stdbool.h>

    #define N 100010
    int queue[N];
    int tail;
    int head;

    void push(int x) {
        queue[++tail] = x;
    }

    bool isEmpty() {
        return head == tail;
    }

    void pop() {
        head++;
    }

    int query() {
        return queue[head + 1];
    }

    int main() {
        int n;
        scanf("%d", &n);
        char op[10];
        while (n--) {
            scanf("%s", op);
            if (strcmp(op, "push") == 0) {
                int x;
                scanf("%d", &x);
                push(x);
            }
            if (strcmp(op, "pop") == 0) {
                pop();
            }
            if (strcmp(op, "empty") == 0) {
                printf("%s\n", isEmpty() ? "YES" : "NO");
            }
            if (strcmp(op, "query") == 0) {
                printf("%d\n", query());
            }
        }
        return 0;
    }

\end{mycpptwocol}


\section{单调栈}

\subsection{AcWing 830. 单调栈}

\begin{titledbox}{AcWing 830. 单调栈}
    给定一个长度为 $N$ 的整数数列,输出每个数左边第一个比它小的数,如果不存在则输出 $-1$。

    输入格式:

    第一行包含整数 $N$,表示数列长度。第二行包含 $N$ 个整数,表示整数数列。

    输出格式:

    共一行,包含 $N$ 个整数,其中第 $i$ 个数表示第 $i$ 个数的左边第一个比它小的数,如果不存在则输出 $-1$。

    数据范围:

    $1 \le N \le 10^5$,$1 \le \text{数列中元素} \le 10^9$

    \begin{inputblock}
        \inlinecode{5} \\
        \inlinecode{3 4 2 7 5}
    \end{inputblock}
    \begin{outputblock}
        \inlinecode{-1 3 -1 2 2}
    \end{outputblock}

\end{titledbox}

\begin{mycpptwocol}[单调栈]
    #include <stdio.h>
    #include <stdlib.h>

    int main() {
        int n;
        scanf("%d", &n);
        int *stack = (int *)calloc(sizeof(int), n);
        int top = 0;
        while (n--) {
            int x;
            scanf("%d", &x);
            while (top != 0 && stack[top] >= x) {
                top--;
            }
            if (top == 0) {
                printf("%d ", -1);
            } else {
                printf("%d ", stack[top]);
            }
            stack[++top] = x;
        }
        free(stack);
        return 0;
    }
\end{mycpptwocol}


\section{单调队列}

\subsection{AcWing 154. 滑动窗口}

\begin{titledbox}{AcWing 154. 滑动窗口}
    给定一个大小为 $n \le 10^6$ 的数组。有一个大小为 $k$ 的滑动窗口,它从数组的最左边移动到最右边。你只能在窗口中看到 $k$ 个数字。每次滑动窗口向右移动一个位置。

    以下是一个例子:

    该数组为 \inlinecode{[1 3 -1 -3 5 3 6 7]},$k$ 为 $3$。

    \begin{tabular}{|c|c|c|}
        \hline
        窗口位置               & 最小值 & 最大值 \\ \hline
        [1 3  -1] -3 5 3 6 7 & -1  & 3   \\ \hline
        1 [3  -1  -3] 5 3 6 7  & -3  & 3   \\ \hline
        1 3 [-1  -3 5] 3 6 7 & -3  & 5   \\ \hline
        1 3  -1 [-3 5 3] 6 7 & -3  & 5   \\ \hline
        1 3  -1  -3 [5 3 6] 7  & 3   & 6   \\ \hline
        1 3  -1  -3 5 [3 6 7]  & 3   & 7   \\ \hline
    \end{tabular}

    你的任务是确定滑动窗口位于每个位置时,窗口中的最大值和最小值。

    输入格式:
    输入包含两行。 第一行包含两个整数 $n$ 和 $k$,分别代表数组长度和滑动窗口的长度。第二行有 $n$ 个整数,代表数组的具体数值。同行数据之间用空格隔开。

    输出格式:
    输出包含两个。第一行输出,从左至右,每个位置滑动窗口中的最小值。第二行输出,从左至右,每个位置滑动窗口中的最大值。

    \begin{inputblock}
        \inlinecode{8 3} \\
        \inlinecode{1 3 -1 -3 5 3 6 7}
    \end{inputblock}
    \begin{outputblock}
        \inlinecode{-1 -3 -3 -3 3 3} \\
        \inlinecode{3 3 5 5 6 7}
    \end{outputblock}
\end{titledbox}

\begin{mycpptwocol}[滑动窗口:单调队列]
    #include <stdio.h>
    #include <stdlib.h>

    int main() {
        int n;
        int k;
        scanf("%d %d", &n, &k);
        int *a = (int *)calloc(sizeof(int), n);
        for (int i = 0; i < n; i++) {
            scanf("%d", a + i);
        }
        int *queue = (int *)calloc(sizeof(int), n);
        int head = 0;
        int tail = -1;
        for (int i = 0; i < n; i++) {
            if (head <= tail && i - k + 1 > queue[head]) {
                head++;
            }
            while (head <= tail && a[queue[tail]] >= a[i]) {
                tail--;
            }
            queue[++tail] = i;
            if (i >= k - 1) {
                printf("%d ", a[queue[head]]);
            }
        }
        puts("");
        head = 0;
        tail = -1;
        for (int i = 0; i < n; i++) {
            if (head <= tail && i - k + 1 > queue[head]) {
                head++;
            }
            while (head <= tail && a[queue[tail]] <= a[i]) {
                tail--;
            }
            queue[++tail] = i;
            if (i >= k - 1) {
                printf("%d ", a[queue[head]]);
            }
        }
        free(queue);
        free(a);
        return 0;
    }
\end{mycpptwocol}


\section{KMP}

\subsection{AcWing 831. KMP字符串}

\begin{titledbox}{AcWing 831. KMP字符串}
    给定一个模式串 $S$,以及一个模板串 $P$,所有字符串中只包含大小写英文字母以及阿拉伯数字。模板串 $P$ 在模式串 $S$ 中多次作为子串出现。求出模板串 $P$ 在模式串 $S$ 中所有出现的位置的起始下标。

    输入格式:

    第一行输入整数 $N$,表示字符串 $P$ 的长度。第二行输入字符串 $P$。第三行输入整数 $M$,表示字符串 $S$ 的长度。第四行输入字符串 $S$。

    输出格式:

    共一行,输出所有出现位置的起始下标(下标从 $0$ 开始计数),整数之间用空格隔开。

    数据范围:

    $1 \le N \le 10^5$,$1 \le M \le 10^6$

    \begin{inputblock}
        \inlinecode{3} \\
        \inlinecode{aba} \\
        \inlinecode{5} \\
        \inlinecode{ababa}
    \end{inputblock}
    \begin{outputblock}
        \inlinecode{0 2}
    \end{outputblock}
\end{titledbox}

\begin{mycpptwocol}[KMP]

\end{mycpptwocol}


\section{Trie}
高效地存储和查找字符串集合的数据结构,可以用传统结构体实现,亦可用数组来模拟指针。

\subsection{AcWing 835. Trie字符串统计}

\begin{titledbox}{AcWing 835. Trie字符串统计}
    维护一个字符串集合,支持两种操作:
    \begin{myenum}
        \item \inlinecode{I x} 向集合中插入一个字符串 $x$;
        \item \inlinecode{Q x} 询问一个字符串在集合中出现了多少次。
    \end{myenum}

    共有 $N$ 个操作,输入的字符串总长度不超过 $10^5$,字符串仅包含小写英文字母。

    输入格式:

    第一行包含整数 $N$,表示操作数。接下来 $N$ 行,每行包含一个操作指令,指令为 \inlinecode{I x} 或 \inlinecode{Q x} 中的一种。

    输出格式:

    对于每个询问指令 \inlinecode{Q x},都要输出一个整数作为结果,表示 $x$ 在集合中出现的次数。
    每个结果占一行。

    数据范围:

    $1 \le N \le 2*10^4$

    \begin{inputblock}
        \inlinecode{5} \\
        \inlinecode{I abc} \\
        \inlinecode{Q abc} \\
        \inlinecode{Q ab} \\
        \inlinecode{I ab} \\
        \inlinecode{Q ab}
    \end{inputblock}
    \begin{outputblock}
        \inlinecode{1} \\
        \inlinecode{0} \\
        \inlinecode{1}
    \end{outputblock}
\end{titledbox}

这里有两种写法:链表形式和数组模拟的方式。

\begin{mycpptwocol}[链表形式的Trie]
    #include <stdio.h>
    #include <stdlib.h>
    #include <string.h>
    #include <stdbool.h>

    #define ALPH_NUM 26
    #define N 100

    typedef struct Node_ {
        struct Node_ *nodes[ALPH_NUM];
        int cnt;
        bool isEnd;
    } Node;

    void Insert(Node *head, char *str) {
        int len = strlen(str);
        Node *tmp = head;
        for (int i = 0; i < len; i++) {
            int idx = str[i] - 'a';
            if (tmp->nodes[idx] == NULL) {
                tmp->nodes[idx] = (Node *)calloc(sizeof(Node), 1);
            }
            tmp = tmp->nodes[idx];
        }
        tmp->cnt += 1;
        tmp->isEnd = true;
    }

    int Query(Node *head, char *str) {
        int len = strlen(str);
        Node *tmp = head;
        for (int i = 0; i < len; i++) {
            int idx = str[i] - 'a';
            if (tmp->nodes[idx] == NULL) {
                return 0;
            }
            tmp = tmp->nodes[idx];
        }
        return tmp->isEnd ? tmp->cnt : 0;
    }

    void Free(Node *head) {
        Node *tmp = head;
        for (int i = 0; i < ALPH_NUM; i++) {
            if (tmp->nodes[i] == NULL) {
                continue;
            }
            Free(tmp->nodes[i]);
        }
        free(tmp);
    }

    int main() {
        int n;
        scanf("%d", &n);
        char op[2];
        char str[N];
        Node *head = (Node *)calloc(sizeof(Node), 1);
        while (n--) {
            scanf("%s %s", op, str);
            if (op[0] == 'I') {
                Insert(head, str);
            }
            if (op[0] == 'Q') {
                printf("%d\n", Query(head, str));
            }
        }
        Free(head);
        return 0;
    }
\end{mycpptwocol}

\begin{mycpptwocol}[数组模拟Trie]
    #include <stdio.h>
    #include <stdlib.h>
    #include <string.h>

    // define a macro for alphabets number
    #define ALPHABETS 26
    #define N 20010
    #define MAX_BUF 100

    int trie[N][ALPHABETS];
    int idx;
    int cnt[N];

    // insert str to trie
    void Insert(char *str) {
        int i;
        int len = strlen(str);
        int cur = 0;
        for (i = 0; i < len; i++) {
            int ch = str[i] - 'a';
            if (trie[cur][ch] == 0) {
                trie[cur][ch] = ++idx;
            }
            cur = trie[cur][ch];
        }
        cnt[cur]++;
    }

    // query str from trie
    int Query(char *str) {
        int i;
        int len = strlen(str);
        int cur = 0;
        for (i = 0; i < len; i++) {
            int ch = str[i] - 'a';
            if (trie[cur][ch] == 0) {
                return 0;
            }
            cur = trie[cur][ch];
        }
        return cnt[cur];
    }

    int main() {
        int n;
        scanf("%d", &n);
        char op[2];
        char str[MAX_BUF];
        while (n--) {
            scanf("%s %s", op, str);
            if (op[0] == 'I') {
                Insert(str);
            }
            if (op[0] == 'Q') {
                printf("%d\n", Query(str));
            }
        }

        return 0;
    }
\end{mycpptwocol}

\begin{information}
    根据上述两种方法,其实数组模拟Trie和链表形式的Trie在本质上也只是一个使用数组模拟的链表,所以对应到之前数组模拟的单链表可知:数组模拟无非就是将链表节点展平后每个数组元素存储下一个链表节点对应的数组下标。
\end{information}

\subsection{AcWing 143. 最大异或对}

\begin{titledbox}{143. 最大异或对}
    在给定的 $N$ 个整数 $A_1, A_2, \dots A_N$ 中选出两个进行 $xor$(异或)运算,得到的结果最大是多少?

    输入格式:

    第一行输入一个整数 $N$。第二行输入 $N$ 个整数 $A_1, \dots A_N$。

    输出格式:

    输出一个整数表示答案。

    数据范围:

    $1 \le N \le 10^5$, $0 \le A_i < 2^{31}$

    \begin{inputblock}
        \inlinecode{3} \\
        \inlinecode{1 2 3}
    \end{inputblock}
    \begin{outputblock}
        \inlinecode{3}
    \end{outputblock}
\end{titledbox}

同样的,这里采用两种写法:

\begin{mycpptwocol}[链表形式的Trie]
    #include <stdio.h>
    #include <stdlib.h>

    typedef struct Node {
        struct Node *nexts[2];
    } Trie;

    // Insert int value into trie
    void Insert(Trie *trie, int value) {
        Trie *tmp = trie;
        for (int i = 30; i >= 0; i--) {
            int bit = (value >> i) & 1;
            if (tmp->nexts[bit] == NULL) {
                tmp->nexts[bit] = calloc(sizeof(Trie), 1);
            }
            tmp = tmp->nexts[bit];
        }
    }

    // Query the trie to get the value which xor to the value get the max
    int Query(Trie *trie, int value) {
        Trie *tmp = trie;
        int max = 0;
        for (int i = 30; i >= 0; i--) {
            int bit = (value >> i) & 1;
            if (tmp->nexts[!bit] != NULL) {
                max += (1 << i);
                tmp = tmp->nexts[!bit];
            } else {
                tmp = tmp->nexts[bit];
            }
        }
        return max;
    }

    // get max value from two int value
    int Max(int a, int b) {
        return a > b ? a : b;
    }

    // free entire trie
    void FreeTrie(Trie *trie) {
        for (int i = 0; i < 2; i++) {
            if (trie->nexts[i] != NULL) {
                FreeTrie(trie->nexts[i]);
            }
        }
        free(trie);
    }

    int main() {
        int n;
        scanf("%d", &n);
        int *arr = (int *) calloc(n, sizeof(int));
        Trie *trie = (Trie *)calloc(1, sizeof(Trie));
        for (int i = 0; i < n; i++) {
            int x;
            scanf("%d", &x);
            arr[i] = x;
            Insert(trie, x);
        }
        int res = 0;
        for (int i = 0; i < n; i++) {
            res = Max(res, Query(trie, arr[i]));
        }
        printf("%d", res);
        FreeTrie(trie);
        free(arr);
        return 0;
    }
\end{mycpptwocol}


\section{并查集}

两个操作:

\begin{myenum}
    \item 将两个集合合并
    \item 询问两个元素是否在一个集合中
\end{myenum}

基本原理:每个集合用一棵树来表示。树根的编号就是整个集合的编号,每个节点存储它的父节点, \inlinecode{p[x]} 标识 \inlinecode{x} 的父节点

问题:

\begin{myenum}
    \item 如何判断树根: \inlinecode{if(p[x] == x)}
    \item 如何求x的集合编号: \inlinecode{while(p[x] != x) x = p[x]}
    \item 如何合并两个集合: \inlinecode{p[p[x]] = p[y]}
\end{myenum}

\subsection{AcWing 836. 合并集合}

\begin{titledbox}{AcWing 836. 合并集合}
    一共有 $n$ 个数,编号是 $1 \sim n$,最开始每个数各自在一个集合中。现在要进行 $m$ 个操作,操作共有两种:
    \begin{myenum}
        \item \inlinecode{M a b},将编号为 $a$ 和 $b$ 的两个数所在的集合合并,如果两个数已经在同一个集合中,则忽略这个操作;
        \item \inlinecode{Q a b},询问编号为 $a$ 和 $b$ 的两个数是否在同一个集合中;
    \end{myenum}

    输入格式:

    第一行输入整数 $n$ 和 $m$。接下来 $m$ 行,每行包含一个操作指令,指令为 \inlinecode{M a b} 或 \inlinecode{Q a b} 中的一种。

    输出格式:

    对于每个询问指令 \inlinecode{Q a b},都要输出一个结果,如果 $a$ 和 $b$ 在同一集合内,则输出 \inlinecode{Yes},否则输出 \inlinecode{No}。每个结果占一行。

    数据范围:

    $1 \le n,m \le 10^5$

    \begin{inputblock}
        \inlinecode{4 5} \\
        \inlinecode{M 1 2} \\
        \inlinecode{M 3 4} \\
        \inlinecode{Q 1 2} \\
        \inlinecode{Q 1 3} \\
        \inlinecode{Q 3 4}
    \end{inputblock}
    \begin{outputblock}
        \inlinecode{Yes} \\
        \inlinecode{No} \\
        \inlinecode{Yes}
    \end{outputblock}
\end{titledbox}

\begin{mycpptwocol}[合并集合]
    #include <stdlib.h>
    #include <stdio.h>

    int Find(int *parent, int a) {
        if (parent[a] != a) {
            parent[a] = Find(parent, parent[a]);
        }
        return parent[a];
    }

    void Union(int *parent, int a, int b) {
        int aP = Find(parent, a);
        int bP = Find(parent, b);
        parent[aP] = bP;
    }

    int main() {
        int m, n;
        scanf("%d %d", &m, &n);
        int *parent = (int *) calloc(n + 1, sizeof(int));
        if (parent == NULL) {
            return -1;
        }
        for (int i = 0; i < n + 1; i++) {
            parent[i] = i;
        }
        while (n--) {
            char op[2];
            int a, b;
            scanf("%s %d %d", op, &a, &b);
            if (op[0] == 'M') {
                Union(parent, a, b);
            }
            if (op[0] == 'Q') {
                printf("%s\n", Find(parent, a) == Find(parent, b) ? "Yes" : "No");
            }
        }
        free(parent);
        return 0;
    }
\end{mycpptwocol}

\subsection{AcWing 837. 连通块中点的数量}

\begin{titledbox}{AcWing 837. 连通块中点的数量}
    给定一个包含 $n$ 个点(编号为 $1 \sim n$)的无向图,初始时图中没有边。现在要进行 $m$ 个操作,操作共有三种:
    \begin{myenum}
        \item \inlinecode{C a b},在点 $a$ 和点 $b$ 之间连一条边,$a$ 和 $b$ 可能相等;
        \item \inlinecode{Q1 a b},询问点 $a$ 和点 $b$ 是否在同一个连通块中,$a$ 和 $b$ 可能相等;
        \item \inlinecode{Q2 a},询问点 $a$ 所在连通块中点的数量;
    \end{myenum}

    输入格式:

    第一行输入整数 $n$ 和 $m$。接下来 $m$ 行,每行包含一个操作指令,指令为 \inlinecode{C a b},\inlinecode{Q1 a b} 或 \inlinecode{Q2 a} 中的一种。

    输出格式:

    对于每个询问指令 \inlinecode{Q1 a b},如果 $a$ 和 $b$ 在同一个连通块中,则输出 \inlinecode{Yes} ,否则输出 \inlinecode{No} 。对于每个询问指令 \inlinecode{Q2 a} ,输出一个整数表示点 $a$ 所在连通块中点的数量,每个结果占一行。

    数据范围:

    $1 \le n,m \le 10^5$

    \begin{inputblock}
        \inlinecode{5 5} \\
        \inlinecode{C 1 2} \\
        \inlinecode{Q1 1 2} \\
        \inlinecode{Q2 1} \\
        \inlinecode{C 2 5} \\
        \inlinecode{Q2 5}
    \end{inputblock}
    \begin{outputblock}
        \inlinecode{Yes} \\
        \inlinecode{2} \\
        \inlinecode{3}
    \end{outputblock}
\end{titledbox}

\begin{mycpptwocol}[]
    #include <stdio.h>

    int Find(int *p, int a) {
        if (p[a] != a) {
            p[a] = Find(p, p[a]);
        }
        return p[a];
    }

    void Union(int *p, int *cnt, int a, int b) {
        int ap = Find(p, a);
        int bp = Find(p, b);
        if (ap != bp) {
            p[ap] = bp;
            cnt[bp] += cnt[ap];
        }
    }

    int main() {
        int n;
        int m;
        scanf("%d %d", &n, &m);
        int *p = (int *)calloc(sizeof(int), n + 1);
        int *cnt = (int *)calloc(sizeof(int), n + 1);
        for (int i = 0; i <= n; i++) {
            p[i] = i;
            cnt[i] = 1;
        }
        char op[3];
        int a;
        int b;
        while (m--) {
            scanf("%s", op);
            if (op[0] == 'C') {
                scanf("%d %d", &a, &b);
                Union(p, cnt, a, b);
            }
            if (op[0] == 'Q') {
                if (op[1] == '1') {
                    scanf("%d %d", &a, &b);
                    printf("%s\n", Find(p, a) == Find(p, b) ? "Yes" : "No");
                } else {
                    scanf("%d", &a);
                    printf("%d\n", cnt[Find(p, a)]);
                }
            }
        }
        free(cnt);
        free(p);
        return 0;
    }
\end{mycpptwocol}

\begin{information}
    此处维护了一个 \inlinecode{cnt} 数组来标记集合中的元素个数,仅保证根节点的 \inlinecode{cnt} 是有意义的。
\end{information}

\subsection{AcWing 240. 食物链}


\section{堆}

用一维数组来模拟一个堆,起始坐标从 \inlinecode{1} 开始,因为 \inlinecode{x} 节点的左儿子是 \inlinecode{2x} 右儿子是 \inlinecode{2x + 1} ,不能用 \inlinecode{0} 开始。几种操作的实现如下:

\begin{myenum}
    \item 插入一个数: \inlinecode{heap[++size] = x; up(size);}
    \item 求集合中的最小值: \inlinecode{heap[1];}
    \item 删除最小值: \inlinecode{heap[1] = heap[size]; size--; down(1);}
    \item 删除任意一个元素: \inlinecode{heap[k] = heap[size]; size--; down(k); up(k);}
    \item 修改任意一个元素: \inlinecode{heap[k] = x; down(k); up(k);}
\end{myenum}

\subsection{AcWing 838. 堆排序}
此时只涉及到2和3两个操作即可,故仅需使用 \inlinecode{down} 操作,无需 \inlinecode{up} 操作。

\begin{titledbox}{AcWing 838. 堆排序}
    输入一个长度为 $n$ 的整数数列,从小到大输出前 $m$ 小的数。

    输入格式:

    第一行包含整数 $n$ 和 $m$。第二行包含 $n$ 个整数,表示整数数列。

    输出格式:
    共一行,包含 $m$ 个整数,表示整数数列中前 $m$ 小的数。

    数据范围:

    $1 \le m \le n \le 10^5$,$1 \le 数列中元素 \le 10^9$

    \begin{inputblock}
        \inlinecode{5 3} \\
        \inlinecode{4 5 1 3 2}
    \end{inputblock}
    \begin{outputblock}
        \inlinecode{1 2 3}
    \end{outputblock}

\end{titledbox}

\begin{mycpptwocol}[堆排序]
    #include <stdio.h>
    #include <stdlib.h>

    #define N 100010
    int h[N];
    int size;

    void down(int k) {
        int t = k;
        if (2 * k <= size && h[2 * k] < h[t]) {
            t = 2 * k;
        }
        if (2 * k + 1 <= size && h[2 * k + 1] < h[t]) {
            t = 2 * k + 1;
        }
        if (t != k) {
            int tmp = h[t];
            h[t] = h[k];
            h[k] = tmp;
            down(t);
        }
    }

    int main() {
        int n;
        int m;
        scanf("%d %d", &n, &m);
        for (int i = 1; i < n + 1; i++) {
            scanf("%d", &h[i]);
        }
        size = n;
        for (int i = n / 2; i > 0; i--) {
            down(i);
        }

        while (m--) {
            printf("%d ", h[1]);
            h[1] = h[size];
            size--;
            down(1);
        }
        return 0;
    }
\end{mycpptwocol}

\subsection{AcWing 839. 模拟堆}

此处为了支持删除和修改任意一个元素,用了两个数组 \inlinecode{hp} 和 \inlinecode{ph} 来保存从堆( \inlinecode{h}, heap)到插入的操作数( \inlinecode{p}, pointer)之间的映射关系。

\begin{titledbox}{AcWing 839. 模拟堆}

    维护一个集合,初始时集合为空,支持如下几种操作:

    \begin{myenum}
        \item \inlinecode{I x},插入一个数 $x$;
        \item \inlinecode{PM},输出当前集合中的最小值;
        \item \inlinecode{DM},删除当前集合中的最小值(数据保证此时的最小值唯一);
        \item \inlinecode{D k},删除第 $k$ 个插入的数;
        \item \inlinecode{C k x},修改第 $k$ 个插入的数,将其变为 $x$;
    \end{myenum}

    现在要进行 N 次操作,对于所有第 2 个操作,输出当前集合的最小值。

    输入格式:

    第一行包含整数 N。接下来 N 行,每行包含一个操作指令,操作指令为 \inlinecode{I x},\inlinecode{PM},\inlinecode{DM},\inlinecode{D k} 或 \inlinecode{C k x} 中的一种。

    输出格式:

    对于每个输出指令 \inlinecode{PM},输出一个结果,表示当前集合中的最小值。每个结果占一行。

    数据范围

    $1 \le N \le 10^5$, $−10^9 \le x \le 10^9$

    数据保证合法。

    \begin{inputblock}
        \inlinecode{8} \\
        \inlinecode{I -10} \\
        \inlinecode{PM} \\
        \inlinecode{I -10} \\
        \inlinecode{D 1} \\
        \inlinecode{C 2 8} \\
        \inlinecode{I 6} \\
        \inlinecode{PM} \\
        \inlinecode{DM}
    \end{inputblock}
    \begin{outputblock}
        \inlinecode{-10} \\
        \inlinecode{6}
    \end{outputblock}
\end{titledbox}

\begin{mycpptwocol}[可修改任意元素的堆]
    #include <stdio.h>
    #include <stdlib.h>
    #include <string.h>

    #define N 100010

    // heap[N]存储堆中的值, heap[1]是堆顶,
    // $x$的左儿子是$2x$, 右儿子是$2x + 1$
    int heap[N];
    // hp[k]存储堆中下标是k的点是第几个插入的
    int hp[N];
    // ph[k]存储第$k$个插入的点在堆中的位置
    int ph[N];
    int size;

    void Swap(int *a, int *b) {
        int tmp = *a;
        *a = *b;
        *b = tmp;
    }

    void HeapSwap(int k, int t) {
        Swap(&ph[hp[k]], &ph[hp[t]]);
        Swap(&hp[k], &hp[t]);
        Swap(&heap[k], &heap[t]);
    }

    void Up(int k) {
        while (k / 2 > 0 &&
        heap[k / 2] > heap[k]) {
            HeapSwap(k, k / 2);
            k >>= 1;
        }
    }

    void Down(int k) {
        int t = k;
        if (2 * k <= size &&
        heap[2 * k] < heap[t]) {
            t = 2 * k;
        }
        if (2 * k + 1 <= size &&
        heap[2 * k + 1] < heap[t]) {
            t = 2 * k + 1;
        }
        if (k != t) {
            HeapSwap(k, t);
            Down(t);
        }
    }

    int main() {
        int n;
        scanf("%d", &n);
        int idx = 0;
        while (n--) {
            char op[3];
            scanf("%s", op);
            int x;
            if (strcmp(op, "I") == 0) {
                scanf("%d", &x);
                heap[++size] = x;
                hp[size] = ++idx;
                ph[idx] = size;
                Up(size);
            }
            if (strcmp(op, "PM") == 0) {
                printf("%d\n", heap[1]);
            }
            if (strcmp(op, "DM") == 0) {
                HeapSwap(1, size);
                size--;
                Down(1);
            }
            if (strcmp(op, "D") == 0) {
                int k;
                scanf("%d", &k);
                k = ph[k];
                HeapSwap(k, size--);
                Down(k);
                Up(k);
            }
            if (strcmp(op, "C") == 0) {
                int k;
                scanf("%d %d", &k, &x);
                k = ph[k];
                heap[k] = x;
                Down(k);
                Up(k);
            }
        }
        return 0;
    }
\end{mycpptwocol}


\section{哈希表}
哈希表可以用来将一堆范围很大($10^9$)很分散的数据集合存储在相对较小的范围内(约 $10^5 \sim 10^6$)。

\subsection{AcWing 840. 模拟散列表}

\begin{titledbox}{AcWing 840. 模拟散列表}
    维护一个集合,支持如下几种操作:

    \inlinecode{I x} ,插入一个数 $x$;

    \inlinecode{Q x} ,询问数 $x$ 是否在集合中出现过;

    现在要进行 $N$ 次操作,对于每个询问操作输出对应的结果。

    输入格式:

    第一行包含整数 $N$,表示操作数量。接下来 $N$ 行,每行包含一个操作指令,操作指令为 \inlinecode{I x} ,\inlinecode{Q x}  中的一种。

    输出格式:

    对于每个询问指令 \inlinecode{Q x} ,输出一个询问结果,如果 $x$ 在集合中出现过,则输出 \inlinecode{Yes} ,否则输出 \inlinecode{No} 。每个结果占一行。

    数据范围:

    $1 \le N \le 10^5$,$-10^9 \le x \le 10^9$

    \begin{inputblock}
        \inlinecode{5} \\
        \inlinecode{I 1} \\
        \inlinecode{I 2} \\
        \inlinecode{I 3} \\
        \inlinecode{Q 2} \\
        \inlinecode{Q 5}
    \end{inputblock}
    \begin{outputblock}
        \inlinecode{Yes} \\
        \inlinecode{No}
    \end{outputblock}
\end{titledbox}

从该题描述可看出:操作的数据范围很大($-10^9 \sim 10^9$),但是操作次数却不大($10^5$)。是故,这里使用哈希表进行离散即可。Hash函数的设计通常是对大质数取模,这里使用的函数为:
\begin{equation*}
    h(x) = (x \mod N + N) \mod N, \quad N = 200003
\end{equation*}
该函数第一次取模后加$N$的原因在于:在C/C++中,负数取模会得到负数,这与数学定义不同(数学定义中不论正数还是负数取模的结果都是正数)。为保证结果始终是正数,需要做此操作。

Hash函数难免产生碰撞(冲突:$h(t_1) = h(t_2), \text{当} t_1 \neq t_2$),解决冲突通常有两种写法:

1. 开放寻址法

此法将从哈希后的那个位置开始查找,如果已经有了不是自己的人占用该位置则继续向后寻找,直到找到一个空位置,返回该位置。此法中 \inlinecode{Find} 函数将返回 \inlinecode{x} 应该存放的位置。此法仅需开一个数组即可,通常来讲要开数据规模的两倍到三倍,以防 \inlinecode{buffer overflow}。

\begin{mycpptwocol}[开放寻址法]
    #include <stdio.h>
    #include <stdlib.h>
    #include <string.h>

    #define N 200003

    int h[N];
    const int null = 0x3f3f3f3f;

    int Find(int x) {
        int k = (x % N + N) % N;
        while (h[k] != null && h[k] != x) {
            k++;
            if (k == N) {
                k = 0;
            }
        }
        return k;
    }

    int main() {
        int n;
        scanf("%d", &n);
        memset(h, 0x3f, sizeof(h));
        int x;
        int k;
        while (n--) {
            char op[2];
            scanf("%s %d", op, &x);
            k = Find(x);
            if (op[0] == 'I') {
                h[k] = x;
            } else {
                if (h[k] == x) {
                    puts("Yes");
                } else {
                    puts("No")
                }
            }
        }
        return 0;
    }
\end{mycpptwocol}

2. 拉链法

此法将在哈希数组的每一个位置 \inlinecode{h[t]} 处放置一个单链表,来存储所有哈希值为 \inlinecode{t} 的数字。该单链表可以仅有一个数组表示,每个 \inlinecode{h[t]} 中存储这一个数组中的不同下标表示单链表的头节点位置。

\begin{mycpptwocol}[拉链法]
    #include <stdio.h>
    #include <stdlib.h>
    #include <stdbool.h>
    #include <string.h>

    #define N 100003

    // 这几个结构为一个数组拉了很多链,
    // 图的邻接表同此写法
    int h[N]; // Hash表的N个槽
    int e[N];
    int ne[N];
    int idx;

    void Insert(int x) {
        int t = (x % N + N) % N; // 找到对应的槽
        // 链表的头插法
        e[++idx] = x;
        ne[idx] = h[t];
        h[t] = idx;
    }

    bool Find(int x) {
        int k = (x % N + N) % N;
        for (int i = h[k]; i != -1; i = ne[i]) {
            if (e[i] == x) {
                return true;
            }
        }
        return false;
    }

    int main() {
        int n;
        scanf("%d", &n);
        memset(h, -1, sizeof(h));
        int x;
        int k;
        while (n--) {
            char op[2];
            scanf("%s %d", op, &x);
            if (op[0] == 'I') {
                Insert(x);
            } else {
                if (Find(x)) {
                    puts("Yes");
                } else {
                    puts("No");
                }
            }
        }
        return 0;
    }
\end{mycpptwocol}

\subsection{AcWing 841. 字符串哈希}

字符串哈希使用的方法叫做:字符串前缀哈希法。若有一字符串 \inlinecode{str = "abcedacbd"},则哈希表中的元素表示从左边起到第i个位置上的子字符串的哈希值。即, \inlinecode{h[1] -> "a"; h[2] -> "ab"; h[3] -> "abc"},以此类推。

\begin{myenum}
    \item 将字符串存储在下标为1的 \inlinecode{char} 数组中;
    \item 计算字符串的前缀数字,将他们哈希化,将字符串看作是一个 \inlinecode{P} 进制数,这里 \inlinecode{P = 131} 或者 \inlinecode{P = 13331}。按照经验,这两个值对字符串哈希会产生非常小的冲突可能。
    \item 将这个 \inlinecode{P} 进制数转换为10进制,通常转换出来的数字会很大,则将其对 \inlinecode{Q} 取模。这样任意一个字符串都可以转换成 $0 \cdots Q - 1$ 之间的数字:$(1234)_P => (1 * P^3 + 2 * P^2 + 3 * P^1 + 4 * P^0) \mod Q$
    \item 对于 \inlinecode{Hash} 函数,将计算字符串中从 $l$ 到 $r$ 的哈希值,通过前缀哈希表以及下公式计算,注意左侧为低位,右侧为高位。

    \inlinecode{h[r] - h[l - 1] * p[r - l + 1]} ,其中 \inlinecode{p} 数组存放了从 $P^0 \sim P^N$ 的所有数字,方便查询。
\end{myenum}

注意:一般来讲不能将字符串映射成 $0$ ,因为任意进制的 $0$ 都将映射成十进制的 $0$ 。此时有:$\mlq A \mrq \mapsto 0; \mlq AA \mrq \mapsto 0$,这样不同的字符串映射成了相同的值。另外,这种方法假定了冲突不存在,不需要解决冲突。按照经验一般取 $P = 131\, \text{or}\, 13331, Q = 2^{64}$ 时冲突的概率几乎为 $0$ 。

\begin{titledbox}{AcWing 841. 字符串哈希}
    给定一个长度为 $n$ 的字符串,再给定 $m$ 个询问,每个询问包含四个整数 $l_1, r_1, l_2, r_2$,请你判断 $[l_1, r_1]$ 和 $[l_2, r_2]$ 这两个区间所包含的字符串子串是否完全相同。
    字符串中只包含大小写英文字母和数字。

    输入格式:

    第一行包含整数 $n$ 和 $m$,表示字符串长度和询问次数。第二行包含一个长度为 $n$ 的字符串,字符串中只包含大小写英文字母和数字。接下来 $m$ 行,每行包含四个整数 $l_1, r_1, l_2, r_2$,表示一次询问所涉及的两个区间。注意,字符串的位置从 $1$ 开始编号。

    输出格式:

    对于每个询问输出一个结果,如果两个字符串子串完全相同则输出 \inlinecode{Yes} ,否则输出 \inlinecode{No}。每个结果占一行。

    数据范围:

    $1 \le n, m \le 10^5$

    \begin{inputblock}
        \inlinecode{8 3} \\
        \inlinecode{aabbaabb} \\
        \inlinecode{1 3 5 7} \\
        \inlinecode{1 3 6 8} \\
        \inlinecode{1 2 1 2}
    \end{inputblock}
    \begin{outputblock}
        \inlinecode{Yes} \\
        \inlinecode{No} \\
        \inlinecode{Yes}
    \end{outputblock}
\end{titledbox}

\begin{mycpptwocol}[字符串前缀哈希法]
    #include <stdio.h>
    #include <stdbool.h>

    #define N 100010
    char str[N];
    unsigned long long h[N];
    unsigned long long p[N];
    const int P = 131;

    int Hash(int l, int r) {
        return h[r] - h[l - 1] * p[r - l + 1];
    }

    int main() {
        int n;
        int m;
        scanf("%d %d", &n, &m);
        scanf("%s", str + 1);

        p[0] = 1;
        for (int i = 1; i <= n; i++) {
            p[i] = p[i - 1] * P;
            h[i] = str[i] + h[i - 1] * P;
        }

        while (m--) {
            int l1, r1, l2, r2;
            scanf("%d %d", &l1, &r1);
            scanf("%d %d", &l2, &r2);
            if (Hash(l1, r1) == Hash(l2, r2)) {
                puts("Yes");
            } else {
                puts("No");
            }
        }
        return 0;
    }
\end{mycpptwocol}

\begin{information}
    注意:此处将字符串存储在下标为1的地方开始。是因为这里将字符串看作是 \inlinecode{P} 进制数,如果从 0 开始会将同一个字符构成的字符串都算做相同的值,会有问题。

    另外,这里没有显式的对 \inlinecode{Q = $2^{64}$} 取模,是因为用了 \inlinecode{unsinged long long} 溢出后就相当于对 \inlinecode{Q} 取模。
\end{information}

