\chapter{数学知识}
本章主要分为四个部分:
\begin{myenum}
    \item 数论
    \item 组合计数
    \item 高斯消元
    \item 简单博弈论
\end{myenum}


\section{质数}
质数(Prime number),又称素数,指在大于1的自然数中,除了1和该数自身外,无法被其他自然数整除的数(也可定义为只有1与该数本身两个正因数的数)。大于1的自然数若不是素数,则称之为合数(也称为合成数)。算术基本定理确立了素数于数论里的核心地位:任何大于1的整数均可被表示成一串唯一素数之乘积。

\textbf{注意:}质数是从2开始定义的。所有$\leq 1$的整数既不是质数也不是合数。

\subsection{AcWing 866. 试除法判定质数}
\begin{titledbox}{AcWing 866. 试除法判定质数}
    给定 $n$ 个正整数 $a_i$,判定每个数是否是质数。

    输入格式:

    第一行包含整数 $n$。 接下来 $n$ 行,每行包含一个正整数 $a_i$。

    输出格式:

    共 $n$ 行,其中第 $i$ 行输出第 $i$ 个正整数 $a_i$ 是否为质数,是则输出 \inlinecode{Yes},否则输出 \inlinecode{No}。

    数据范围:

    $1 \le n \le 100$, $1 \le a_i \le 2^{31}-1$

    \begin{inputblock}
        \inlinecode{2} \\
        \inlinecode{2} \\
        \inlinecode{6}
    \end{inputblock}
    \begin{outputblock}
        \inlinecode{Yes} \\
        \inlinecode{No}
    \end{outputblock}
\end{titledbox}

从定义出发,可以使用简单的试除法来判定一个数是否为质数。

\begin{mycpptwocol}[朴素试除法]
    #include <stdio.h>
    #include <stdbool.h>

    bool IsPrime(int x) {
        if (x < 2) {
            return false;
        }
        for (int i = 2; i < x; i++) {
            if (x \% i == 0) {
                return false;
            }
        }
        return true;
    }

    int main() {
        int n;
        scanf("%d", &n);

        while (n--) {
            int x;
            scanf("%d", &x);
            printf("%s\n", IsPrime(x) ? "Yes" : "No");
        }

        return 0;
    }
\end{mycpptwocol}

显而易见的,该朴素试除法的时间复杂度为 \bigo{$n$},效率比较低。考虑到数的性质:
\begin{equation*}
    d \mid n \Rightarrow \frac{n}{d} \mid n
\end{equation*}
可见数的约数是成对出现的。所以枚举的过程中可以仅枚举一对中较小的那一个($d \le n / d \Leftrightarrow d \le \sqrt{n}$)即可。这样时间复杂度从\bigo{$n$}降低到\bigo{$\sqrt{n}$}。

\begin{mycpponecol}[优化试除法]
    for (int i = 2; i <= x / i; i++) {
        if (x \% i == 0) {
            return false;
        }
    }
\end{mycpponecol}

此外,判断条件那里还有两种不推荐的写法:
\begin{mylist}
    \item \inlinecode{i * i <= n} 这种会导致乘法的溢出
    \item \inlinecode{i <= sqrt(n)} 由于函数 \inlinecode{sqrt(n)} 计算速度很慢而不推荐
\end{mylist}

\subsection{AcWing 867. 分解质因数}

\begin{algorithm}[H] %or another one check
    \caption{试除法分解质因数}
    \SetAlgoLined
    \KwResult{$x$ 的所有素因子}
    初始化结果数组 result \\

    \For{each vertex $v \neq s$ in $V(G)$}{
        $d(v) \leftarrow \infty$
    }
    $d(s) \leftarrow 0$\\
    push $s$ into $Q$\\
    \While{$Q$ is not empty}{
        $u \leftarrow \text{poll } Q$\\
        \For{each edge ($u, v$) in $E(G)$} {
            \If{$d(u) + w(u, v) < d(v)$}{
                $d(v) \leftarrow d(u) + w(u, v)$\\
                \If{$v$ is not in $Q$}{
                    push $v$ into $Q$
                }
            }
        }
    }
\end{algorithm}

\begin{titledbox}{AcWing 867. 分解质因数}
    给定 $n$ 个正整数 $a_i$,将每个数分解质因数,并按照质因数从小到大的顺序输出每个质因数的底数和指数。

    输入格式:

    第一行包含整数 $n$。 接下来 $n$ 行,每行包含一个正整数 $a_i$。

    输出格式:

    对于每个正整数 $a_i$,按照从小到大的顺序输出其分解质因数后,每个质因数的底数和指数,每个底数和指数占一行。 每个正整数的质因数全部输出完毕后,输出一个空行。

    数据范围:

    $1 \le n \le 100$, $1 \le a_i \le 2 \times 10^9$

    \begin{inputblock}
        \inlinecode{2} \\
        \inlinecode{6} \\
        \inlinecode{8}
    \end{inputblock}
    \begin{outputblock}
        \inlinecode{2 1} \\
        \inlinecode{3 1} \\
        \\
        \inlinecode{2 3} \\

    \end{outputblock}
\end{titledbox}

\begin{mycpptwocol}[试除法分解质因数]
    #include <stdio.h>
    #include <stdlib.h>

    void split(int x) {
        for (int i = 2; i <= x; i++) {
            if (x \% i == 0) {
                int cnt = 0;
                while (x \% i == 0) {
                    x /= i;
                    cnt++;
                }
                printf("%d %d\n", i, cnt);
            }
        }
        printf("\n");
    }

    int main() {
        int n;
        scanf("%d", &n);
        while (n--) {
            int x;
            scanf("%d", &x);
            split(x);
        }
        return 0;
    }
\end{mycpptwocol}

\begin{proof}
    首先证明 $i$ 均为 $x$ 的素因数:因为当且仅当 $x \% i == 0$ 满足时,result 发生变化:储存 $i$,说明此时 $i$ 能整除 $x / A$ ,说明了存在一个数 $p$ 使得 $pi = x / A$,即 $piA = x$(其中,$A$ 为 $x$ 自身发生变化后遇到 $i$ 时所除的数。我们注意到 result 若在 push $i$ 之前就已经有数了,为 $R_1, R_2, \dots, R_n$,那么有 $x = x / (R_1^{q_1}\cdot R_2^{q_2}\cdots R_n^{q_n})$ ,被除的乘积即为 $A$ )。所以 $i$ 为 $x$ 的因子。

    其次证明 result 中均为素数。我们假设存在一个在 result 中的合数 $K$,并根据整数基本定理,分解为一个素数序列 $K = K_1^{e_1}\cdot K_2^{e_2}\cdots K_n^{e_n}$,而因为 $K_1 < K$,所以它一定会在 $K$ 之前被遍历到,并令 \inlinecode{while(N \% k1 == 0) N /= k1},即让 $N$ 没有了素因子$K_1$ ,故遍历到 $K$ 时,$N$ 和 $K$ 已经没有了整除关系了。
\end{proof}

\subsection{AcWing 868. 筛质数}
\begin{titledbox}{AcWing 868. 筛质数}
    给定一个正整数 $n$,请你求出 $1 \sim n$ 中质数的个数。

    输入格式:

    共一行,包含整数 $n$。

    输出格式:

    共一行,包含一个整数,表示 $1 \sim n$ 中质数的个数。

    数据范围:

    $1 \le n \le 10^6$

    \begin{inputblock}
        \inlinecode{8}
    \end{inputblock}
    \begin{outputblock}
        \inlinecode{4}
    \end{outputblock}
\end{titledbox}


\section{约数}

\subsection{AcWing 869. 试除法求约数}
\begin{titledbox}{AcWing 869. 试除法求约数}
    给定 $n$ 个正整数 $a_i$,对于每个整数 $a_i$,请你按照从小到大的顺序输出它的所有约数。

    输入格式:

    第一行包含整数 $n$。 接下来 $n$ 行,每行包含一个整数 $a_i$。

    输出格式:

    输出共 $n$ 行,其中第 $i$ 行输出第 $i$ 个整数 $a_i$ 的所有约数。

    数据范围:

    $1 \le n \le 100$, $2 \le a_i \le 2 \times 10^9$

    \begin{inputblock}
        \inlinecode{2} \\
        \inlinecode{6} \\
        \inlinecode{8}
    \end{inputblock}
    \begin{outputblock}
        \inlinecode{1 2 3 6} \\
        \inlinecode{1 2 4 8}
    \end{outputblock}
\end{titledbox}

\subsection{AcWing 870. 约数个数}

\begin{titledbox}{AcWing 870. 约数个数}
    给定 $n$ 个正整数 $a_i$,请你输出这些数的乘积的约数个数,答案对 $10^9+7$ 取模。

    输入格式:

    第一行包含整数 $n$。 接下来 $n$ 行,每行包含一个整数 $a_i$。

    输出格式:

    输出一个整数,表示所给正整数的乘积的约数个数,答案需对 $10^9+7$ 取模。

    数据范围:

    $1 \le n \le 100$, $1 \le a_i \le 2 \times 10^9$

    \begin{inputblock}
        \inlinecode{3} \\
        \inlinecode{2} \\
        \inlinecode{6} \\
        \inlinecode{8}
    \end{inputblock}
    \begin{outputblock}
        \inlinecode{12}
    \end{outputblock}
\end{titledbox}

\subsection{AcWing 871. 约数之和}
\begin{titledbox}{AcWing 871. 约数之和}
    给定 $n$ 个正整数 $a_i$,请你输出这些数的乘积的约数之和,答案对 $10^9+7$ 取模。

    输入格式:

    第一行包含整数 $n$。 接下来 $n$ 行,每行包含一个整数 $a_i$。

    输出格式:

    输出一个整数,表示所给正整数的乘积的约数之和,答案需对 $10^9+7$ 取模。

    数据范围:

    $1 \le n \le 100$, $1 \le a_i \le 2 \times 10^9$

    \begin{inputblock}
        \inlinecode{3} \\
        \inlinecode{2} \\
        \inlinecode{6} \\
        \inlinecode{8}
    \end{inputblock}
    \begin{outputblock}
        \inlinecode{252}
    \end{outputblock}
\end{titledbox}

\subsection{AcWing 872. 最大公约数}
\begin{titledbox}{AcWing 872. 最大公约数}
    给定 $n$ 对正整数 $a_i, b_i$,请你求出每对数的最大公约数。

    输入格式:

    第一行包含整数 $n$。 接下来 $n$ 行,每行包含一个整数对 $a_i,b_i$。

    输出格式:

    输出共 $n$ 行,每行输出一个整数对的最大公约数。

    数据范围:

    $1 \le n \le 10^5$, $1 \le a_i, b_i \le 2 \times 10^9$

    \begin{inputblock}
        \inlinecode{2} \\
        \inlinecode{3 6} \\
        \inlinecode{4 6}
    \end{inputblock}
    \begin{outputblock}
        \inlinecode{3} \\
        \inlinecode{2}
    \end{outputblock}
\end{titledbox}


\section{欧拉函数}

\subsection{AcWing 873. 欧拉函数}
\begin{titledbox}{AcWing 873. 欧拉函数}
    给定 $n$ 个正整数 $a_i$,请你求出每个数的欧拉函数。

    欧拉函数的定义:

    \begin{quote}
        $1 \sim N$ 中与 $N$ 互质的数的个数被称为欧拉函数,记为 $\varphi(N)$。

        若在算数基本定理中,$N = p_1^{a_1}p_2^{a_2}\dots p_m^{a_m}$,则:

        $\varphi (N)$ = $N \times \frac{p_1-1}{p_1} \times \frac{p_2-1}{p_2} \times \dots \times \frac{p_m-1}{p_m}$
    \end{quote}

    输入格式:

    第一行包含整数 $n$。 接下来 $n$ 行,每行包含一个正整数 $a_i$。

    输出格式:

    输出共 $n$ 行,每行输出一个正整数 $a_i$ 的欧拉函数。

    数据范围:

    $1 \le n \le 100$, $1 \le a_i \le 2 \times 10^9$

    \begin{inputblock}
        \inlinecode{3} \\
        \inlinecode{3} \\
        \inlinecode{6} \\
        \inlinecode{8}
    \end{inputblock}
    \begin{outputblock}
        \inlinecode{2} \\
        \inlinecode{2} \\
        \inlinecode{4}
    \end{outputblock}
\end{titledbox}

\subsection{AcWing 874. 筛法求欧拉函数}
\begin{titledbox}{AcWing 874. 筛法求欧拉函数}
    给定一个正整数 $n$,求 $1 \sim n$ 中每个数的欧拉函数之和。

    输入格式:

    共一行,包含一个整数 $n$。

    输出格式:

    共一行,包含一个整数,表示 $1 \sim n$ 中每个数的欧拉函数之和。

    数据范围:

    $1 \le n \le 10^6$

    \begin{inputblock}
        \inlinecode{6}
    \end{inputblock}
    \begin{outputblock}
        \inlinecode{12}
    \end{outputblock}
\end{titledbox}


\section{快速幂}

\subsection{AcWing 875. 快速幂}
\begin{titledbox}{AcWing 875. 快速幂}
    给定 $n$ 组 $a_i, b_i, p_i$,对于每组数据,求出 $a_i ^ {b_i} \bmod p_i$ 的值。

    输入格式:

    第一行包含整数 $n$。 接下来 $n$ 行,每行包含三个整数 $a_i, b_i, p_i$。

    输出格式:

    对于每组数据,输出一个结果,表示 $a_i ^ {b_i} \bmod p_i$ 的值。 每个结果占一行。

    数据范围:

    $1 \le n \le 100000$, $1 \le a_i,b_i,p_i \le 2 \times 10^9$

    \begin{inputblock}
        \inlinecode{2} \\
        \inlinecode{3 2 5} \\
        \inlinecode{4 3 9}
    \end{inputblock}
    \begin{outputblock}
        \inlinecode{4} \\
        \inlinecode{1}
    \end{outputblock}
\end{titledbox}

\subsection{AcWing 876. 快速幂求逆元}
\begin{titledbox}{AcWing 876. 快速幂求逆元}
    给定 $n$ 组 $a_i, p_i$,其中 $p_i$ 是质数,求 $a_i$ 模 $p_i$ 的乘法逆元,若逆元不存在则输出 \inlinecode{impossible}。\textbf{注意}:请返回在 $0 \sim p-1$ 之间的逆元。

    \begin{description}
        \item[乘法逆元的定义] \\
        若整数 $b,m$ 互质,并且对于任意的整数 $a$,如果满足 $b|a$,则存在一个整数 $x$,使得 $a/b≡a \times x \pmod m$,则称 $x$ 为 $b$ 的模 $m$ 乘法逆元,记为 $b^{-1} \pmod m$。\\
        $b$ 存在乘法逆元的充要条件是 $b$ 与模数 $m$ 互质。当模数 $m$ 为质数时,$b^{m-2}$ 即为 $b$ 的乘法逆元。
    \end{description}

    输入格式:

    第一行包含整数 $n$。 接下来 $n$ 行,每行包含一个数组 $a_i, p_i$,数据保证 $p_i$ 是质数。

    输出格式:

    输出共 $n$ 行,每组数据输出一个结果,每个结果占一行。 若 $a_i$ 模 $p_i$ 的乘法逆元存在,则输出一个整数,表示逆元,否则输出 \inlinecode{impossible}。

    数据范围:

    $1 \le n \le 10^5$, $1 \le a_i,p_i \le 2*10^9$

    \begin{inputblock}
        \inlinecode{4 3} \\
        \inlinecode{8 5} \\
        \inlinecode{6 3}
    \end{inputblock}
    \begin{outputblock}
        \inlinecode{1} \\
        \inlinecode{2} \\
        \inlinecode{impossible}
    \end{outputblock}
\end{titledbox}


\section{扩展欧几里得算法}

\subsection{AcWing 877. 扩展欧几里得算法}
\begin{titledbox}{AcWing 877. 扩展欧几里得算法}
    给定 $n$ 对正整数 $a_i, b_i$,对于每对数,求出一组 $x_i, y_i$,使其满足 $a_i \times x_i + b_i \times y_i = gcd(a_i, b_i)$。

    输入格式:

    第一行包含整数 $n$。 接下来 $n$ 行,每行包含两个整数 $a_i, b_i$。

    输出格式:

    输出共 $n$ 行,对于每组 $a_i, b_i$,求出一组满足条件的 $x_i, y_i$,每组结果占一行。 本题答案不唯一,输出任意满足条件的 $x_i, y_i$ 均可。

    数据范围:

    $1 \le n \le 10^5$, $1 \le a_i,b_i \le 2 \times 10^9$

    \begin{inputblock}
        \inlinecode{2} \\
        \inlinecode{4 6} \\
        \inlinecode{8 18}
    \end{inputblock}
    \begin{outputblock}
        \inlinecode{-1 1} \\
        \inlinecode{-2 1}
    \end{outputblock}
\end{titledbox}

\subsection{AcWing 878. 线性同余方程}
\begin{titledbox}{AcWing 878. 线性同余方程}
    给定 $n$ 组数据 $a_i,b_i,m_i$,对于每组数求出一个 $x_i$,使其满足 $a_i \times x_i \equiv b_i \pmod {m_i}$,如果无解则输出 \inlinecode{impossible}。

    输入格式:

    第一行包含整数 $n$。 接下来 $n$ 行,每行包含一组数据 $a_i,b_i,m_i$。

    输出格式:

    输出共 $n$ 行,每组数据输出一个整数表示一个满足条件的 $x_i$,如果无解则输出 \inlinecode{impossible}。 每组数据结果占一行,结果可能不唯一,输出任意一个满足条件的结果均可。 输出答案必须在 $int$ 范围之内。

    数据范围:

    $1 \le n \le 10^5$, $1 \le a_i,b_i,m_i \le 2 \times 10^9$

    \begin{inputblock}
        \inlinecode{2} \\
        \inlinecode{2 3 6} \\
        \inlinecode{4 3 5}
    \end{inputblock}
    \begin{outputblock}
        \inlinecode{impossible} \\
        \inlinecode{-3}
    \end{outputblock}
\end{titledbox}


\section{中国剩余定理}

\subsection{AcWing 204. 表达整数的奇怪方式}
\begin{titledbox}{AcWing 204. 表达整数的奇怪方式}
    给定 $2n$ 个整数 $a_1,a_2,\dots,a_n$ 和 $m_1,m_2,\dots ,m_n$,求一个最小的非负整数 $x$,满足 $ \forall i \in [1,n],x \equiv m_i(mod\ a_i)$。

    输入格式:

    第 $1$ 行包含整数 $n$。 第 $2 \dots n+1$ 行:每 $i+1$ 行包含两个整数 $a_i$ 和 $m_i$,数之间用空格隔开。

    输出格式:

    输出最小非负整数 $x$,如果 $x$ 不存在,则输出 $-1$。如果存在 $x$,则数据保证 $x$ 一定在 $64$ 位整数范围内。

    数据范围:

    $1 \le a_i \le 2^{31}-1$, $0 \le m_i < a_i$, $1 \le n \le 25$

    \begin{inputblock}
        \inlinecode{2} \\
        \inlinecode{8 7} \\
        \inlinecode{11 9}
    \end{inputblock}
    \begin{outputblock}
        \inlinecode{31}
    \end{outputblock}
\end{titledbox}


\section{高斯消元}

\subsection{AcWing 883. 高斯消元解线性方程组}
\begin{titledbox}{AcWing 883. 高斯消元解线性方程组}
    输入一个包含 $n$ 个方程 $n$ 个未知数的线性方程组。 方程组中的系数为实数。 求解这个方程组。 下为一个包含 $m$ 个方程 $n$ 个未知数的线性方程组示例:
    \begin{equation*}
        \left\{
        \begin{array}{c}
            a_{11}x_1+a_{12}x_2+\cdots+a_{1n}x_n=b_1 \\
            a_{21}x_1+a_{22}x_2+\cdots+a_{2n}x_n=b_2 \\
            \cdots                                   \\
            a_{m1}x_1+a_{m2}x_2+\cdots+a_{mn}x_n=b_m
        \end{array}
        \right.
    \end{equation*}
    输入格式:

    第一行包含整数 $n$。 接下来 $n$ 行,每行包含 $n+1$ 个实数,表示一个方程的 $n$ 个系数以及等号右侧的常数。

    输出格式:

    如果给定线性方程组存在唯一解,则输出共 $n$ 行,其中第 $i$ 行输出第 $i$ 个未知数的解,结果保留两位小数。 如果给定线性方程组存在无数解,则输出 \inlinecode{Infinite group solutions}。 如果给定线性方程组无解,则输出 \inlinecode{No solution}。

    数据范围:

    $1 \le n \le 100$, 所有输入系数以及常数均保留两位小数,绝对值均不超过 $100$。

    \begin{inputblock}
        \inlinecode{3} \\
        \inlinecode{1.00 2.00 -1.00 -6.00} \\
        \inlinecode{2.00 1.00 -3.00 -9.00} \\
        \inlinecode{-1.00 -1.00 2.00 7.00}
    \end{inputblock}
    \begin{outputblock}
        \inlinecode{1.00} \\
        \inlinecode{-2.00} \\
        \inlinecode{3.00}
    \end{outputblock}
\end{titledbox}

\subsection{AcWing 884. 高斯消元解异或线性方程组}
\begin{titledbox}{AcWing 884. 高斯消元解异或线性方程组}
    输入一个包含 $n$ 个方程 $n$ 个未知数的异或线性方程组。方程组中的系数和常数为 $0$ 或 $1$,每个未知数的取值也为 $0$ 或 $1$。 求解这个方程组。

    异或线性方程组示例如下:

    \inlinecode{M[1][1]x[1] ^ M[1][2]x[2] ^ … ^ M[1][n]x[n] = B[1] } \\
    \inlinecode{M[2][1]x[1] ^ M[2][2]x[2] ^ … ^ M[2][n]x[n] = B[2] } \\
    \inlinecode{… } \\
    \inlinecode{M[n][1]x[1] ^ M[n][2]x[2] ^ … ^ M[n][n]x[n] = B[n] }


    其中 \inlinecode{^} 表示异或($XOR$),$M[i][j]$ 表示第 $i$ 个式子中 $x[j]$ 的系数,$B[i]$ 是第 $i$ 个方程右端的常数,取值均为 $0$ 或 $1$。

    输入格式:

    第一行包含整数 $n$。 接下来 $n$ 行,每行包含 $n+1$ 个整数 $0$ 或 $1$,表示一个方程的 $n$ 个系数以及等号右侧的常数。

    输出格式:

    如果给定线性方程组存在唯一解,则输出共 $n$ 行,其中第 $i$ 行输出第 $i$ 个未知数的解。 如果给定线性方程组存在多组解,则输出 \inlinecode{Multiple sets of solutions}。 如果给定线性方程组无解,则输出 \inlinecode{No solution}。

    数据范围:

    $1 \le n \le 100$

    \begin{inputblock}
        \inlinecode{3} \\
        \inlinecode{1 1 0 1} \\
        \inlinecode{0 1 1 0} \\
        \inlinecode{1 0 0 1}
    \end{inputblock}
    \begin{outputblock}
        \inlinecode{1} \\
        \inlinecode{0} \\
        \inlinecode{0}
    \end{outputblock}
\end{titledbox}


\section{求组合数}

\subsection{AcWing 885. 求组合数 I}
\begin{titledbox}{AcWing 885. 求组合数 I}
    给定 $n$ 组询问,每组询问给定两个整数 $a, b$,请你输出 $C_a^b \mod (10^9 + 7)$ 的值。

    输入格式:

    第一行包含整数 $n$。 接下来 $n$ 行,每行包含一组 $a$ 和 $b$。

    输出格式:

    共 $n$ 行,每行输出一个询问的解。

    数据范围:

    $1 \le n \le 10000$, $1 \le b \le a \le 2000$

    \begin{inputblock}
        \inlinecode{3} \\
        \inlinecode{3 1} \\
        \inlinecode{5 3} \\
        \inlinecode{2 2}
    \end{inputblock}
    \begin{outputblock}
        \inlinecode{3} \\
        \inlinecode{10} \\
        \inlinecode{1}
    \end{outputblock}
\end{titledbox}

\subsection{AcWing 886. 求组合数 II}

\begin{titledbox}{AcWing 885. 求组合数 II}
    给定 $n$ 组询问,每组询问给定两个整数 $a, b$,请你输出 $C_a^b \mod (10^9 + 7)$ 的值。

    输入格式:

    第一行包含整数 $n$。 接下来 $n$ 行,每行包含一组 $a$ 和 $b$。

    输出格式:

    共 $n$ 行,每行输出一个询问的解。

    数据范围:

    $1 \le n \le 10000$, $1 \le b \le a \le 10^5$

    \begin{inputblock}
        \inlinecode{3} \\
        \inlinecode{3 1} \\
        \inlinecode{5 3} \\
        \inlinecode{2 2}
    \end{inputblock}
    \begin{outputblock}
        \inlinecode{3} \\
        \inlinecode{10} \\
        \inlinecode{1}
    \end{outputblock}
\end{titledbox}

\subsection{AcWing 887. 求组合数 III}
\begin{titledbox}{AcWing 887. 求组合数 III}
    给定 $n$ 组询问,每组询问给定三个整数 $a, b, p$,其中 $p$ 是质数,请你输出 $C_a^b \bmod p$ 的值。

    输入格式:

    第一行包含整数 $n$。 接下来 $n$ 行,每行包含一组 $a, b, p$。

    输出格式:

    共 $n$ 行,每行输出一个询问的解。

    数据范围:

    $1 \le n \le 20$, $1 \le b \le a \le 10^{18}$, $1 \le p \le 10^5$,

    \begin{inputblock}
        \inlinecode{3} \\
        \inlinecode{5 3 7} \\
        \inlinecode{3 1 5} \\
        \inlinecode{6 4 13}
    \end{inputblock}
    \begin{outputblock}
        \inlinecode{3}
        \inlinecode{3}
        \inlinecode{2}
    \end{outputblock}
\end{titledbox}

\subsection{AcWing 888. 求组合数 IV}
\begin{titledbox}{AcWing 888. 求组合数 IV}
    输入 $a, b$,求 $C_a^b$ 的值。 注意结果可能很大,需要使用高精度计算。

    输入格式:

    共一行,包含两个整数 $a$ 和 $b$。

    输出格式:

    共一行,输出 $C_a^b$ 的值。

    数据范围:

    $1 \le b \le a \le 5000$

    \begin{inputblock}
        \inlinecode{5 3}
    \end{inputblock}
    \begin{outputblock}
        \inlinecode{10}
    \end{outputblock}
\end{titledbox}

\subsection{AcWing 889. 满足条件的01序列}
\begin{titledbox}{AcWing 889. 满足条件的01序列}
    给定 $n$ 个 $0$ 和 $n$ 个 $1$,它们将按照某种顺序排成长度为 $2n$ 的序列,求它们能排列成的所有序列中,能够满足任意前缀序列中 $0$ 的个数都不少于 $1$ 的个数的序列有多少个。 输出的答案对 $10^9+7$ 取模。

    输入格式:

    共一行,包含整数 $n$。

    输出格式:

    共一行,包含一个整数,表示答案。

    数据范围:

    $1 \le n \le 10^5$

    \begin{inputblock}
        \inlinecode{3}
    \end{inputblock}
    \begin{outputblock}
        \inlinecode{5}
    \end{outputblock}
\end{titledbox}


\section{容斥原理}

\subsection{AcWing 890. 能被整除的数}
\begin{titledbox}{AcWing 890. 能被整除的数}
    给定一个整数 $n$ 和 $m$ 个不同的质数 $p_1, p_2, \dots, p_m$。 请你求出 $1 \sim n$ 中能被 $p_1, p_2, \dots, p_m$ 中的至少一个数整除的整数有多少个。

    输入格式:

    第一行包含整数 $n$ 和 $m$。 第二行包含 $m$ 个质数。

    输出格式:

    输出一个整数,表示满足条件的整数的个数。

    数据范围:

    $1 \le m \le 16$, $1 \le n,p_i \le 10^9$

    \begin{inputblock}
        \inlinecode{10 2} \\
        \inlinecode{2 3}
    \end{inputblock}
    \begin{outputblock}
        \inlinecode{7}
    \end{outputblock}
\end{titledbox}


\section{博弈论}

\subsection{AcWing 891. Nim游戏}
\begin{titledbox}{AcWing 891. Nim游戏}
    给定 $n$ 堆石子,两位玩家轮流操作,每次操作可以从任意一堆石子中拿走任意数量的石子(可以拿完,但不能不拿),最后无法进行操作的人视为失败。 问如果两人都采用最优策略,先手是否必胜。

    输入格式:

    第一行包含整数 $n$。第二行包含 $n$ 个数字,其中第 $i$ 个数字表示第 $i$ 堆石子的数量。

    输出格式:

    如果先手方必胜,则输出 \inlinecode{Yes}。 否则,输出 \inlinecode{No}。

    数据范围:

    $1 \le n \le 10^5$, $1 \le 每堆石子数 \le 10^9$

    \begin{inputblock}
        \inlinecode{2} \\
        \inlinecode{2 3}
    \end{inputblock}
    \begin{outputblock}
        \inlinecode{Yes}
    \end{outputblock}
\end{titledbox}

\subsection{AcWing 892. 台阶-Nim游戏}
\begin{titledbox}{AcWing 892. 台阶-Nim游戏}
    现在,有一个 $n$ 级台阶的楼梯,每级台阶上都有若干个石子,其中第 $i$ 级台阶上有 $a_i$ 个石子($i \ge 1$)。 两位玩家轮流操作,每次操作可以从任意一级台阶上拿若干个石子放到下一级台阶中(不能不拿)。 已经拿到地面上的石子不能再拿,最后无法进行操作的人视为失败。 问如果两人都采用最优策略,先手是否必胜。

    输入格式:

    第一行包含整数 $n$。 第二行包含 $n$ 个整数,其中第 $i$ 个整数表示第 $i$ 级台阶上的石子数 $a_i$。

    输出格式:

    如果先手方必胜,则输出 \inlinecode{Yes}。 否则,输出 \inlinecode{No}。

    数据范围:

    $1 \le n \le 10^5$, $1 \le a_i \le 10^9$

    \begin{inputblock}
        \inlinecode{3} \\
        \inlinecode{2 1 3}
    \end{inputblock}
    \begin{outputblock}
        \inlinecode{Yes}
    \end{outputblock}
\end{titledbox}

\subsection{AcWing 893. 集合-Nim游戏}
\begin{titledbox}{AcWing 893. 集合-Nim游戏}
    给定 $n$ 堆石子以及一个由 $k$ 个不同正整数构成的数字集合 $S$。 现在有两位玩家轮流操作,每次操作可以从任意一堆石子中拿取石子,每次拿取的石子数量必须包含于集合 $S$,最后无法进行操作的人视为失败。 问如果两人都采用最优策略,先手是否必胜。

    输入格式:

    第一行包含整数 $k$,表示数字集合 $S$ 中数字的个数。 第二行包含 $k$ 个整数,其中第 $i$ 个整数表示数字集合 $S$ 中的第 $i$ 个数 $s_i$。 第三行包含整数 $n$。 第四行包含 $n$ 个整数,其中第 $i$ 个整数表示第 $i$ 堆石子的数量 $h_i$。

    输出格式:

    如果先手方必胜,则输出 \inlinecode{Yes}。 否则,输出 \inlinecode{No}。

    数据范围:

    $1 \le n, k \le 100$, $1 \le s_i,h_i \le 10000$

    \begin{inputblock}
        \inlinecode{2} \\
        \inlinecode{2 5} \\
        \inlinecode{3} \\
        \inlinecode{2 4 7}
    \end{inputblock}
    \begin{outputblock}
        \inlinecode{Yes}
    \end{outputblock}
\end{titledbox}

\subsection{AcWing 894. 拆分-Nim游戏}
\begin{titledbox}{AcWing 894. 拆分-Nim游戏}
    给定 $n$ 堆石子,两位玩家轮流操作,每次操作可以取走其中的一堆石子,然后放入两堆\textbf{规模更小}的石子(新堆规模可以为 $0$,且两个新堆的石子总数可以大于取走的那堆石子数),最后无法进行操作的人视为失败。 问如果两人都采用最优策略,先手是否必胜。

    输入格式:

    第一行包含整数 $n$。 第二行包含 $n$ 个整数,其中第 $i$ 个整数表示第 $i$ 堆石子的数量 $a_i$。

    输出格式:

    如果先手方必胜,则输出 \inlinecode{Yes}。 否则,输出 \inlinecode{No}。

    数据范围:

    $1 \le n, a_i \le 100$

    \begin{inputblock}
        \inlinecode{2} \\
        \inlinecode{2 3}
    \end{inputblock}
    \begin{outputblock}
        \inlinecode{Yes}
    \end{outputblock}
\end{titledbox}
