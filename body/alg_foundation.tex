\chapter{算法基础}


\section{快速排序}

\subsection{AcWing 785. 快速排序}

\begin{titledbox}{AcWing 785. 快速排序}
    给定你一个长度为 $n$ 的整数数列。请你使用快速排序对这个数列按照从小到大进行排序。并将排好序的数列按顺序输出。

    输入格式:

    输入共两行,第一行包含整数 $n$。第二行包含 $n$ 个整数(所有整数均在 $1 \sim 10^9$ 范围内),表示整个数列。

    输出格式:

    输出共一行,包含 $n$ 个整数,表示排好序的数列。

    数据范围:$1 \le n \le 100000$

    \begin{minipage}[t]{.5\textwidth}
        % \raggedright
        输入样例:\\
        5 \\
        3 1 2 4 5
    \end{minipage}% <---------------- Note the use of "%"
    \begin{minipage}[t]{.5\textwidth}
        % \raggedcenter
        输出样例:\\
        1 2 3 4 5
    \end{minipage}
\end{titledbox}

快速排序算法基于\textbf{分治}算法,以一个数来作为分治的节点。

随机选取数组中的某个元素$x$作为分界点,操作数组中的元素使得数组被分割为两个部分,左边一侧的元素小于等于$x$,右边一侧则大于等于$x$。接下来递归的对左右两侧数组进行操作,直到最小数组只有一个元素则完成排序。

主要步骤如下:
\begin{enumerate}
    \item 确定分界点$x$,可取值:\lstinline{q[l]}, \lstinline{q[r]}, \lstinline{q[(l + r) >> 1]}, random value
    \item 调整数组,使得左边小于等于$x$,右边大于等于$x$
    \item 递归处理左右两段
\end{enumerate}

\begin{mycpptwocol}[quick sort]
    #include <stdio.h>
    #include <stdlib.h>

    void swap(int *q, int a, int b) {
        int tmp = q[a];
        q[a] = q[b];
        q[b] = tmp;
    }

    void quick_sort(int *q, int l, int r)
        {
        if (l >= r) {
            return;
        }
        int x = q[(l + r) >> 1];
        int i = l - 1;
        int j = r + 1;
        while (i < j) {
            do i++; while (q[i] < x);
            do j--; while (q[j] > x);
            if (i < j) {
                swap(q, i, j);
            }
        }
        quick_sort(q, l, j);
        quick_sort(q, j + 1, r);
    }

    int main() {
        int n;
        scanf("%d", &n);
        int *q = (int *)calloc(sizeof(int), n);
        if (q == NULL) {
            return -1;
        }
        for (int i = 0; i < n; i++) {
            scanf("%d", q + i);
        }

        quick_sort(q, 0, n - 1);

        for (int i = 0; i < n; i++) {
            printf("%d ", q[i]);
        }
        free(q);
        return 0;
    }
\end{mycpptwocol}

从上述代码段中可以清晰看到递归处理的过程,每次选取分界点,之后将左右两侧的元素进行调整,此处采用双指针算法。

\begin{keypoint}
    这里有两个问题:
    \begin{enumerate}
        \item 在选择x时选择\lstinline{q[l]}则在递归是不能选用\lstinline{i},会出现边界问题 | \lstinline{i - 1, i}
        \item 在选择x时选择\lstinline{q[r]}则在递归是不能选用\lstinline{j},会出现边界问题 | \lstinline{j, j + 1}
    \end{enumerate}

    边界用例可使用\lstinline{1, 2}这个例子,会有递归不结束的问题
\end{keypoint}

\begin{information}
    该算法\textbf{不稳定},因为\lstinline{q[i]}和\lstinline{q[j]}相等的时候会发生交换。

    这里调整数组的部分是难点,怎么优雅的调整数组?暴力做法可以开辟两个辅助数组来存储。双指针做法优雅简洁。
\end{information}

时间复杂度分析:

\subsection{AcWing 786. 第k个数}
快速选择算法可选出有序数组中的第$k$个数,与快排中逻辑相同,左侧的元素都小于$x$右侧元素都大于$x$。如果左侧元素的数量大于等于$k$则表示第$k$个元素在左侧数组中,反之则在右侧数组中寻找$k-\text{left length}$的元素。

\begin{titledbox}{AcWing 786. 第k个数}
    给定一个长度为 $n$ 的整数数列,以及一个整数 $k$,请用快速选择算法求出数列从小到大排序后的第 $k$ 个数。

    输入格式:

    第一行包含两个整数 $n$ 和 $k$。
    第二行包含 $n$ 个整数(所有整数均在 $1 \sim 10^9$ 范围内),表示整数数列。

    输出格式:

    输出一个整数,表示数列的第 $k$ 小数。

    数据范围:

    $1 \le n \le 100000$,
    $1 \le k \le n$

    \begin{minipage}[t]{.5\textwidth}
        % \raggedright
        输入样例:\\
        5 3\\
        3 1 2 4 5
    \end{minipage}% <---------------- Note the use of "%"
    \begin{minipage}[t]{.5\textwidth}
        % \raggedcenter
        输出样例:\\
        3
    \end{minipage}
\end{titledbox}


\begin{mycpptwocol}[find kth smallest number]
    #include <stdio.h>
    #include <stdlib.h>

    int q_select(int *q, int l,
    int r, int k) {
        if (r <= l) {
            return q[l];
        }

        int x = q[(l + r) >> 1];
        int i = l - 1;
        int j = r + 1;

        while (i < j) {
            do i++; while(q[i] < x);
            do j--; while(q[j] > x);

            if (i < j) {
                int tmp = q[i];
                q[i] = q[j];
                q[j] = tmp;
            }
        }

        int length = j - l + 1;
        if (length < k) {
            return q_select(q, j + 1, r,
            k - length);
        } else {
            return q_select(q, l, j, k);
        }
    }

    int main()
        {
        int n;
        int k;
        scanf("%d %d", &n, &k);
        int *q = (int *)calloc(sizeof(int), n);
        if (q == NULL) {
            return -1;
        }
        for (int i = 0; i < n; i++) {
            scanf("%d", q + i);
        }
        int ret = q_select(q, 0, n - 1, k);
        printf("%d", ret);
        return 0;
    }
\end{mycpptwocol}


\section{归并排序}
归并排序同样是基于\textbf{分治}算法,不过是以整个数组的中间位置来分。

将数组分割成两个已经分别排序好的有序数组,再将其二者合并即可。此方法需要有单独的空间来存放合并的临时结果,再将临时结果写入到原始区域中。

主要步骤如下:
\begin{enumerate}
    \item 确定分界点,\lstinline{mid = (l + r) >> 1}
    \item 递归排序左右两边
    \item 归并,将两个有序的子数组合二为一
\end{enumerate}

\subsection{AcWing 787. 归并排序}
\begin{titledbox}{AcWing 787. 归并排序}
    给定你一个长度为 $n$ 的整数数列。请你使用归并排序对这个数列按照从小到大进行排序。并将排好序的数列按顺序输出。

    输入格式:

    输入共两行,第一行包含整数 $n$。第二行包含 $n$ 个整数(所有整数均在 $1 \sim 10^9$ 范围内),表示整个数列。

    输出格式:

    输出共一行,包含 $n$ 个整数,表示排好序的数列。

    数据范围:$1 \le n \le 100000$

    \begin{inputblock}
        5 \\
        3 1 2 4 5
    \end{inputblock}% <---------------- Note the use of "%"
    \begin{minipage}[t]{.5\textwidth}
        % \raggedcenter
        输出样例:\\
        1 2 3 4 5
    \end{minipage}
\end{titledbox}

\begin{mycpptwocol}[merge sort]
    #include <stdio.h>
    #include <stdlib.h>

    #define N 100010

    int backup[N];

    void merge_sort(int *q, int l, int r)
        {
        if (r <= l) {
            return;
        }
        int mid = (l + r) >> 1;
        merge_sort(q, l, mid);
        merge_sort(q, mid + 1, r);
        int k = 0;
        int i = l;
        int j = mid + 1;
        while (i <= mid && j <= r) {
            if (q[i] <= q[j]) {
                backup[k++] = q[i++];
            }
            if (q[j] < q[i]) {
                backup[k++] = q[j++];
            }
        }
        while (i <= mid) {
            backup[k++] = q[i++];
        }
        while (j <= r) {
            backup[k++] = q[j++];
        }

        for (i = l, j = 0; j < k; i++, j++) {
            q[i] = backup[j];
        }
    }

    int main()
        {
        int n;
        scanf("%d", &n);
        int *q = (int *)calloc(sizeof(int), n);
        if (q == NULL) {
            return -1;
        }
        for (int i = 0; i < n; i++) {
            scanf("%d", q + i);
        }
        merge_sort(q, 0, n - 1);
        for (int i = 0; i < n; i++) {
            printf("%d ", q[i]);
        }
        return 0;
    }
\end{mycpptwocol}

双指针算法做归并

\begin{keypoint}
    这里归并两个子数组之后要写回去,\lstinline{backup}数组只是临时存储使用。
\end{keypoint}

\subsection{AcWing 788. 逆序对的数量}
\begin{titledbox}{AcWing 788. 逆序对的数量}
    给定一个长度为 $n$ 的整数数列,请你计算数列中的逆序对的数量。
    逆序对的定义如下:对于数列的第 $i$ 个和第 $j$ 个元素,如果满足 $i < j$ 且 $a[i] > a[j]$,则其为一个逆序对;否则不是。

    输入格式:

    第一行包含整数 $n$,表示数列的长度。第二行包含 $n$ 个整数,表示整个数列。

    输出格式:

    输出一个整数,表示逆序对的个数。

    数据范围

    $1 \le n \le 100000$,
    数列中的元素的取值范围 $[1,10^9]$。

    \begin{minipage}[t]{.5\textwidth}
        % \raggedright
        输入样例:\\
        2 3 4 5 6 1
    \end{minipage}% <---------------- Note the use of "%"
    \begin{minipage}[t]{.5\textwidth}
        % \raggedcenter
        输出样例:\\
        5
    \end{minipage}
\end{titledbox}

分治思路,将整个区间一分为二。考虑到归并排序的时候需要将两个有序数组合并,此时恰好可以做逆序对的统计。假设有一种算法,可以将数组排序的过程中统计该数组中的逆序对数量,则问题变为怎么统计两个有序数组中合起来的逆序对。

\begin{mycpptwocol}[归并排序计算逆序对数量]
    #include <stdio.h>
    #include <stdlib.h>

    long long merge_sort(int *q, int *tmp,
    int l, int r) {
        if (l >= r) {
            return 0;
        }
        int mid = (l + r) >> 1;
        // 左侧的数组已统计逆序对且已经排序,右侧同样
        long long res = merge_sort(q, tmp, l, mid) + merge_sort(q, tmp, mid + 1, r);
        // 统计两个有序数组合起来的逆序对数量
        int i = l;
        int j = mid + 1;
        int k = 0;
        while (i <= mid && j <= r) {
            if (q[i] > q[j]) {
                res += mid - i + 1;
                tmp[k++] = q[j++];
            } else {
                tmp[k++] = q[i++];
            }
        }

        while (i <= mid) {
            tmp[k++] = q[i++];
        }
        while (j <= r) {
            tmp[k++] = q[j++];
        }
        for (i = l, j = 0; i <= r;
        i++, j++) {
            q[i] = tmp[j];
        }

        return res;
    }

    int main()
        {
        int n;
        scanf("%d", &n);
        int *q = (int *)calloc(n, sizeof(int));
        if (q == NULL) {
            return -1;
        }
        int *tmp = (int *)calloc(n, sizeof(int));
        if (tmp == NULL) {
            free(q);
            return -1;
        }
        for(int i = 0; i < n; i++) {
            scanf("%d", &q[i]);
        }

        printf("%ld", merge_sort(q, tmp, 0, n - 1));

        return 0;
    }
\end{mycpptwocol}


\section{二分}
整数二分和浮点数二分,二分即查找一个边界值,在左侧满足某种性质,右侧不满足。

二分用模版如下:
\begin{mycpptwocol}[二分模版]
    // 区间[l, r]被划分为[l, mid] 和
    // [mid + 1, r]时使用, 往左找
    int bsearch_1(int l, int r)
        {
        while (l < r) {
            mid = (l + r) / 2;
            if (Check(mid)) {
                r = mid;
            } else {
                l = mid + 1;
            }
        }
    }
    // 区间[l, r]被划分为[l, mid - 1] 和
    // [mid, r]时使用,往右找
    int bsearch_2(int l, int r)
        {
        while (l < r) {
            mid = (l + r + 1) / 2;
            if (Check(mid)) {
                l = mid;
            } else {
                r = mid - 1;
            }
        }
    }
\end{mycpptwocol}

\begin{keypoint}
    每次要保证答案在区间中。

    第二个模版加一的原因在于,如果某次循环结束后,\lstinline{l = r - 1},如果不加1,此时因为向下取整的缘故\lstinline{mid = l},check成功后\lstinline{l}被再次赋值为\lstinline{mid}即\lstinline{l},则此时进入死循环。
\end{keypoint}

\subsection{AcWing 789. 数的范围}
\begin{titledbox}{AcWing 789. 数的范围}
    给定一个按照升序排列的长度为 $n$ 的整数数组,以及 $q$ 个查询。对于每个查询,返回一个元素 $k$ 的起始位置和终止位置(位置从 $0$ 开始计数)。如果数组中不存在该元素,则返回\lstinline{-1 -1}。

    输入格式:

    第一行包含整数 $n$ 和 $q$,表示数组长度和询问个数。第二行包含 $n$ 个整数(均在 $1 \sim 10000$ 范围内),表示完整数组。接下来 $q$ 行,每行包含一个整数 $k$,表示一个询问元素。

    输出格式:

    共 $q$ 行,每行包含两个整数,表示所求元素的起始位置和终止位置。

    数据范围
    $1 \le n \le 100000$,
    $1 \le q \le 10000$,
    $1 \le k \le 10000$

    \begin{minipage}[t]{.5\textwidth}
        % \raggedright
        输入样例:\\
        6 3 \\
        1 2 2 3 3 4 \\
        3 \\
        4 \\
        5
    \end{minipage}% <---------------- Note the use of "%"
    \begin{minipage}[t]{.5\textwidth}
        % \raggedcenter
        输出样例:\\
        3 4 \\
        5 5 \\
        -1 -1
    \end{minipage}
\end{titledbox}


\begin{mycpptwocol}[binary search]
    #include <stdio.h>
    #include <stdlib.h>

    int b_search_l(int *q, int l,
    int r, int t) {
        while (l < r) {
            int mid = (l + r) >> 1;
            if (q[mid] >= t) {
                r = mid;
            } else {
                l = mid + 1;
            }
        }
        if (q[l] != t) {
            return -1;
        }
        return l;
    }

    int b_search_r(int *q, int l,
    int r, int t) {
        while (l < r) {
            int mid = (l + r) / 2 + 1;
            if (q[mid] <= t) {
                l = mid;
            } else {
                r = mid - 1;
            }
        }
        if (q[l] != t) {
            return -1;
        }
        return l;
    }

    int main()
        {
        int n;
        int q;
        scanf("%d %d", &n, &q);
        int *q = (int *)calloc(sizeof(int), n);
        if (q == NULL) {
            return -1;
        }
        for (int i = 0; i < n; i++) {
            scanf("%d", q + i);
        }
        while(q--) {
            int t;
            scanf("%d", &t);
            printf("%d %d\n",
            b_search_l(q, 0, n - 1, t),
            b_search_r(q, 0, n - 1, t));
        }
        return 0;
    }
\end{mycpptwocol}

\begin{exclamation}
    这里不能使用\lstinline{bsearch}函数来完成左端点的搜索,因为该函数在面对重复值时返回值不确定,是未定义行为。
\end{exclamation}

\subsection{AcWing 790. 数的三次方根}
\begin{titledbox}{AcWing 790. 数的三次方根}
    给定一个浮点数 $n$,求它的三次方根。

    输入格式:

    共一行,包含一个浮点数 $n$。

    输出格式:

    共一行,包含一个浮点数,表示问题的解。

    注意,结果保留 $6$ 位小数。

    数据范围:

    $-10000 \le n \le 10000$

    \begin{minipage}[t]{.5\textwidth}
        % \raggedright
        输入样例:\\
        1000.00
    \end{minipage}% <---------------- Note the use of "%"
    \begin{minipage}[t]{.5\textwidth}
        % \raggedcenter
        输出样例:\\
        10.000000
    \end{minipage}
\end{titledbox}

\begin{mycpptwocol}[数的三次方根]
    #include <stdio.h>

    #define N 10000

    int main()
        {
        double n;
        scanf("%lf", &n);
        double l = 0 - N;
        double r = N;

        while (r - l > 1e-8) {
            double mid = (l + r) / 2;
            if (mid * mid * mid < n) {
                l = mid;
            } else {
                r = mid;
            }
        }
        printf("%.6f", l);
        return 0;
    }
\end{mycpptwocol}


\section{高精度}


\section{前缀和与差分}
\begin{enumerate}
    \item 前缀和可方便地求取数组中某个区间的元素和,或者矩阵的子矩阵的和 O(1)
    \item 差分和方便的将数组某个区间的所有元素加上一个数,O(1)时间复杂度
\end{enumerate}

\subsection{AcWing 795. 前缀和}
\begin{titledbox}{AcWing 795. 前缀和}
    输入一个长度为 $n$ 的整数序列。接下来再输入 $m$ 个询问,每个询问输入一对 $l, r$。对于每个询问,输出原序列中从第 $l$ 个数到第 $r$ 个数的和。

    输入格式:

    第一行包含两个整数 $n$ 和 $m$。第二行包含 $n$ 个整数,表示整数数列。
    接下来 $m$ 行,每行包含两个整数 $l$ 和 $r$,表示一个询问的区间范围。

    输出格式:

    共 $m$ 行,每行输出一个询问的结果。

    数据范围:

    $1 \le l \le r \le n$, $1 \le n,m \le 100000$, $-1000 \le \text{数列中元素的值} \le 1000$

    \begin{minipage}[t]{.5\textwidth}
        % \raggedright
        输入样例:\\
        5 3 \\
        2 1 3 6 4 \\
        1 2 \\
        1 3 \\
        2 4
    \end{minipage}% <---------------- Note the use of "%"
    \begin{minipage}[t]{.5\textwidth}
        % \raggedcenter
        输出样例:\\
        3 \\
        6 \\
        10
    \end{minipage}
\end{titledbox}

\begin{mycpponecol}[前缀和]
    #include <stdio.h>
    #include <stdlib.h>

    int main()
        {
        int n;
        int m;
        scanf("%d %d", &n, &m);
        int *q = (int *)calloc(n + 1, sizeof(int));
        int *preSum = (int *)calloc(n + 1, sizeof(int));
        for (int i = 1; i <= n; i++) {
            scanf("%d", q + i);
            preSum[i] = preSum[i - 1] + q[i];
        }
        while (m--) {
            int l;
            int r;
            scanf("%d %d", &l, &r);
            printf("%d\n", preSum[r] - preSum[l - 1]);
        }
        return 0;
    }
\end{mycpponecol}

\subsection{AcWing 796. 子矩阵的和}
子矩阵
\begin{problembox}{AcWing 796. 子矩阵的和}
    \small{子矩阵的和}
    输入一个 $n$ 行 $m$ 列的整数矩阵,再输入 $q$ 个询问,每个询问包含四个整数 $x_1$, $y_1$, $x_2$, $y_2$,表示一个子矩阵的左上角坐标和右下角坐标。对于每个询问输出子矩阵中所有数的和。

    输入格式:

    第一行包含三个整数 $n$,$m$,$q$。接下来 $n$ 行,每行包含 $m$ 个整数,表示整数矩阵。接下来 $q$ 行,每行包含四个整数 $x_1, y_1, x_2, y_2$,表示一组询问。

    输出格式:

    共 $q$ 行,每行输出一个询问的结果。

    数据范围:

    $1 \le n,m \le 1000$,
    $1 \le q \le 200000$,
    $1 \le x_1 \le x_2 \le n$,
    $1 \le y_1 \le y_2 \le m$,
    $-1000 \le 矩阵内元素的值 \le 1000$

    \begin{inputblock}
        3 4 3 \\
        1 7 2 4 \\
        3 6 2 8 \\
        2 1 2 3 \\
        1 1 2 2 \\
        2 1 3 4\\
        1 3 3 4
    \end{inputblock}% <---------------- Note the use of "%"
    \begin{minipage}[t]{.5\textwidth}
        % \raggedcenter
        输出样例:\\
        17 \\
        27 \\
        21
    \end{minipage}
\end{problembox}

\begin{mycpponecol}[子矩阵的和]
    #include<stdio.h>

    #define N 1010

    int mat[N][N];
    int preMat[N][N];

    int main()
        {
        int n, m, q;
        scanf("%d %d %d", &n, &m, &q);
        for (int i = 1; i <= n; i++) {
            for (int j = 1; j <= m; j++) {
                scanf("%d", &mat[i][j]);
                preMat[i][j] = preMat[i - 1][j] + preMat[i][j - 1] - preMat[i - 1][j - 1] + mat[i][j];
            }
        }
        while (q--) {
            int x1, y1, x2, y2;
            scanf("%d %d %d %d", &x1, &y1, &x2, &y2);
            printf("%d\n", preMat[x2][y2] - preMat[x2][y1 - 1] - preMat[x1 - 1][y2] + preMat[x1 - 1][y1 - 1]);
        }
        return 0;
    }
\end{mycpponecol}

\subsection{AcWing 797. 差分}
\begin{titledbox}{AcWing 797. 差分}
    输入一个长度为 $n$ 的整数序列。
    接下来输入 $m$ 个操作,每个操作包含三个整数 $l, r, c$,表示将序列中 $[l, r]$ 之间的每个数加上 $c$。
    请你输出进行完所有操作后的序列。

    输入格式:\\
    第一行包含两个整数 $n$ 和 $m$。
    第二行包含 $n$ 个整数,表示整数序列。
    接下来 $m$ 行,每行包含三个整数 $l,r,c$,表示一个操作。

    输出格式:\\
    共一行,包含 $n$ 个整数,表示最终序列。

    数据范围:\\
    $1 \le n,m \le 100000$,
    $1 \le l \le r \le n$,
    $-1000 \le c \le 1000$,
    $-1000 \le \text{整数序列中元素的值} \le 1000$

    \begin{inputblock}
        6 3 \\
        1 2 2 1 2 1 \\
        1 3 1 \\
        3 5 1 \\
        1 6 1 \\
    \end{inputblock}
    \begin{outputblock}
        3 4 5 3 4 2
    \end{outputblock}
\end{titledbox}

\begin{mycpptwocol}[差分]
    #include <stdio.h>
    #include <stdlib.h>

    void Insert(int *diff, int l, int r, int c)
        {
        diff[l] += c;
        diff[r + 1] -= c;
    }

    int main()
        {
        int n;
        int m;
        scanf("%d %d", &n, &m);
        int *q = (int *)calloc(n + 1, sizeof(int));
        int *diff = (int *)calloc(n + 1, sizeof(int));
        for (int i = 1; i <= n; i++) {
            scanf("%d", q + i);
            Insert(diff, i, i, q[i]);
        }
        while (m--) {
            int l;
            int r;
            int c;
            scanf("%d %d %d", &l, &r, &c);
            Insert(diff, l, r, c);
        }
        for (int i = 1; i <= n; i++) {
            diff[i] += diff[i - 1];
            printf("%d ", diff[i]);
        }
        free(diff);
        free(q);
        return 0;
    }
\end{mycpptwocol}

\subsection{AcWing 798. 差分矩阵}
\begin{titledbox}{AcWing 798. 差分矩阵}
    输入一个 $n$ 行 $m$ 列的整数矩阵,再输入 $q$ 个操作,每个操作包含五个整数 $x_1, y_1, x_2, y_2, c$,其中 $(x_1, y_1)$ 和 $(x_2, y_2)$ 表示一个子矩阵的左上角坐标和右下角坐标。每个操作都要将选中的子矩阵中的每个元素的值加上 $c$。请你将进行完所有操作后的矩阵输出。

    输入格式:

    第一行包含整数 $n,m,q$。接下来 $n$ 行,每行包含 $m$ 个整数,表示整数矩阵。接下来 $q$ 行,每行包含 $5$ 个整数 $x_1, y_1, x_2, y_2, c$,表示一个操作。

    输出格式:

    共 $n$ 行,每行 $m$ 个整数,表示所有操作进行完毕后的最终矩阵。

    数据范围:

    $1 \le n,m \le 1000$, $1 \le q \le 100000$, $1 \le x_1 \le x_2 \le n$, $1 \le y_1 \le y_2 \le m$, $-1000 \le c \le 1000$, $-1000 \le 矩阵内元素的值 \le 1000$

    \begin{inputblock}
        3 4 3 \\
        1 2 2 1 \\
        3 2 2 1 \\
        1 1 1 1 \\
        1 1 2 2 1 \\
        1 3 2 3 2 \\
        3 1 3 4 1
    \end{inputblock}
    \begin{outputblock}
        4 3 4 1 \\
        2 2 2 2
    \end{outputblock}
\end{titledbox}

\begin{mycpptwocol}[差分矩阵]
    #include <stdio.h>
    #include <stdlib.h>

    #define N 1010

    int arr[N][N];
    int diff[N][N];

    void Insert(int x1, int y1, int x2, int y2, int c)
        {
        diff[x1][y1] += c;
        diff[x1][y2 + 1] -= c;
        diff[x2 + 1][y1] -= c;
        diff[x2 + 1][y2 + 1] += c;
    }

    int main()
        {
        int n;
        int m;
        int q;
        scanf("%d %d %d", &n, &m, &q);
        for (int i = 1; i <= n; i++) {
            for (int j = 1; j <= m; j++) {
                scanf("%d", &arr[i][j]);
                Insert(i, j, i, j, arr[i][j]);
            }
        }
        while (q--) {
            int x1, y1, x2, y2, c;
            scanf("%d %d %d %d %d", &x1, &y1, &x2, &y2, &c);
            Insert(x1, y1, x2, y2, c);
        }

        // construct origin matrix
        for (int i = 1; i <= n; i++) {
            for (int j = 1; j <= m; j++) {
                diff[i][j] += diff[i - 1][j] + diff[i][j - 1] - diff[i - 1][j - 1];
                printf("%d ", diff[i][j]);
            }
            printf("\n");
        }
        return 0;
    }
\end{mycpptwocol}


\section{双指针算法}
归并排序中的双指针,分别指向两个序列;指向同一个序列的不同位置,同向移动或者相向而行

\subsection{AcWing 799. 最长连续不重复子序列}
\begin{titledbox}{AcWing 799. 最长连续不重复子序列}
    给定一个长度为 $n$ 的整数序列,请找出最长的不包含重复的数的连续区间,输出它的长度。

    输入格式:

    第一行包含整数 $n$。第二行包含 $n$ 个整数(均在 $0 \sim 10^5$ 范围内),表示整数序列。

    输出格式:

    共一行,包含一个整数,表示最长的不包含重复的数的连续区间的长度。

    数据范围:

    $1 \le n \le 10^5$

    \begin{inputblock}
        5 \\
        1 2 2 3 5
    \end{inputblock}
    \begin{outputblock}
        3
    \end{outputblock}
\end{titledbox}
滑动窗口,需要有个visited数组来统计每个元素的出现数量

\begin{mycpptwocol}[最长连续不重复子序列]
    #include <stdio.h>
    #include <stdlib.h>

    #define N 100010
    int visited[N];

    int main()
        {
        int n;
        scanf("%d", &n);
        int *q = (int *)calloc(n, sizeof(int));
        if (q == NULL) {
            return -1;
        }
        for (int i = 0; i < n; i++) {
            scanf("%d", &q[i]);
        }
        int start = 0;
        int end = 0;
        int max = 0;
        for (end = 0; end < n; end++) {
            // 标记当前元素
            visited[q[end]] += 1;
            while (visited[q[end]] > 1) {
                // 如果当前元素重复出现,缩小窗口
                visited[q[start]]--;
                start++;
            }
            max = max > (end - start + 1) ? max : (end - start + 1);
        }
        printf("%d", max);
        free(arr);
        return 0;
    }
\end{mycpptwocol}

\subsection{AcWing 800. 数组元素的目标和}
\begin{titledbox}{AcWing 800. 数组元素的目标和}
    给定两个升序排序的有序数组 $A$ 和 $B$,以及一个目标值 $x$。数组下标从 $0$ 开始。
    请你求出满足 $A[i] + B[j] = x$ 的数对 $(i, j)$。数据保证有唯一解。

    输入格式:

    第一行包含三个整数 $n,m,x$,分别表示 $A$ 的长度,$B$ 的长度以及目标值 $x$。第二行包含 $n$ 个整数,表示数组 $A$。第三行包含 $m$ 个整数,表示数组 $B$。

    输出格式:

    共一行,包含两个整数 $i$ 和 $j$。

    数据范围:

    数组长度不超过 $10^5$。同一数组内元素各不相同。$1 \le \text{数组元素} \le 10^9$

    \begin{inputblock}
        4 5 6 \\
        1 2 4 7 \\
        3 4 6 8 9
    \end{inputblock}
    \begin{outputblock}
        1 1
    \end{outputblock}
\end{titledbox}

\begin{mycpptwocol}[数组元素的目标和]
    #include <stdio.h>
    #include <stdlib.h>

    int main()
        {
        int n;
        int m;
        int x;
        scanf("%d %d %d", &n, &m, &x);
        int *a = (int *)calloc(n, sizeof(int));
        int *b = (int *)calloc(m, sizeof(int));
        for (int i = 0; i < n; i++) {
            scanf("%d", a + i);
        }
        for (int i = 0; i < m; i++) {
            scanf("%d", b + i);
        }

        int i = 0;
        int j = m - 1;
        while (i < n && j >= 0) {
            int sum = a[i] + b[j];
            if (sum == x) {
                printf("%d %d", i, j);
                return 0;
            } else if (sum > x) {
                j--;
            } else {
                i++;
            }
        }
        free(a);
        free(b);
        return 0;
    }
\end{mycpptwocol}

\subsection{AcWing 2816. 判断子序列}
\begin{titledbox}{AcWing 2816. 判断子序列}
    给定一个长度为 $n$ 的整数序列 $a_1,a_2,\dots;,a_n$ 以及一个长度为 $m$ 的整数序列 $b_1,b_2,\dots;,b_m$。</p>
    请你判断 $a$ 序列是否为 $b$ 序列的子序列。子序列指序列的一部分项按\textbf{原有次序排列}而得的序列,例如序列 $\{a_1,a_3,a_5\}$ 是序列 $\{a_1,a_2,a_3,a_4,a_5\}$ 的一个子序列。

    输入格式:

    第一行包含两个整数 $n,m$。第二行包含 $n$ 个整数,表示 $a_1,a_2,\dots;,a_n$。第三行包含 $m$ 个整数,表示 $b_1,b_2,\dots;,b_m$。

    输出格式:

    如果 $a$ 序列是 $b$ 序列的子序列,输出一行 Yes。否则,输出 No。

    数据范围:

    $1 \le n \le m \le 10^5$, $-10^9 \le a_i,b_i \le 10^9$

    \begin{inputblock}
        3 5 \\
        1 3 5 \\
        1 2 3 4 5
    \end{inputblock}
    \begin{outputblock}
        Yes
    \end{outputblock}
\end{titledbox}

\begin{mycpptwocol}[判断子序列]
    #include <stdio.h>
    #include <stdlib.h>
    #include <stdbool.h>

    bool IsSubSeq(int *a, int n, int *b, int m)
        {
        int i = 0;
        int j = 0;
        while (i < n && j < m) {
            if (a[i] == b[j]) {
                i++;
                j++;
            } else {
                j++;
            }
        }
        return i == n;
    }

    int main()
        {
        int n;
        int m;
        scanf("%d %d", &n, &m);
        int *a = (int *)calloc(n, sizeof(int));
        int *b = (int *)calloc(m, sizeof(int));
        for (int i = 0; i < n; i++) {
            scanf("%d", a + i);
        }
        for (int i = 0; i < m; i++) {
            scanf("%d", b + i);
        }
        printf("%s",
        IsSubSeq(a, n, b, m) ? "Yes" : "No");
        free(a);
        free(b);
        return 0;
    }
\end{mycpptwocol}


\section{位运算}

\subsection{AcWing 801. 二进制中1的个数}
\begin{titledbox}{AcWing 801. 二进制中1的个数}
    给定一个长度为 $n$ 的数列,请你求出数列中每个数的二进制表示中 $1$ 的个数。

    输入格式:

    第一行包含整数 $n$。第二行包含 $n$ 个整数,表示整个数列。

    输出格式:
    共一行,包含 $n$ 个整数,其中的第 $i$ 个数表示数列中的第 $i$ 个数的二进制表示中 $1$ 的个数。

    数据范围:

    $1 \le n \le 100000$, $0 \le \text{数列中元素的值} \le 10^9$

    \begin{inputblock}
        5 \\
        1 2 3 4 5
    \end{inputblock}
    \begin{outputblock}
        1 1 2 1 2
    \end{outputblock}
\end{titledbox}

\begin{mycpptwocol}[LowBit 运算]
    #include <stdio.h>
    #include <stdlib.h>

    int LowBit(int x)
        {
        return x & (-x);
    }

    int Count(int x)
        {
        int cnt = 0;
        while (x != 0) {
            cnt++;
            x -= LowBit(x);
        }
        return cnt;
    }

    int main()
        {
        int n;
        scanf("%d", &n);
        while (n--) {
            int x;
            scanf("%d", &x);
            printf("%d ", Count(x));
        }
        return 0;
    }
\end{mycpptwocol}


\section{离散化}

\subsection{AcWing 802. 区间和}
\begin{titledbox}{AcWing 802. 区间和}
    假定有一个无限长的数轴,数轴上每个坐标上的数都是 $0$。现在,我们首先进行 $n$ 次操作,每次操作将某一位置 $x$ 上的数加 $c$。接下来,进行 $m$ 次询问,每个询问包含两个整数 $l$ 和 $r$,你需要求出在区间 $[l, r]$ 之间的所有数的和。

    输入格式:

    第一行包含两个整数 $n$ 和 $m$。接下来 $n$ 行,每行包含两个整数 $x$ 和 $c$。再接下来 $m$ 行,每行包含两个整数 $l$ 和 $r$。

    输出格式:

    共 $m$ 行,每行输出一个询问中所求的区间内数字和。

    数据范围:

    $-10^9 \le x \le 10^9$, $1 \le n,m \le 10^5$, $-10^9 \le l \le r \le 10^9$,$-10000 \le c \le 10000$

    \begin{inputblock}
        3 3 \\
        1 2 \\
        3 6 \\
        7 5 \\
        1 3 \\
        4 6 \\
        7 8
    \end{inputblock}
    \begin{outputblock}
        8 \\
        0 \\
        5
    \end{outputblock}
\end{titledbox}


\section{区间合并}

\subsection{AcWing 803. 区间合并}
\begin{titledbox}{AcWing 803. 区间合并}
    给定 $n$ 个区间 $[l_i, r_i]$,要求合并所有有交集的区间。注意如果在端点处相交,也算有交集。
    输出合并完成后的区间个数。例如:$[1,3]$ 和 $[2,6]$ 可以合并为一个区间 $[1,6]$。

    输入格式:

    第一行包含整数 $n$。接下来 $n$ 行,每行包含两个整数 $l$ 和 $r$。

    输出格式:

    共一行,包含一个整数,表示合并区间完成后的区间个数。

    数据范围:

    $1 \le n \le 100000$, $-10^9 \le l_i \le r_i \le 10^9$

    \begin{inputblock}
        5 \\
        1 2 \\
        2 4 \\
        5 6 \\
        7 8 \\
        7 9
    \end{inputblock}
    \begin{outputblock}
        3
    \end{outputblock}
\end{titledbox}

\begin{mycpptwocol}[区间合并]
    #include <stdio.h>
    #include <stdlib.h>

    typedef struct {
        int l;
        int r;
    } Seg;

    int Max(int a, int b)
        {
        return a > b ? a : b;
    }

    int CmpFunc(const void *a, const void *b)
        {
        return ((Seg *)a)->l - ((Seg *)b)->l;
    }

    int main()
        {
        int n;
        scanf("%d", &n);
        Seg *segs = (Seg *)calloc(n, sizeof(Seg));
        for (int i = 0; i < n; i++) {
            scanf("%d %d", &(segs[i].l), &(segs[i].r));
        }
        qsort(segs, n, sizeof(Seg), CmpFunc);
        Seg init = segs[0];
        int ans = 1;
        for (int i = 1; i < n; i++) {
            if (segs[i].l > init.r) {
                ans++;
                init = segs[i];
            } else {
                init.r = Max(segs[i].r, init.r);
            }
        }
        printf("%d", ans);
        return 0;
    }
\end{mycpptwocol}