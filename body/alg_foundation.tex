\chapter{算法基础}

\section{快速排序}
\subsection{AcWing 785. 快速排序}

\begin{titledbox}{AcWing 785. 快速排序}
  给定你一个长度为 $n$ 的整数数列。请你使用快速排序对这个数列按照从小到大进行排序。并将排好序的数列按顺序输出。

  输入格式:

  输入共两行,第一行包含整数 $n$。第二行包含 $n$ 个整数(所有整数均在 $1 \sim 10^9$ 范围内),表示整个数列。

  输出格式:

  输出共一行,包含 $n$ 个整数,表示排好序的数列。

  数据范围:$1 \le n \le 100000$

  \begin{minipage}[t]{.5\textwidth}
    % \raggedright
    输入样例:\\
    5 \\
    3 1 2 4 5
    \end{minipage}% <---------------- Note the use of "%"
    \begin{minipage}[t]{.5\textwidth}
    % \raggedcenter
    输出样例:\\
    1 2 3 4 5
  \end{minipage}
\end{titledbox}

快速排序算法基于\textbf{分治}算法,以一个数来作为分治的节点。

随机选取数组中的某个元素$x$作为分界点,操作数组中的元素使得数组被分割为两个部分,左边一侧的元素小于等于$x$,右边一侧则大于等于$x$。接下来递归的对左右两侧数组进行操作,直到最小数组只有一个元素则完成排序。

主要步骤如下:
\begin{enumerate}
  \item 确定分界点$x$,可取值:\lstinline{q[l]}, \lstinline{q[r]}, \lstinline{q[(l + r) >> 1]}, random value
  \item 调整数组,使得左边小于等于$x$,右边大于等于$x$
  \item 递归处理左右两段
\end{enumerate}

\begin{mycpptwocol}[quick sort]
#include <stdio.h>
#include <stdlib.h>

void swap(int *q, int a, int b) {
    int tmp = q[a];
    q[a] = q[b];
    q[b] = tmp;
}

void quick_sort(int *q, int l, int r)
{
    if (l >= r) {
        return;
    }
    int x = q[(l + r) >> 1];
    int i = l - 1;
    int j = r + 1;
    while (i < j) {
        do i++; while (q[i] < x);
        do j--; while (q[j] > x);
        if (i < j) {
            swap(q, i, j);
        }
    }
    quick_sort(q, l, j);
    quick_sort(q, j + 1, r);
}

int main() {
    int n;
    scanf("%d", &n);
    int *q = (int *)calloc(sizeof(int), n);
    if (q == NULL) {
        return -1;
    }
    for (int i = 0; i < n; i++) {
        scanf("%d", q + i);
    }

    quick_sort(q, 0, n - 1);

    for (int i = 0; i < n; i++) {
        printf("%d ", q[i]);
    } 
    free(q);
    return 0;
}
\end{mycpptwocol}

从上述代码段中可以清晰看到递归处理的过程,每次选取分界点,之后将左右两侧的元素进行调整,此处采用双指针算法。

\begin{keypoint}
    这里有两个问题:
    \begin{enumerate}
        \item 在选择x时选择\lstinline{q[l]}则在递归是不能选用\lstinline{i},会出现边界问题 | \lstinline{i - 1, i}
        \item 在选择x时选择\lstinline{q[r]}则在递归是不能选用\lstinline{j},会出现边界问题 | \lstinline{j, j + 1}
    \end{enumerate}

    边界用例可使用\lstinline{1, 2}这个例子,会有递归不结束的问题
\end{keypoint}

\begin{information}
  该算法\textbf{不稳定},因为\lstinline{q[i]}和\lstinline{q[j]}相等的时候会发生交换。

  这里调整数组的部分是难点,怎么优雅的调整数组?暴力做法可以开辟两个辅助数组来存储。双指针做法优雅简洁。
\end{information}

时间复杂度分析:

\subsection{AcWing 786. 第k个数}
快速选择算法可选出有序数组中的第$k$个数,与快排中逻辑相同,左侧的元素都小于$x$右侧元素都大于$x$。如果左侧元素的数量大于等于$k$则表示第$k$个元素在左侧数组中,反之则在右侧数组中寻找$k-\text{left length}$的元素。

\begin{titledbox}{AcWing 786. 第k个数}
给定一个长度为 $n$ 的整数数列,以及一个整数 $k$,请用快速选择算法求出数列从小到大排序后的第 $k$ 个数。

输入格式:

第一行包含两个整数 $n$ 和 $k$。
第二行包含 $n$ 个整数(所有整数均在 $1 \sim 10^9$ 范围内),表示整数数列。

输出格式:

输出一个整数,表示数列的第 $k$ 小数。

数据范围:

$1 \le n \le 100000$,
$1 \le k \le n$

输入样例:

5 3

2 4 1 5 3


输出样例:

3
\end{titledbox}


\begin{mycpptwocol}[find kth smallest number]
#include <stdio.h>
#include <stdlib.h>

int q_select(int *q, int l,
             int r, int k) {
    if (r <= l) {
        return q[l];
    }

    int x = q[(l + r) >> 1];
    int i = l - 1;
    int j = r + 1;

    while (i < j) {
        do i++; while(q[i] < x);
        do j--; while(q[j] > x);

        if (i < j) {
            int tmp = q[i];
            q[i] = q[j];
            q[j] = tmp;
        }
    }

    int length = j - l + 1;
    if (length < k) {
        return q_select(q, j + 1, r,
            k - length);
    } else {
        return q_select(q, l, j, k);
    }
}

int main()
{
    int n;
    int k;
    scanf("%d %d", &n, &k);
    int *q = (int *)calloc(sizeof(int), n);
    if (q == NULL) {
        return -1;
    }
    for (int i = 0; i < n; i++) {
        scanf("%d", q + i);
    }
    int ret = q_select(q, 0, n - 1, k);
    printf("%d", ret);
    return 0;
}
\end{mycpptwocol}

\section{归并排序}
归并排序同样是基于\textbf{分治}算法,不过是以整个数组的中间位置来分。

将数组分割成两个已经分别排序好的有序数组,再将其二者合并即可。此方法需要有单独的空间来存放合并的临时结果,再将临时结果写入到原始区域中。

主要步骤如下:
\begin{enumerate}
  \item 确定分界点,\lstinline{mid = (l + r) >> 1}
  \item 递归排序左右两边
  \item 归并,将两个有序的子数组合二为一
\end{enumerate}

\subsection{AcWing 787. 归并排序}
\begin{titledbox}{AcWing 787. 归并排序}
    给定你一个长度为 $n$ 的整数数列。请你使用归并排序对这个数列按照从小到大进行排序。并将排好序的数列按顺序输出。
  
    输入格式:
  
    输入共两行,第一行包含整数 $n$。第二行包含 $n$ 个整数(所有整数均在 $1 \sim 10^9$ 范围内),表示整个数列。
  
    输出格式:
  
    输出共一行,包含 $n$ 个整数,表示排好序的数列。
  
    数据范围:$1 \le n \le 100000$
  
    输入样例:
  
    5
  
    3 1 2 4 5
  
    输出样例:
  
    1 2 3 4 5
\end{titledbox}

\begin{mycpptwocol}[merge sort]
#include <stdio.h>
#include <stdlib.h>

#define N 100010

int backup[N];

void merge_sort(int *q, int l, int r)
{
    if (r <= l) {
        return;
    }
    int mid = (l + r) >> 1;
    merge_sort(q, l, mid);
    merge_sort(q, mid + 1, r);
    int k = 0;
    int i = l;
    int j = mid + 1;
    while (i <= mid && j <= r) {
        if (q[i] <= q[j]) {
            backup[k++] = q[i++];
        }
        if (q[j] < q[i]) {
            backup[k++] = q[j++];
        }
    }
    while (i <= mid) {
        backup[k++] = q[i++];
    }
    while (j <= r) {
        backup[k++] = q[j++];
    }
    
    for (i = l, j = 0; j < k; i++, j++) {
        q[i] = backup[j];
    }
}

int main()
{
    int n;
    scanf("%d", &n);
    int *q = (int *)calloc(sizeof(int), n);
    if (q == NULL) {
        return -1;
    }
    for (int i = 0; i < n; i++) {
        scanf("%d", q + i);
    }
    merge_sort(q, 0, n - 1);
    for (int i = 0; i < n; i++) {
        printf("%d ", q[i]);
    }
    return 0;
}
\end{mycpptwocol}

双指针算法做归并

\begin{keypoint}
    这里归并两个子数组之后要写回去,\lstinline{backup}数组只是临时存储使用。
\end{keypoint}

\subsection{AcWing 788. 逆序对的数量}
\begin{titledbox}{AcWing 788. 逆序对的数量}
    给定一个长度为 $n$ 的整数数列,请你计算数列中的逆序对的数量。
    逆序对的定义如下:对于数列的第 $i$ 个和第 $j$ 个元素,如果满足 $i < j$ 且 $a[i] > a[j]$,则其为一个逆序对;否则不是。
    
    输入格式:

    第一行包含整数 $n$,表示数列的长度。第二行包含 $n$ 个整数,表示整个数列。
    
    输出格式:
    
    输出一个整数,表示逆序对的个数。

    数据范围

    $1 \le n \le 100000$,
    数列中的元素的取值范围 $[1,10^9]$。
    
    输入样例:

    2 3 4 5 6 1
    
    输出样例:
    
    5
\end{titledbox}

分治思路,将整个区间一分为二。考虑到归并排序的时候需要将两个有序数组合并,此时恰好可以做逆序对的统计。假设有一种算法,可以将数组排序的过程中统计该数组中的逆序对数量,则问题变为怎么统计两个有序数组中合起来的逆序对。

\begin{mycpptwocol}[归并排序计算逆序对数量]
#include <stdio.h>
#include <stdlib.h>

long long merge_sort(int *q, int *tmp,
                     int l, int r) {
    if (l >= r) {
        return 0;
    }
    int mid = (l + r) >> 1;
    // 左侧的数组已统计逆序对且已经排序,右侧同样
    long long res = merge_sort(q, tmp, l, mid) + merge_sort(q, tmp, mid + 1, r);
    // 统计两个有序数组合起来的逆序对数量
    int i = l;
    int j = mid + 1;
    int k = 0;
    while (i <= mid && j <= r) {
        if (q[i] > q[j]) {
            res += mid - i + 1;
            tmp[k++] = q[j++];
        } else {
            tmp[k++] = q[i++];
        }
    }
    
    while (i <= mid) {
        tmp[k++] = q[i++];
    }
    while (j <= r) {
        tmp[k++] = q[j++];
    }
    for (i = l, j = 0; i <= r;
         i++, j++) {
        q[i] = tmp[j];
    }
    
    return res;
}

int main()
{
    int n;
    scanf("%d", &n);
    int *q = (int *)calloc(n, sizeof(int));
    if (q == NULL) {
        return -1;
    }
    int *tmp = (int *)calloc(n, sizeof(int));
    if (tmp == NULL) {
        free(q);
        return -1;
    }
    for(int i = 0; i < n; i++) {
        scanf("%d", &q[i]);
    }
    
    printf("%ld", merge_sort(q, tmp, 0, n - 1));

    return 0;
}
\end{mycpptwocol}
\section{二分}
整数二分和浮点数二分,二分即查找一个边界值,在左侧满足某种性质,右侧不满足。

二分用模版如下:
\begin{mycpptwocol}[二分模版]
// 区间[l, r]被划分为[l, mid] 和
// [mid + 1, r]时使用, 往左找
int bsearch_1(int l, int r)
{
    while (l < r) {
        mid = (l + r) / 2;
        if (Check(mid)) {
            r = mid;
        } else {
            l = mid + 1;
        }
    }
}
// 区间[l, r]被划分为[l, mid - 1] 和
// [mid, r]时使用,往右找
int bsearch_2(int l, int r)
{
    while (l < r) {
        mid = (l + r + 1) / 2;
        if (Check(mid)) {
            l = mid;
        } else {
            r = mid - 1;
        }
    }
}
\end{mycpptwocol}

\begin{keypoint}
    每次要保证答案在区间中。

    第二个模版加一的原因在于,如果某次循环结束后,\lstinline{l = r - 1},如果不加1,此时因为向下取整的缘故\lstinline{mid = l},check成功后\lstinline{l}被再次赋值为\lstinline{mid}即\lstinline{l},则此时进入死循环。
\end{keypoint}

\subsection{AcWing 789. 数的范围}
\begin{titledbox}{AcWing 789. 数的范围}
给定一个按照升序排列的长度为 $n$ 的整数数组,以及 $q$ 个查询。对于每个查询,返回一个元素 $k$ 的起始位置和终止位置(位置从 $0$ 开始计数)。如果数组中不存在该元素,则返回\lstinline{-1 -1}。

输入格式:

第一行包含整数 $n$ 和 $q$,表示数组长度和询问个数。第二行包含 $n$ 个整数(均在 $1 \sim 10000$ 范围内),表示完整数组。接下来 $q$ 行,每行包含一个整数 $k$,表示一个询问元素。

输出格式:

共 $q$ 行,每行包含两个整数,表示所求元素的起始位置和终止位置。

数据范围
$1 \le n \le 100000$,
$1 \le q \le 10000$,
$1 \le k \le 10000$

输入样例:

6 3

1 2 2 3 3 4

3

4

5

输出样例:

3 4

5 5

-1 -1
\end{titledbox}


\begin{mycpptwocol}[binary search]
#include <stdio.h>
#include <stdlib.h>

int b_search_l(int *q, int l,
               int r, int t) {
    while (l < r) {
        int mid = (l + r) >> 1;
        if (q[mid] >= t) {
            r = mid;
        } else {
            l = mid + 1;
        }
    }
    if (q[l] != t) {
        return -1;
    }
    return l;
}

int b_search_r(int *q, int l,
               int r, int t) {
    while (l < r) {
        int mid = (l + r) / 2 + 1;
        if (q[mid] <= t) {
            l = mid;
        } else {
            r = mid - 1;
        }
    }
    if (q[l] != t) {
        return -1;
    }
    return l;
}

int main()
{
    int n;
    int q;
    scanf("%d %d", &n, &q);
    int *q = (int *)calloc(sizeof(int), n);
    if (q == NULL) {
        return -1;
    }
    for (int i = 0; i < n; i++) {
        scanf("%d", q + i);
    }
    while(q--) {
        int t;
        scanf("%d", &t);
        printf("%d %d\n",
        b_search_l(q, 0, n - 1, t),
        b_search_r(q, 0, n - 1, t));
    }
    return 0;
}
\end{mycpptwocol}

\begin{exclamation}
    这里不能使用\lstinline{bsearch}函数来完成左端点的搜索,因为该函数在面对重复值时返回值不确定,是未定义行为。
\end{exclamation}

\subsection{AcWing 790. 数的三次方根}
\begin{titledbox}{AcWing 790. 数的三次方根}
    给定一个浮点数 $n$,求它的三次方根。

    输入格式:

    共一行,包含一个浮点数 $n$。

    输出格式:

    共一行,包含一个浮点数,表示问题的解。
    
    注意,结果保留 $6$ 位小数。
    
    数据范围:

    $-10000 \le n \le 10000$
    
    输入样例:
    
    1000.00
    
    输出样例:

    10.000000
\end{titledbox}

\begin{mycpptwocol}[数的三次方根]
#include <stdio.h>

#define N 10000

int main()
{
    double n;
    scanf("%lf", &n);
    double l = 0 - N;
    double r = N;

    while (r - l > 1e-8) {
        double mid = (l + r) / 2;
        if (mid * mid * mid < n) {
            l = mid;
        } else {
            r = mid;
        }
    }
    printf("%.6f", l);
    return 0;
}
\end{mycpptwocol}

\section{高精度}

\section{前缀和与差分}
\begin{enumerate}
    \item 前缀和可方便地求取数组中某个区间的元素和,或者矩阵的子矩阵的和 O(1)
    \item 差分和方便的将数组某个区间的所有元素加上一个数,O(1)时间复杂度
\end{enumerate}

\subsection{AcWing 795. 前缀和}
\begin{titledbox}{AcWing 795. 前缀和}
    输入一个长度为 $n$ 的整数序列。接下来再输入 $m$ 个询问,每个询问输入一对 $l, r$。对于每个询问,输出原序列中从第 $l$ 个数到第 $r$ 个数的和。
    
    输入格式:

    第一行包含两个整数 $n$ 和 $m$。第二行包含 $n$ 个整数,表示整数数列。
    接下来 $m$ 行,每行包含两个整数 $l$ 和 $r$,表示一个询问的区间范围。
    
    输出格式:

    共 $m$ 行,每行输出一个询问的结果。

    数据范围:

    $1 \le l \le r \le n$, $1 \le n,m \le 100000$, $-1000 \le \text{数列中元素的值} \le 1000$
    
    \begin{minipage}[t]{.5\textwidth}
        % \raggedright
        输入样例:\\
        5 3 \\
        2 1 3 6 4 \\
        1 2 \\
        1 3 \\
        2 4
        \end{minipage}% <---------------- Note the use of "%"
        \begin{minipage}[t]{.5\textwidth}
        % \raggedcenter
        输出样例:\\
        3 \\
        6 \\
        10
    \end{minipage}
\end{titledbox}

\begin{mycpponecol}[前缀和]
#include <stdio.h>
#include <stdlib.h>

int main()
{
    int n;
    int m;
    scanf("%d %d", &n, &m);
    int *q = (int *)calloc(n + 1, sizeof(int));
    int *preSum = (int *)calloc(n + 1, sizeof(int));
    for (int i = 1; i <= n; i++) {
        scanf("%d", q + i);
        preSum[i] = preSum[i - 1] + q[i];
    }
    while (m--) {
        int l;
        int r;
        scanf("%d %d", &l, &r);
        printf("%d\n", preSum[r] - preSum[l - 1]);
    }
    return 0;
}
\end{mycpponecol}

\subsection{AcWing 796. 子矩阵的和}
子矩阵
\begin{problembox}{AcWing 796. 子矩阵的和}
    \small{子矩阵的和}
    输入一个 $n$ 行 $m$ 列的整数矩阵,再输入 $q$ 个询问,每个询问包含四个整数 $x_1$, $y_1$, $x_2$, $y_2$,表示一个子矩阵的左上角坐标和右下角坐标。对于每个询问输出子矩阵中所有数的和。

    输入格式:

    第一行包含三个整数 $n$,$m$,$q$。接下来 $n$ 行,每行包含 $m$ 个整数,表示整数矩阵。接下来 $q$ 行,每行包含四个整数 $x_1, y_1, x_2, y_2$,表示一组询问。
    
    输出格式:

    共 $q$ 行,每行输出一个询问的结果。
    
    数据范围:

    $1 \le n,m \le 1000$, 
    $1 \le q \le 200000$, 
    $1 \le x_1 \le x_2 \le n$, 
    $1 \le y_1 \le y_2 \le m$, 
    $-1000 \le 矩阵内元素的值 \le 1000$

    \begin{minipage}[t]{.5\textwidth}
        % \raggedright
        输入样例:\\
        3 4 3 \\
        1 7 2 4 \\
        3 6 2 8 \\
        2 1 2 3 \\
        1 1 2 2 \\
        2 1 3 4\\
        1 3 3 4
        \end{minipage}% <---------------- Note the use of "%"
        \begin{minipage}[t]{.5\textwidth}
        % \raggedcenter
        输出样例:\\
        17 \\
        27 \\
        21
    \end{minipage}
\end{problembox}

\begin{mycpponecol}[子矩阵的和]
#include<stdio.h>

#define N 1010

int mat[N][N];
int preMat[N][N];

int main()
{
    int n, m, q;
    scanf("%d %d %d", &n, &m, &q);
    for (int i = 1; i <= n; i++) {
        for (int j = 1; j <= m; j++) {
            scanf("%d", &mat[i][j]);
            preMat[i][j] = preMat[i - 1][j] + preMat[i][j - 1] - preMat[i - 1][j - 1] + mat[i][j];
        }
    }
    while (q--) {
        int x1, y1, x2, y2;
        scanf("%d %d %d %d", &x1, &y1, &x2, &y2);
        printf("%d\n", preMat[x2][y2] - preMat[x2][y1 - 1] - preMat[x1 - 1][y2] + preMat[x1 - 1][y1 - 1]);
    }
    return 0;
}
\end{mycpponecol}

\section{双指针算法}


\section{位运算}


\section{离散化}


\section{区间合并}