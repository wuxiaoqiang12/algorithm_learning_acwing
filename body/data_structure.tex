\chapter{数据结构}

\section{单链表}
\subsection{AcWing 826. 单链表}

\begin{titledbox}{AcWing 826. 单链表}
实现一个单链表,链表初始为空,支持三种操作:
\begin{enumerate}
    \itemsep=-5pt
    \item 向链表头插入一个数;
    \item 删除第 $k$ 个插入的数后面的数;
    \item 在第 $k$ 个插入的数后插入一个数。
\end{enumerate}

现在要对该链表进行 $M$ 次操作,进行完所有操作后,从头到尾输出整个链表。
\textbf{注意}: 题目中第 $k$ 个插入的数并不是指当前链表的第 $k$ 个数。例如操作过程中一共插入了 $n$ 个数,则按照插入的时间顺序,这 $n$ 个数依次为:第 $1$ 个插入的数,第 $2$ 个插入的数,$\dots$第 $n$ 个插入的数。

输入格式:

第一行包含整数 $M$,表示操作次数。接下来 $M$ 行,每行包含一个操作命令,操作命令可能为以下几种:

\begin{enumerate}
    \itemsep=-5pt
    \item H x,表示向链表头插入一个数 $x$。
    \item D k,表示删除第 $k$ 个插入的数后面的数(当 $k$ 为 $0$ 时,表示删除头结点)。
    \item I k x,表示在第 $k$ 个插入的数后面插入一个数 $x$(此操作中 $k$ 均大于 $0$)。
\end{enumerate}

输出格式:

共一行,将整个链表从头到尾输出。

数据范围:

$1 \le M \le 100000$ 所有操作保证合法。

\begin{inputblock}
    10 \\
    H 9 \\
    I 1 1 \\
    D 1 \\
    D 0 \\
    H 6 \\
    I 3 6 \\
    I 4 5 \\
    I 4 5 \\
    I 3 4 \\
    D 6
\end{inputblock}
\begin{outputblock}
    6 4 6 5
\end{outputblock}
\end{titledbox}

\begin{mycpptwocol}[linked list]
#include <stdio.h>
#include <stdlib.h>
#include <stdbool.h>

#define N 100010
int idx; // 标识节点
int head; // 头节点指向的元素
int e[N]; // element array
int ne[N]; // next array

void Init()
{
    idx = -1;
    head = -1;
}
    
// 在头节点后插入元素
void AddHead(int x) {
    e[++idx] = x;
    ne[idx] = head;
    head = idx;
}

// 在k节点之后插入元素
void Add(int k, int x)
{
    e[++idx] = x;
    ne[idx] = ne[k];
    ne[k] = idx;
}

// 删除k之后的那个元素
void Remove(int k) {
    ne[k] = ne[ne[k]];
}

int main()
{
    Init();
    int n;
    scanf("%d", &n);
    while (n--) {
        int x;
        int k;
        char op;
        scanf(" %c", &op);
        if (op == 'H') {
            scanf("%d", &x);
            AddHead(x);
        }
        if (op == 'I') {
            scanf("%d %d", &k, &x);
            Add(k - 1, x);
        }
        if (op == 'D') {
            scanf("%d", &k);
            if (k == 0) {
                head = ne[head];
            }
            Remove(k - 1);
        }
    }

    int tmp = head;
    while (tmp != -1) {
        printf("%d ", e[tmp]);
        tmp = ne[tmp];
    }
    return 0;
}
\end{mycpptwocol}

\section{双链表}

\subsection{AcWing 827. 双链表}
\begin{titledbox}{AcWing 827. 双链表}
实现一个双链表,双链表初始为空,支持 $5$ 种操作:

\begin{enumerate}
    \itemsep=-5pt
    \item 在最左侧插入一个数;
    \item 在最右侧插入一个数;
    \item 将第 $k$ 个插入的数删除;
    \item 在第 $k$ 个插入的数左侧插入一个数;
    \item 在第 $k$ 个插入的数右侧插入一个数。
\end{enumerate}

现在要对该链表进行 $M$ 次操作,进行完所有操作后,从左到右输出整个链表。\textbf{注意}: 题目中第 $k$ 个插入的数并不是指当前链表的第 $k$ 个数。例如操作过程中一共插入了 $n$ 个数,则按照插入的时间顺序,这 $n$ 个数依次为:第 $1$ 个插入的数,第 $2$ 个插入的数,…第 $n$ 个插入的数。

输入格式:

第一行包含整数 $M$,表示操作次数。接下来 $M$ 行,每行包含一个操作命令,操作命令可能为以下几种:

\begin{enumerate}
    \itemsep=-5pt
    \item L x,表示在链表的最左端插入数 $x$。
    \item R x,表示在链表的最右端插入数 $x$。
    \item D k,表示将第 $k$ 个插入的数删除。
    \item IL k x,表示在第 $k$ 个插入的数左侧插入一个数。
    \item IR k x,表示在第 $k$ 个插入的数右侧插入一个数。
\end{enumerate}

输出格式:

共一行,将整个链表从左到右输出

数据范围:

$1 \le M \le 100000$ 所有操作保证合法。

\begin{inputblock}
    10 \\
    R 7 \\
    D 1 \\
    L 3 \\
    IL 2 10 \\
    D 3 \\
    IL 2 7 \\
    L 8 \\
    R 9 \\
    IL 4 7 \\
    IR 2 2
\end{inputblock}
\begin{outputblock}
    8 7 7 3 2 9
\end{outputblock}
\end{titledbox}

\begin{mycpptwocol}[双链表]
#include <stdio.h>
#include <stdlib.h>

#define N 100010

int e[N];
int l[N];
int r[N];
int idx = 1;

void Init()
{
    r[0] = 1;
    l[1] = 0;
    idx = 1;
}

void AddRight(int k, int x)
{
    e[++idx] = x;
    r[idx] = r[k];
    l[r[k]] = idx;
    l[idx] = k;
    r[k] = idx;
}

void Remove(int k)
{
    r[l[k]] = r[k];
    l[r[k]] = l[k];
}

int main()
{
    Init();
    int n;
    scanf("%d", &n);
    while (n--) {
        char op[3];
        scanf("%s", op);
        int k;
        int x;
        if (op[0] == 'R') {
            scanf("%d", &x);
            AddRight(l[1], x);
        }
        if (op[0] == 'D') {
            scanf("%d", &k);
            Remove(k + 1);
        }
        if (op[0] == 'L') {
            scanf("%d", &x);
            AddRight(0, x);
        }
        if (op[0] == 'I') {
            scanf("%d %d", &k, &x);
            if (op[1] == 'L') {
                AddRight(l[k + 1], x);
            } else {
                AddRight(k + 1, x);
            }
        }
    }
    for (int i = r[0]; i != 1; i = r[i]) {
        printf("%d ", e[i]);
    }
    return 0;
}
\end{mycpptwocol}

\section{栈}
\subsection{AcWing 828. 模拟栈}
\begin{titledbox}{AcWing 828. 模拟栈}
实现一个栈,栈初始为空,支持四种操作:

\begin{enumerate}
    \itemsep=-5pt
    \item push x - 向栈顶插入一个数 $x$;
    \item pop - 从栈顶弹出一个数;
    \item empty - 判断栈是否为空;
    \item query - 查询栈顶元素
\end{enumerate}

现在要对栈进行 $M$ 个操作,其中的每个操作 $3$ 和操作 $4$ 都要输出相应的结果。

输入格式:

第一行包含整数 $M$,表示操作次数。接下来 $M$ 行,每行包含一个操作命令,操作命令为 <code>push x</code>,<code>pop</code>,<code>empty</code>,<code>query</code> 中的一种。

输出格式:

对于每个 <code>empty</code> 和 <code>query</code> 操作都要输出一个查询结果,每个结果占一行。其中,<code>empty</code> 操作的查询结果为 <code>YES</code> 或 <code>NO</code>,<code>query</code> 操作的查询结果为一个整数,表示栈顶元素的值。

数据范围:

$1 \le M \le 100000$, $1 \le x \le 10^9$ 所有操作保证合法。

\begin{inputblock}
    10 \\
    push 5  \\
    query  \\
    push 6  \\
    pop  \\
    query  \\
    pop  \\
    empty  \\
    push 4  \\
    query  \\
    empty 
\end{inputblock}
\begin{outputblock}
    5 \\
    5 \\
    YES \\
    4 \\
    NO
\end{outputblock}
\end{titledbox}

\begin{mycpptwocol}[模拟栈]
#include <stdio.h>
#include <stdlib.h>

int main()
{
    int n;
    scanf("%d", &n);
    int *stack = (int *)calloc(n, sizeof(int));
    int top = 0;
    while (n--) {
        char op[10];
        scanf("%s", op);
        int x;
        if (strcmp(op, "push") == 0) {
            scanf("%d", &x);
            stack[top++] = x;
        }
        if (strcmp(op, "query") == 0) {
            printf("%d\n", stack[top - 1]);
        }
        if (strcmp(op, "pop") == 0) {
            top--;
        }
        if (strcmp(op, "empty") == 0) {
            printf("%s\n", top == 0 ? "YES" : "NO");
        }
    }
    free(stack);
    return 0;
}
\end{mycpptwocol}
\subsection{AcWing 3302. 表达式求值}

\section{队列}
\subsection{AcWing 829. 模拟队列}

\section{单调栈}

\subsection{AcWing 830. 单调栈}

\section{单调队列}
\subsection{AcWing 154. 滑动窗口}

\section{KMP}
\subsection{AcWing 831. KMP字符串}

\section{Trie}
\subsection{AcWing 835. Trie字符串统计}
\subsection{AcWing 143. 最大异或对}

\section{并查集}
\subsection{AcWing 836. 合并集合}
\subsection{AcWing 837. 连通块中点的数量}
\subsection{AcWing 240. 食物链}

\section{堆}
\subsection{AcWing 838. 堆排序}
\subsection{AcWing 839. 模拟堆}

\section{哈希表}
\subsection{AcWing 840. 模拟散列表}
\subsection{AcWing 841. 字符串哈希}