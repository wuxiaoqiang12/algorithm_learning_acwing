\chapter{数据结构}

\section{单链表}
\subsection{AcWing 826. 单链表}

\begin{titledbox}{AcWing 826. 单链表}
实现一个单链表,链表初始为空,支持三种操作:
\begin{enumerate}
    \itemsep=-5pt
    \item 向链表头插入一个数;
    \item 删除第 $k$ 个插入的数后面的数;
    \item 在第 $k$ 个插入的数后插入一个数。
\end{enumerate}

现在要对该链表进行 $M$ 次操作,进行完所有操作后,从头到尾输出整个链表。
\textbf{注意}: 题目中第 $k$ 个插入的数并不是指当前链表的第 $k$ 个数。例如操作过程中一共插入了 $n$ 个数,则按照插入的时间顺序,这 $n$ 个数依次为:第 $1$ 个插入的数,第 $2$ 个插入的数,$\dots$第 $n$ 个插入的数。

输入格式:

第一行包含整数 $M$,表示操作次数。接下来 $M$ 行,每行包含一个操作命令,操作命令可能为以下几种:

\begin{enumerate}
    \itemsep=-5pt
    \item H x,表示向链表头插入一个数 $x$。
    \item D k,表示删除第 $k$ 个插入的数后面的数(当 $k$ 为 $0$ 时,表示删除头结点)。
    \item I k x,表示在第 $k$ 个插入的数后面插入一个数 $x$(此操作中 $k$ 均大于 $0$)。
\end{enumerate}

输出格式:

共一行,将整个链表从头到尾输出。

数据范围:

$1 \le M \le 100000$ 所有操作保证合法。

\begin{inputblock}
    10 \\
    H 9 \\
    I 1 1 \\
    D 1 \\
    D 0 \\
    H 6 \\
    I 3 6 \\
    I 4 5 \\
    I 4 5 \\
    I 3 4 \\
    D 6
\end{inputblock}
\begin{outputblock}
    6 4 6 5
\end{outputblock}
\end{titledbox}

\begin{mycpptwocol}[linked list]
#include <stdio.h>
#include <stdlib.h>
#include <stdbool.h>

#define N 100010
int idx; // 标识节点
int head; // 头节点指向的元素
int e[N]; // element array
int ne[N]; // next array

void Init()
{
    idx = -1;
    head = -1;
}
    
// 在头节点后插入元素
void AddHead(int x) {
    e[++idx] = x;
    ne[idx] = head;
    head = idx;
}

// 在k节点之后插入元素
void Add(int k, int x)
{
    e[++idx] = x;
    ne[idx] = ne[k];
    ne[k] = idx;
}

// 删除k之后的那个元素
void Remove(int k) {
    ne[k] = ne[ne[k]];
}

int main()
{
    Init();
    int n;
    scanf("%d", &n);
    while (n--) {
        int x;
        int k;
        char op;
        scanf(" %c", &op);
        if (op == 'H') {
            scanf("%d", &x);
            AddHead(x);
        }
        if (op == 'I') {
            scanf("%d %d", &k, &x);
            Add(k - 1, x);
        }
        if (op == 'D') {
            scanf("%d", &k);
            if (k == 0) {
                head = ne[head];
            }
            Remove(k - 1);
        }
    }

    int tmp = head;
    while (tmp != -1) {
        printf("%d ", e[tmp]);
        tmp = ne[tmp];
    }
    return 0;
}
\end{mycpptwocol}

\section{双链表}

\subsection{AcWing 827. 双链表}
\begin{titledbox}{AcWing 827. 双链表}
实现一个双链表,双链表初始为空,支持 $5$ 种操作:

\begin{enumerate}
    \itemsep=-5pt
    \item 在最左侧插入一个数;
    \item 在最右侧插入一个数;
    \item 将第 $k$ 个插入的数删除;
    \item 在第 $k$ 个插入的数左侧插入一个数;
    \item 在第 $k$ 个插入的数右侧插入一个数。
\end{enumerate}

现在要对该链表进行 $M$ 次操作,进行完所有操作后,从左到右输出整个链表。\textbf{注意}: 题目中第 $k$ 个插入的数并不是指当前链表的第 $k$ 个数。例如操作过程中一共插入了 $n$ 个数,则按照插入的时间顺序,这 $n$ 个数依次为:第 $1$ 个插入的数,第 $2$ 个插入的数,…第 $n$ 个插入的数。

输入格式:

第一行包含整数 $M$,表示操作次数。接下来 $M$ 行,每行包含一个操作命令,操作命令可能为以下几种:

\begin{enumerate}
    \itemsep=-5pt
    \item L x,表示在链表的最左端插入数 $x$。
    \item R x,表示在链表的最右端插入数 $x$。
    \item D k,表示将第 $k$ 个插入的数删除。
    \item IL k x,表示在第 $k$ 个插入的数左侧插入一个数。
    \item IR k x,表示在第 $k$ 个插入的数右侧插入一个数。
\end{enumerate}

输出格式:

共一行,将整个链表从左到右输出

数据范围:

$1 \le M \le 100000$ 所有操作保证合法。

\begin{inputblock}
    10 \\
    R 7 \\
    D 1 \\
    L 3 \\
    IL 2 10 \\
    D 3 \\
    IL 2 7 \\
    L 8 \\
    R 9 \\
    IL 4 7 \\
    IR 2 2
\end{inputblock}
\begin{outputblock}
    8 7 7 3 2 9
\end{outputblock}
\end{titledbox}

\begin{mycpptwocol}[双链表]
#include <stdio.h>
#include <stdlib.h>

#define N 100010

int e[N];
int l[N];
int r[N];
int idx = 1;

void Init()
{
    r[0] = 1;
    l[1] = 0;
    idx = 1;
}

void AddRight(int k, int x)
{
    e[++idx] = x;
    r[idx] = r[k];
    l[r[k]] = idx;
    l[idx] = k;
    r[k] = idx;
}

void Remove(int k)
{
    r[l[k]] = r[k];
    l[r[k]] = l[k];
}

int main()
{
    Init();
    int n;
    scanf("%d", &n);
    while (n--) {
        char op[3];
        scanf("%s", op);
        int k;
        int x;
        if (op[0] == 'R') {
            scanf("%d", &x);
            AddRight(l[1], x);
        }
        if (op[0] == 'D') {
            scanf("%d", &k);
            Remove(k + 1);
        }
        if (op[0] == 'L') {
            scanf("%d", &x);
            AddRight(0, x);
        }
        if (op[0] == 'I') {
            scanf("%d %d", &k, &x);
            if (op[1] == 'L') {
                AddRight(l[k + 1], x);
            } else {
                AddRight(k + 1, x);
            }
        }
    }
    for (int i = r[0]; i != 1; i = r[i]) {
        printf("%d ", e[i]);
    }
    return 0;
}
\end{mycpptwocol}

\section{栈}
\subsection{AcWing 828. 模拟栈}
\begin{titledbox}{AcWing 828. 模拟栈}
实现一个栈,栈初始为空,支持四种操作:

\begin{enumerate}
    \itemsep=-5pt
    \item push x - 向栈顶插入一个数 $x$;
    \item pop - 从栈顶弹出一个数;
    \item empty - 判断栈是否为空;
    \item query - 查询栈顶元素
\end{enumerate}

现在要对栈进行 $M$ 个操作,其中的每个操作 $3$ 和操作 $4$ 都要输出相应的结果。

输入格式:

第一行包含整数 $M$,表示操作次数。接下来 $M$ 行,每行包含一个操作命令,操作命令为 push x,pop,empty,query 中的一种。

输出格式:

对于每个 empty 和 query 操作都要输出一个查询结果,每个结果占一行。其中,empty 操作的查询结果为 YES 或 NO,query 操作的查询结果为一个整数,表示栈顶元素的值。

数据范围:

$1 \le M \le 100000$, $1 \le x \le 10^9$ 所有操作保证合法。

\begin{inputblock}
    10 \\
    push 5  \\
    query  \\
    push 6  \\
    pop  \\
    query  \\
    pop  \\
    empty  \\
    push 4  \\
    query  \\
    empty 
\end{inputblock}
\begin{outputblock}
    5 \\
    5 \\
    YES \\
    4 \\
    NO
\end{outputblock}
\end{titledbox}

\begin{mycpptwocol}[模拟栈]
#include <stdio.h>
#include <stdlib.h>

int main()
{
    int n;
    scanf("%d", &n);
    int *stack = (int *)calloc(n, sizeof(int));
    int top = 0;
    while (n--) {
        char op[10];
        scanf("%s", op);
        int x;
        if (strcmp(op, "push") == 0) {
            scanf("%d", &x);
            stack[top++] = x;
        }
        if (strcmp(op, "query") == 0) {
            printf("%d\n", stack[top - 1]);
        }
        if (strcmp(op, "pop") == 0) {
            top--;
        }
        if (strcmp(op, "empty") == 0) {
            printf("%s\n", top == 0 ? "YES" : "NO");
        }
    }
    free(stack);
    return 0;
}
\end{mycpptwocol}
\subsection{AcWing 3302. 表达式求值}

\section{队列}
\subsection{AcWing 829. 模拟队列}

\begin{titledbox}{AcWing 829. 模拟队列}
实现一个队列,队列初始为空,支持四种操作:

\begin{enumerate}
    \itemsep=-5pt
    \item push x - 向队尾插入一个数 $x$;
    \item pop - 从队头弹出一个数;
    \item empty - 判断队列是否为空;
    \item query - 查询队头元素
\end{enumerate}

现在要对队列进行 $M$ 个操作,其中的每个操作 $3$ 和操作 $4$ 都要输出相应的结果。

输入格式:

第一行包含整数 $M$,表示操作次数。接下来 $M$ 行,每行包含一个操作命令,操作命令为 push x,pop,empty,query 中的一种。

输出格式:

对于每个 empty 和 query 操作都要输出一个查询结果,每个结果占一行。其中,empty 操作的查询结果为 YES 或 NO,query 操作的查询结果为一个整数,表示栈顶元素的值。

数据范围:

$1 \le M \le 100000$, $1 \le x \le 10^9$ 所有操作保证合法。

\begin{inputblock}
    10 \\
    push 6 \\
    empty \\
    query \\
    pop \\
    empty \\
    push 3 \\
    push 4 \\
    pop \\
    query \\
    push 6
\end{inputblock}
\begin{outputblock}
    NO \\
    YES \\
    6 \\
    4
\end{outputblock}
\end{titledbox}

\begin{mycpptwocol}[模拟队列]
#include <stdio.h>
#include <stdbool.h>

#define N 100010
int queue[N];
int tail;
int head;

void push(int x)
{
    queue[++tail] = x;
}

bool isEmpty()
{
    return head == tail;
}

void pop()
{
    head++;
}

int query()
{
    return queue[head + 1];
}

int main()
{
    int n;
    scanf("%d", &n);
    char op[10];
    while (n--) {
        scanf("%s", op);
        if (strcmp(op, "push") == 0) {
            int x;
            scanf("%d", &x);
            push(x);
        }
        if (strcmp(op, "pop") == 0) {
            pop();
        }
        if (strcmp(op, "empty") == 0) {
            printf("%s\n", isEmpty() ? "YES" : "NO");
        }
        if (strcmp(op, "query") == 0) {
            printf("%d\n", query());
        }
    }
    return 0;
}

\end{mycpptwocol}



\section{单调栈}

\subsection{AcWing 830. 单调栈}

\begin{titledbox}{AcWing 830. 单调栈}
给定一个长度为 $N$ 的整数数列,输出每个数左边第一个比它小的数,如果不存在则输出 $-1$。

输入格式:

第一行包含整数 $N$,表示数列长度。第二行包含 $N$ 个整数,表示整数数列。

输出格式:

共一行,包含 $N$ 个整数,其中第 $i$ 个数表示第 $i$ 个数的左边第一个比它小的数,如果不存在则输出 $-1$。

数据范围:

$1 \le N \le 10^5$,$1 \le \text{数列中元素} \le 10^9$

\begin{inputblock}
    5 \\
    3 4 2 7 5
\end{inputblock}
\begin{outputblock}
    -1 3 -1 2 2
\end{outputblock}

\end{titledbox}

\begin{mycpptwocol}[单调栈]
#include <stdio.h>
#include <stdlib.h>

int main()
{
    int n;
    scanf("%d", &n);
    int *stack = (int *)calloc(sizeof(int), n);
    int top = 0;
    while (n--) {
        int x;
        scanf("%d", &x);
        while (top != 0 && stack[top] >= x) {
            top--;
        }
        if (top == 0) {
            printf("%d ", -1);
        } else {
            printf("%d ", stack[top]);
        }
        stack[++top] = x;
    }
    free(stack);
    return 0;
}
\end{mycpptwocol}

\section{单调队列}
\subsection{AcWing 154. 滑动窗口}

\begin{titledbox}{AcWing 154. 滑动窗口}
给定一个大小为 $n \le 10^6$ 的数组。有一个大小为 $k$ 的滑动窗口,它从数组的最左边移动到最右边。你只能在窗口中看到 $k$ 个数字。每次滑动窗口向右移动一个位置。

以下是一个例子:

该数组为 \lstinline{[1 3 -1 -3 5 3 6 7]},$k$ 为 $3$。

\begin{tabular}{|c|c|c|}
    \hline
        窗口位置 & 最小值 & 最大值 \\ \hline
        [1  3  -1] -3  5  3  6  7 & -1 & 3 \\ \hline
        1 [3  -1  -3] 5  3  6  7 & -3 & 3 \\ \hline
        1  3 [-1  -3  5] 3  6  7 & -3 & 5 \\ \hline
        1  3  -1 [-3  5  3] 6  7 & -3 & 5 \\ \hline
        1  3  -1  -3 [5  3  6] 7 & 3 & 6 \\ \hline
        1  3  -1  -3  5 [3  6  7] & 3 & 7 \\ \hline
\end{tabular}

你的任务是确定滑动窗口位于每个位置时,窗口中的最大值和最小值。

输入格式:
输入包含两行。 第一行包含两个整数 $n$ 和 $k$,分别代表数组长度和滑动窗口的长度。第二行有 $n$ 个整数,代表数组的具体数值。同行数据之间用空格隔开。

输出格式:
输出包含两个。第一行输出,从左至右,每个位置滑动窗口中的最小值。第二行输出,从左至右,每个位置滑动窗口中的最大值。
    
\begin{inputblock}
    8 3 \\
    1 3 -1 -3 5 3 6 7
\end{inputblock}
\begin{outputblock}
    -1 -3 -3 -3 3 3 \\
    3 3 5 5 6 7
\end{outputblock}
    
\end{titledbox}

\begin{mycpptwocol}[滑动窗口:单调队列]
#include <stdio.h>
#include <stdlib.h>

int main()
{
    int n;
    int k;
    scanf("%d %d", &n, &k);
    int *a = (int *)calloc(sizeof(int), n);
    for (int i = 0; i < n; i++) {
        scanf("%d", a + i);
    }
    int *queue = (int *)calloc(sizeof(int), n);
    int head = 0;
    int tail = -1;
    for (int i = 0; i < n; i++) {
        if (head <= tail && i - k + 1 > queue[head]) {
            head++;
        }
        while (head <= tail && a[queue[tail]] >= a[i]) {
            tail--;
        }
        queue[++tail] = i;
        if (i >= k - 1) {
            printf("%d ", a[queue[head]]);
        }
    }
    puts("");
    head = 0;
    tail = -1;
    for (int i = 0; i < n; i++) {
        if (head <= tail && i - k + 1 > queue[head]) {
            head++;
        }
        while (head <= tail && a[queue[tail]] <= a[i]) {
            tail--;
        }
        queue[++tail] = i;
        if (i >= k - 1) {
            printf("%d ", a[queue[head]]);
        }
    }
    free(queue);
    free(a);
    return 0;
}
\end{mycpptwocol}
\section{KMP}
\subsection{AcWing 831. KMP字符串}

\section{Trie}
高效地存储和查找字符串集合的数据结构,可以用传统结构体实现,亦可用数组来模拟指针。


\subsection{AcWing 835. Trie字符串统计}

\begin{titledbox}{AcWing 835. Trie字符串统计}
维护一个字符串集合,支持两种操作:

\lstinline{I x} 向集合中插入一个字符串 $x$;

\lstinline{Q x} 询问一个字符串在集合中出现了多少次。

共有 $N$ 个操作,输入的字符串总长度不超过 $10^5$,字符串仅包含小写英文字母。

输入格式:

第一行包含整数 $N$,表示操作数。接下来 $N$ 行,每行包含一个操作指令,指令为 \lstinline{I x} 或 \lstinline{Q x} 中的一种。

输出格式:

对于每个询问指令 \lstinline{Q x},都要输出一个整数作为结果,表示 $x$ 在集合中出现的次数。
每个结果占一行。

数据范围:

$1 \le N \le 2*10^4$

\begin{inputblock}
    5 \\
    I abc \\
    Q abc \\
    Q ab \\
    I ab \\
    Q ab
\end{inputblock}
\begin{outputblock}
    1 \\
    0 \\
    1
\end{outputblock}
\end{titledbox}

这里有两种写法:链表形式和数组模拟的方式

\begin{mycpptwocol}[链表形式的Trie]
#include <stdio.h>
#include <stdlib.h>
#include <string.h>
#include <stdbool.h>

#define ALPH_NUM 26
#define N 100

typedef struct Node_ {
    struct Node_ *nodes[ALPH_NUM];
    int cnt;
    bool isEnd;
} Node;

void Insert(Node *head, char *str)
{
    int len = strlen(str);
    Node *tmp = head;
    for (int i = 0; i < len; i++) {
        int idx = str[i] - 'a';
        if (tmp->nodes[idx] == NULL) {
            tmp->nodes[idx] = (Node *)calloc(sizeof(Node), 1);
        }
        tmp = tmp->nodes[idx];
    }
    tmp->cnt += 1;
    tmp->isEnd = true;
}

int Query(Node *head, char *str)
{
    int len = strlen(str);
    Node *tmp = head;
    for (int i = 0; i < len; i++) {
        int idx = str[i] - 'a';
        if (tmp->nodes[idx] == NULL) {
            return 0;
        }
        tmp = tmp->nodes[idx];
    }
    return tmp->isEnd ? tmp->cnt : 0;
}

void Free(Node *head) {
    Node *tmp = head;
    for (int i = 0; i < ALPH_NUM; i++) {
        if (tmp->nodes[i] == NULL) {
            continue;
        }
        Free(tmp->nodes[i]);
    }
    free(tmp);
}

int main()
{
    int n;
    scanf("%d", &n);
    char op[2];
    char str[N];
    Node *head = (Node *)calloc(sizeof(Node), 1);
    while (n--) {
        scanf("%s %s", op, str);
        if (op[0] == 'I') {
            Insert(head, str);
        }
        if (op[0] == 'Q') {
            printf("%d\n", Query(head, str));
        }
    }
    Free(head);
    return 0;
}
\end{mycpptwocol}

\begin{mycpptwocol}[数组模拟Trie]
#include <stdio.h>
#include <stdlib.h>
#include <string.h>

// define an macro for alphabets number
#define ALPHABETS 26
#define N 20010
#define MAX_BUF 100

int trie[N][ALPHABETS];
int idx;
int cnt[N];

// insert str to trie
void Insert(char *str) {
    int i;
    int len = strlen(str);
    int cur = 0;
    for (i = 0; i < len; i++) {
        int ch = str[i] - 'a';
        if (trie[cur][ch] == 0) {
            trie[cur][ch] = ++idx;
        }
        cur = trie[cur][ch];
    }
    cnt[cur]++;
}

// query str from trie
int Query(char *str) {
    int i;
    int len = strlen(str);
    int cur = 0;
    for (i = 0; i < len; i++) {
        int ch = str[i] - 'a';
        if (trie[cur][ch] == 0) {
            return 0;
        }
        cur = trie[cur][ch];
    }
    return cnt[cur];
}

int main()
{
    int n;
    scanf("%d", &n);
    char op[2];
    char str[MAX_BUF];
    while (n--) {
        scanf("%s %s", op, str);
        if (op[0] == 'I') {
            Insert(str);
        }
        if (op[0] == 'Q') {
            printf("%d\n", Query(str));
        }
    }
    
    return 0;
}
\end{mycpptwocol}

\subsection{AcWing 143. 最大异或对}

\section{并查集}

两个操作:

\begin{enumerate}
    \item 将两个集合合并
    \item 询问两个元素是否在一个集合中
\end{enumerate}

基本原理:每个集合用一棵树来表示。树根的编号就是整个集合的编号,每个节点存储它的父节点,\lstinline{p[x]}标识\lstinline{x}的父节点

问题:

\begin{enumerate}
    \item 如何判断树根:\lstinline{if(p[x] == x)}
    \item 如何求x的集合编号:\lstinline{while(p[x] != x) x = p[x]}
    \item 如何合并两个集合:\lstinline{p[p[x]] = p[y]}
\end{enumerate}

\subsection{AcWing 836. 合并集合}

\begin{titledbox}{AcWing 836. 合并集合}
一共有 $n$ 个数,编号是 $1 \sim n$,最开始每个数各自在一个集合中。现在要进行 $m$ 个操作,操作共有两种:

M a b,将编号为 $a$ 和 $b$ 的两个数所在的集合合并,如果两个数已经在同一个集合中,则忽略这个操作;

Q a b,询问编号为 $a$ 和 $b$ 的两个数是否在同一个集合中;

输入格式:

第一行输入整数 $n$ 和 $m$。接下来 $m$ 行,每行包含一个操作指令,指令为 \lstinline{M a b} 或 \lstinline{Q a b} 中的一种。

输出格式:

对于每个询问指令 \lstinline{Q a b},都要输出一个结果,如果 $a$ 和 $b$ 在同一集合内,则输出 Yes,否则输出 No。每个结果占一行。

数据范围:

$1 \le n,m \le 10^5$
    
\begin{inputblock}
    4 5 \\
    M 1 2 \\
    M 3 4 \\
    Q 1 2 \\
    Q 1 3 \\
    Q 3 4
\end{inputblock}
\begin{outputblock}
    Yes \\
    No \\
    Yes
\end{outputblock}
\end{titledbox}

\begin{mycpptwocol}[合并集合]
#include <stdlib.h>
#include <stdio.h>

int Find(int *parent, int a)
{
    if (parent[a] != a) {
        parent[a] = Find(parent, parent[a]);
    }
    return parent[a];
}

void Union(int *parent, int a, int b)
{
    int aP = Find(parent, a);
    int bP = Find(parent, b);
    parent[aP] = bP;
}

int main()
{
    int m, n;
    scanf("%d %d", &m, &n);
    int *parent = (int *) calloc(n + 1, sizeof(int));
    if (parent == NULL) {
        return -1;
    }
    for (int i = 0; i < n + 1; i++) {
        parent[i] = i;
    }
    while (n--) {
        char op[2];
        int a, b;
        scanf("%s %d %d", op, &a, &b);
        if (op[0] == 'M') {
            Union(parent, a, b);
        }
        if (op[0] == 'Q') {
            printf("%s\n", Find(parent, a) == Find(parent, b) ? "Yes" : "No");
        }
    }
    free(parent);
    return 0;
}
\end{mycpptwocol}
\subsection{AcWing 837. 连通块中点的数量}

\begin{titledbox}{AcWing 837. 连通块中点的数量}
给定一个包含 $n$ 个点(编号为 $1 \sim n$)的无向图,初始时图中没有边。现在要进行 $m$ 个操作,操作共有三种:

\lstinline{C a b},在点 $a$ 和点 $b$ 之间连一条边,$a$ 和 $b$ 可能相等;

\lstinline{Q1 a b},询问点 $a$ 和点 $b$ 是否在同一个连通块中,$a$ 和 $b$ 可能相等;

\lstinline{Q2 a},询问点 $a$ 所在连通块中点的数量;

输入格式:

第一行输入整数 $n$ 和 $m$。接下来 $m$ 行,每行包含一个操作指令,指令为 \lstinline{C a b},\lstinline{Q1 a b} 或 \lstinline{Q2 a} 中的一种。

输出格式:

对于每个询问指令 \lstinline{Q1 a b},如果 $a$ 和 $b$ 在同一个连通块中,则输出 \lstinline{Yes},否则输出 \lstinline{No}。对于每个询问指令 \lstinline{Q2 a},输出一个整数表示点 $a$ 所在连通块中点的数量,每个结果占一行。

数据范围:

$1 \le n,m \le 10^5$
    
\begin{inputblock}
    5 5 \\
    C 1 2 \\
    Q1 1 2 \\
    Q2 1 \\
    C 2 5 \\
    Q2 5
\end{inputblock}
\begin{outputblock}
    Yes \\
    2 \\
    3
\end{outputblock}
\end{titledbox}

\begin{mycpptwocol}[]
#include <stdio.h>

int Find(int *p, int a)
{
    if (p[a] != a) {
        p[a] = Find(p, p[a]);
    }
    return p[a];
}

void Union(int *p, int *cnt, int a, int b)
{
    int ap = Find(p, a);
    int bp = Find(p, b);
    if (ap != bp) {
        p[ap] = bp;
        cnt[bp] += cnt[ap];
    }
}

int main()
{
    int n;
    int m;
    scanf("%d %d", &n, &m);
    int *p = (int *)calloc(sizeof(int), n + 1);
    int *cnt = (int *)calloc(sizeof(int), n + 1);
    for (int i = 0; i <= n; i++) {
        p[i] = i;
        cnt[i] = 1;
    }
    char op[3];
    int a;
    int b;
    while (m--) {
        scanf("%s", op);
        if (op[0] == 'C') {
            scanf("%d %d", &a, &b);
            Union(p, cnt, a, b);
        }
        if (op[0] == 'Q') {
            if (op[1] == '1') {
                scanf("%d %d", &a, &b);
                printf("%s\n", Find(p, a) == Find(p, b) ? "Yes" : "No");
            } else {
                scanf("%d", &a);
                printf("%d\n", cnt[Find(p, a)]);
            }
        }
    }
    free(cnt);
    free(p);
    return 0;
}
\end{mycpptwocol}

\begin{information}
    此处维护了一个size数组来标记集合中的元素个数,仅保证根节点的size是有意义的。
\end{information}

\subsection{AcWing 240. 食物链}

\section{堆}

用一维数组来模拟一个堆,起始坐标从\lstinline{1}开始,因为\lstinline{x}节点的左儿子是\lstinline{2x}右儿子是\lstinline{2x + 1},不能用0开始。几种操作的实现如下:

\begin{enumerate}
    \item 插入一个数:\lstinline{heap[++size] = x; up(size);}
    \item 求集合中的最小值:\lstinline{heap[1];}
    \item 删除最小值:\lstinline{heap[1] = heap[size]; size--; down(1);}
    \item 删除任意一个元素:\lstinline{heap[k] = heap[size]; size--; down(k); up(k);}
    \item 修改任意一个元素:\lstinline{heap[k] = x; down(k); up(k);}
\end{enumerate}

\subsection{AcWing 838. 堆排序}
此时只涉及到2和3两个操作即可,故仅需使用\lstinline{down}操作,无需\lstinline{up}操作。

\begin{titledbox}{AcWing 838. 堆排序}
输入一个长度为 $n$ 的整数数列,从小到大输出前 $m$ 小的数。

输入格式:

第一行包含整数 $n$ 和 $m$。第二行包含 $n$ 个整数,表示整数数列。

输出格式:
共一行,包含 $m$ 个整数,表示整数数列中前 $m$ 小的数。

数据范围:

$1 \le m \le n \le 10^5$,$1 \le 数列中元素 \le 10^9$

\begin{inputblock}
    5 3 \\
    4 5 1 3 2
\end{inputblock}
\begin{outputblock}
    1 2 3
\end{outputblock}
        
\end{titledbox}

\begin{mycpptwocol}[堆排序]
#include <stdio.h>
#include <stdlib.h>

#define N 100010
int h[N];
int size;

void down(int k)
{
    int t = k;
    if (2 * k <= size && h[2 * k] < h[t]) {
        t = 2 * k;
    }
    if (2 * k + 1 <= size && h[2 * k + 1] < h[t]) {
        t = 2 * k + 1;
    }
    if (t != k) {
        int tmp = h[t];
        h[t] = h[k];
        h[k] = tmp;
        down(t);
    }
}

int main()
{
    int n;
    int m;
    scanf("%d %d", &n, &m);
    for (int i = 1; i < n + 1; i++) {
        scanf("%d", &h[i]);
    }
    size = n;
    for (int i = n / 2; i > 0; i--) {
        down(i);
    }
    
    while (m--) {
        printf("%d ", h[1]);
        h[1] = h[size];
        size--;
        down(1);
    }
    return 0;
}
\end{mycpptwocol}

\subsection{AcWing 839. 模拟堆}

此处为了支持删除和修改任意一个元素,用了两个数组\lstinline{hp}和\lstinline{ph}来保存从堆(\lstinline{h}, heap)到插入的操作数(\lstinline{p}, pointer)之间的映射关系。

\begin{titledbox}{AcWing 839. 模拟堆}

维护一个集合,初始时集合为空,支持如下几种操作:

\begin{enumerate}
    \itemsep=-5pt
    \item I x,插入一个数 $x$;
    \item PM,输出当前集合中的最小值;
    \item DM,删除当前集合中的最小值(数据保证此时的最小值唯一);
    \item D k,删除第 $k$ 个插入的数;
    \item C k x,修改第 $k$ 个插入的数,将其变为 $x$;
\end{enumerate}

现在要进行 N 次操作,对于所有第 2 个操作,输出当前集合的最小值。

输入格式:

第一行包含整数 N。接下来 N 行,每行包含一个操作指令,操作指令为 I x,PM,DM,D k 或 C k x 中的一种。

输出格式:

对于每个输出指令 PM,输出一个结果,表示当前集合中的最小值。每个结果占一行。

数据范围

$1 \le N \le 10^5$, $−10^9 \le x \le 10^9$

数据保证合法。

\begin{inputblock}
    8 \\
    I -10 \\
    PM \\
    I -10 \\
    D 1 \\
    C 2 8 \\
    I 6 \\
    PM \\
    DM
\end{inputblock}
\begin{outputblock}
    -10 \\
    6
\end{outputblock}
\end{titledbox}

\begin{mycpptwocol}[可修改任意元素的堆]
#include <stdio.h>
#include <stdlib.h>
#include <string.h>

#define N 100010
int heap[N];
int hp[N];
int ph[N];
int size;

void Swap(int *a, int *b)
{
    int tmp = *a;
    *a = *b;
    *b = tmp;
}

void HeapSwap(int k, int t)
{
    Swap(&ph[hp[k]], &ph[hp[t]]);
    Swap(&hp[k], &hp[t]);
    Swap(&heap[k], &heap[t]);
}

void Up(int k)
{
    while (k / 2 > 0 && heap[k / 2] > heap[k]) {
        HeapSwap(k, k / 2);
        k >>= 1;
    }
}

void Down(int k)
{
    int t = k;
    if (2 * k <= size && heap[2 * k] < heap[t]) {
        t = 2 * k;
    }
    if (2 * k + 1 <= size && heap[2 * k + 1] < heap[t]) {
        t = 2 * k + 1;
    }
    if (k != t) {
        HeapSwap(k, t);
        Down(t);
    }
}

int main()
{
    int n;
    scanf("%d", &n);
    int idx = 0;
    while (n--) {
        char op[3];
        scanf("%s", op);
        int x;
        if (strcmp(op, "I") == 0) {
            scanf("%d", &x);
            heap[++size] = x;
            hp[size] = ++idx;
            ph[idx] = size;
            Up(size);
        }
        if (strcmp(op, "PM") == 0) {
            printf("%d\n", heap[1]);
        }
        if (strcmp(op, "DM") == 0) {
            HeapSwap(1, size);
            size--;
            Down(1);
        }
        if (strcmp(op, "D") == 0) {
            int k;
            scanf("%d", &k);
            k = ph[k];
            HeapSwap(k, size--);
            Down(k);
            Up(k);
        }
        if (strcmp(op, "C") == 0) {
            int k;
            scanf("%d %d", &k, &x);
            k = ph[k];
            heap[k] = x;
            Down(k);
            Up(k);
        }
    }
    return 0;
}
\end{mycpptwocol}

\section{哈希表}
\subsection{AcWing 840. 模拟散列表}

两种写法:

1. 开放寻址法

此法将从哈希后的那个位置开始查找,如果已经有了不是自己的人占用该位置则继续向后寻找,直到找到一个空位置,返回该位置。此法中Find函数将返回\lstinline{x}应该存放的位置。此法仅需开一个数组即可,通常来讲要开数据规模的两倍到三倍,以防buffer overflow。

2. 拉链法

此法将在哈希数组的每一个位置\lstinline{h[t]}处放置一个单链表,来存储所有哈希值为\lstinline{t}的数字。该单链表可以仅有一个数组表示,每个\lstinline{h[t]}中存储这一个数组中的不同下标表示单链表的头节点位置。

\subsection{AcWing 841. 字符串哈希}

注意:此处将字符串存储在下标为1的地方开始。是因为这里将字符串看作是\lstinline{P}进制数,如果从0开始会将同一个字符构成的字符串都算做相同的值,会有问题。

\begin{enumerate}
    \item 将字符串存储在下标为1的\lstinline{char}数组中
    \item 计算字符串的前缀数字,将他们哈希化,看作\lstinline{P}进制数,这里\lstinline{P = 131} 或者 \lstinline{P = 13331}按照经验来说,这两个值对字符串哈希会产生非常小的冲突可能。
    \item 对于Find函数,将计算字符串中从l到r的哈希值,通过前缀哈希表以及公式\lstinline{h[r] - h[l - 1] * p[r - l + 1]}其中\lstinline{p}数组存放了从$P^0 - P^N$的所有数字,方便查询。
\end{enumerate}